\fenicsappendix{Notation}{Notation}

%------------------------------------------------------------------------------

The following notation is used throughout this book.

\small
\linespread{1.2}
  \begin{longtable}{ccl}
    $A$ &--&
    the differential operator of the model $A(u) = f$ \\
    $A$ &--&
    the \emph{global tensor} with entries $\{A_i\}_{i\in\mathcal{I}}$ \\
    $A^0$ &--&
    the \emph{reference tensor} with entries $\{A^0_{i\alpha}\}_{i\in\mathcal{I}_K,\alpha\in\mathcal{A}}$ \\
    $\bar{A}^0$ &--&
    the matrix representation of the (flattened) reference tensor $A^0$ \\
    $A^K$ &--&
    the \emph{element tensor} with entries $\{A^K_i\}_{i\in\mathcal{I}_K}$ \\
    $a$ &--&
    the semilinear, multilinear or bilinear form \\
    $a_K$ &--&
    the local contribution to a multilinear form $a$ from $K$ \\
    $a^K$ &--&
    the vector representation of the (flattened) element tensor $A^K$ \\
    $\mathcal{A}$ &--&
    the set of \emph{secondary indices} \\
    $\mathcal{B}$ &--&
    the set of \emph{auxiliary indices} \\
    $e$ &--&
    the \emph{error}, $e = U - u$ \\
    $F_K$ &--&
    the mapping from $K_0$ to $K$ \\
    $G_K$ &--&
    the \emph{geometry tensor} with entries $\{G_K^{\alpha}\}_{\alpha\in\mathcal{A}}$ \\
    $g_K$ &--&
    the vector representation of the (flattened) geometry tensor $G_K$ \\
    $\mathcal{I}$ &--&
    the set $\prod_{j=1}^r [1,N^j]$ of indices for the global tensor $A$ \\
    $\mathcal{I}_K$ &--&
    the set $\prod_{j=1}^r [1,n_K^j]$ of indices for the element tensor
    $A^K$ (\emph{primary indices}) \\
    $\iota_K$ &--&
    the \emph{local-to-global mapping} from $\mathcal{N}_K$ to $\mathcal{N}$ \\
    $\hat{\iota}_K$ &--&
    the local-to-global mapping from $\hat{\mathcal{N}}_K$ to $\hat{\mathcal{N}}$ \\
    $\iota_K^j$ &--&
    the local-to-global mapping from $\mathcal{N}_K^j$ to $\mathcal{N}^j$ \\
    $K$ &--&
    a \emph{cell} in the mesh~$\mathcal{T}$ \\
    $K_0$ &--&
    the \emph{reference cell} \\
    $L$ &--&
    the linear form (functional) on $\hat{V}$ or $\hat{V}_h$ \\
    $m$ &--&
    the number of discrete function spaces used in the definition of~$a$ \\
    $N$ &--&
    the dimension of $\hat{V}_h$ and $V_h$ \\
    $N^j$ &--&
    the dimension $V_h^j$ \\
    $N_q$ &--&
    the number of quadrature points on a cell \\
    $n_0$ &--&
    the dimension of $\mathcal{P}_0$ \\
    $n_K$ &--&
    the dimension of $\mathcal{P}_K$ \\
    $\hat{n}_K$ &--&
    the dimension of $\hat{\mathcal{P}}_K$ \\
    $n_K^j$ &--&
    the dimension of $\mathcal{P}_K^j$ \\
    $\mathcal{N}$ &--&
    the set of global nodes on $V_h$ \\
    $\hat{\mathcal{N}}$ &--&
    the set of global nodes on $\hat{V}_h$ \\
    $\mathcal{N}^j$ &--&
    the set of global nodes on $V_h^j$ \\
    $\mathcal{N}_0$ &--&
    the set of local nodes on $\mathcal{P}_0$ \\
    $\mathcal{N}_K$ &--&
    the set of local nodes on $\mathcal{P}_K$ \\
    $\hat{\mathcal{N}}_K$ &--&
    the set of local nodes on $\hat{\mathcal{P}}_K$ \\
    $\mathcal{N}_K^j$ &--&
    the set of local nodes on ${\mathcal{P}_K^j}$ \\
    $\nu^0_i$ &--&
    a \emph{node} on $\mathcal{P}_0$ \\
    $\nu^K_i$ &--&
    a node on $\mathcal{P}_K$ \\
    $\hat{\nu}^K_i$ &--&
    a node on $\hat{\mathcal{P}}_K$ \\
    $\nu^{K,j}_i$ &--&
    a node on $\mathcal{P}_K^j$ \\
    $\mathcal{P}_0$ &--&
    the function space on $K_0$ for $V_h$ \\
    $\hat{\mathcal{P}}_0$ &--&
    the function space on $K_0$ for $\hat{V}_h$ \\
    $\mathcal{P}_0^j$ &--&
    the function space on $K_0$ for $V_h^j$ \\
    $\mathcal{P}_K$ &--&
    the local function space on $K$ for $V_h$ \\
    $\hat{\mathcal{P}}_K$ &--&
    the local function space on $K$ for $\hat{V}_h$ \\
    $\mathcal{P}_K^j$ &--&
    the local function space on $K$ for $V_h^j$ \\
    $P_q(K)$ &--&
    the space of polynomials of degree $\leq q$ on $K$ \\
    $\overline{\mathcal{P}}_K$ &--&
    the local function space on $K$ generated by $\{\mathcal{P}_K^j\}_{j=1}^m$ \\
    $R$ &--&
    the \emph{residual}, $R(U) = A(U) - f$ \\
    $r$ &--&
    the arity of the multilinear form $a$ (the rank of $A$ and $A^K$) \\
    $U$ &--&
    the discrete approximate solution, $U \approx u$ \\
    $(U_i)$ &--&
    the vector of expansion coefficients for $U = \sum_{i=1}^N U_i \phi_i$ \\
    $u$ &--&
    the exact solution of the given model $A(u) = f$ \\
    $V$ &--&
    the space of trial functions on $\Omega$ (the trial space) \\
    $\hat{V}$ &--&
    the space of test functions on $\Omega$ (the test space) \\
    $V_h$ &--&
    the space of discrete trial functions on $\Omega$ (the discrete trial space) \\
    $\hat{V}_h$ &--&
    the space of discrete test functions on $\Omega$ (the discrete test space) \\
    $V_h^j$ &--&
    a discrete function space on $\Omega$ \\  
    $|V|$ &--&
    the dimension of a vector space $V$ \\
    $\Phi_i$ &--&
    a basis function in $\mathcal{P}_0$ \\
    $\hat{\Phi}_i$ &--&
    a basis function in $\hat{\mathcal{P}}_0$ \\
    $\Phi_i^j$ &--&
    a basis function in $\mathcal{P}_0^j$ \\
    $\phi_i$ &--&
    a basis function in $V_h$ \\
    $\hat{\phi}_i$ &--&
    a basis function in $\hat{V}_h$ \\
    $\phi_i^j$ &--&
    a basis function in $V_h^j$ \\
    $\phi_i^K$ &--&
    a basis function in $\mathcal{P}_K$ \\
    $\hat{\phi}_i^K$ &--&
    a basis function in $\hat{\mathcal{P}}_K$ \\
    $\phi_i^{K,j}$ &--&
    a basis function in $\mathcal{P}_K^j$ \\
    $z$ &--&
    the \emph{dual solution} \\
    $\mathcal{T}$ &--&
    the \emph{mesh} \\
    $\Omega$ &--&
    a bounded domain in~$\R^d$ \\
  \end{longtable}
\linespread{1.0}
