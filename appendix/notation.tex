\fenicsappendix{Notation}{Notation}

%------------------------------------------------------------------------------

The following notation is used throughout this book.

\small
\linespread{1.2}
  \begin{longtable}{ccl}
    $A$ &--&
    the \emph{global tensor} with entries $\{A_i\}_{i\in\mathcal{I}}$ \\
    $A^K$ &--&
    the \emph{element tensor} with entries $\{A^K_i\}_{i\in\mathcal{I}_K}$ \\
    $A^0$ &--&
    the \emph{reference tensor} with entries $\{A^0_{i\alpha}\}_{i\in\mathcal{I}_K,\alpha\in\mathcal{A}}$ \\
    $a$ &--&
    a multilinear form \\
    $a_K$ &--&
    the local contribution to a multilinear form $a$ from a cell $K$ \\
    $\mathcal{A}$ &--&
    the set of \emph{secondary indices} \\
    $\mathcal{B}$ &--&
    the set of \emph{auxiliary indices} \\
    $e$ &--&
    the \emph{error}, $e = u_h - u$ \\
    $F_K$ &--&
    the mapping from the reference cell $K_0$ to $K$ \\
    $G_K$ &--&
    the \emph{geometry tensor} with entries $\{G_K^{\alpha}\}_{\alpha\in\mathcal{A}}$ \\
    $\mathcal{I}$ &--&
    the set $\prod_{j=1}^{\rho} [1,N^j]$ of indices for the global tensor $A$ \\
    $\mathcal{I}_K$ &--&
    the set $\prod_{j=1}^{\rho} [1,n_K^j]$ of indices for the element tensor $A^K$ (\emph{primary indices}) \\
    $\iota_K$ &--&
    the \emph{local-to-global mapping} from $[1,n_K]$ to $[1,N]$ \\
    $K$ &--&
    a \emph{cell} in the mesh~$\mathcal{T}$ \\
    $K_0$ &--&
    the \emph{reference cell} \\
    $L$ &--&
    a linear form (functional) on $\hat{V}$ or $\hat{V}_h$ \\
    $\mathcal{L}$ &--&
    the degrees of freedom (linear functionals) on $V_h$ \\
    $\mathcal{L}_K$ &--&
    the degrees of freedom (linear functionals) on $\mathcal{P}_K$ \\
    $\mathcal{L}_0$ &--&
    the degrees of freedom (linear functionals) on $\mathcal{P}_0$ \\
    $N$ &--&
    the dimension of $\hat{V}_h$ and $V_h$ \\
    $n_K$ &--&
    the dimension of $\mathcal{P}_K$ \\
    $\ell_i$ &--&
    a degree of freedom (linear functional) on $V_h$ \\
    $\ell^K_i$ &--&
    a degree of freedom (linear functional) on $\mathcal{P}_K$ \\
    $\ell^0_i$ &--&
    a degree of freedom (linear functional) on $\mathcal{P}_0$ \\
    $\mathcal{P}_K$ &--&
    the local function space on a cell $K$ \\
    $\mathcal{P}_0$ &--&
    the local function space on the reference cell $K_0$ \\
    $P_q(K)$ &--&
    the space of polynomials of degree $\leq q$ on $K$ \\
    $r$ &--&
    the (weak) residual, $r(v) = a(v, u_h) - L(v)$ or $r(v) = F(u_h; v)$ \\
    $u_h$ &--&
    the finite element solution, $u_h \in V_h$ \\
    $U$ &--&
    the vector of degrees of freedom for $u_h = \sum_{i=1}^N U_i \phi_i$ \\
    $u$ &--&
    the exact solution of a variational problem, $u \in V$ \\
    $\hat{V}$ &--&
    the test space \\
    $V$ &--&
    the trial space \\
    $\hat{V}^*$ &--&
    the dual test space, $\hat{V}^* = V_0$ \\
    $V^*$ &--&
    the dual trial space, $V^* = \hat{V}$ \\
    $\hat{V}_h$ &--&
    the discrete test space \\
    $V_h$ &--&
    the discrete trial space \\
    $\phi_i$ &--&
    a basis function in $V_h$ \\
    $\hat{\phi}_i$ &--&
    a basis function in $\hat{V}_h$ \\
    $\phi_i^K$ &--&
    a basis function in $\mathcal{P}_K$ \\
    $\Phi_i$ &--&
    a basis function in $\mathcal{P}_0$ \\
    $z$ &--&
    the \emph{dual solution}, $z \in V^*$ \\
    $\mathcal{T}$ &--&
    the \emph{mesh}, $\mathcal{T} = \{K\}$ \\
    $\Omega$ &--&
    a bounded domain in~$\R^d$ \\
  \end{longtable}
\linespread{1.0}
