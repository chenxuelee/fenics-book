\begingroup

\setcounter{chapter}{18}
\setcounter{chpnum}{18}

\fenicschapter{Lessons learned in mixed language programming}
              {Lessons learned in mixed language programming}
              {Lessons learned in mixed language programming}
              {Johan Hake and Kent-Andre Mardal}
              {mardal-2}

This chapter describes decisions made and lessons learned in the
implementation of the Python interface of DOLFIN. The chapter is quite
technical, since we aim at giving the reader a thorough understanding
of the implementation of the DOLFIN Python interface.

\hyphenation{Langtangen}

\vspace*{6.5pt}
\section{Background}

Python has over the last decade become an established platform for
scientific computing. Widely used scientific software such as,
e.g., \citet{www:petsc}, \citet{www:hypre}, \citet{www:trilinos},
\citet{www:vtk}, \citet{www:vmtk}, \ginac~\citep{BauerFrinkKreckel2000}
have all been equipped with Python interfaces. The \fenics
packages \ferari, \fiat , \ffc, \ufl, \viper, as well as other
packages such as \sympy\citep{CertikSeoanePetersonEtAl2009},
\scipy\citep{JonesOliphantPetersonEtAl2009} are pure Python packages.
The \dolfin library has both a C++ and a Python user-interface. Python
makes application building on top of \dolfin more user friendly,
but the Python interface also introduces additional complexity and
new problems. We assume that the reader has basic knowledge of
both C++ and Python. A suitable textbook on Python for scientific
computing is \citet{Langtangen2008}, which covers both Python and its
C interface.  SWIG, which is the software we use to wrap \dolfin, is
well documented and we refer to the user manual that can be found on its
web page~\citep{www:swig}. Finally, we refer to \citet{Langtangen2003b}
and \citet{SalaSpotzHeroux2008} for a description of how SWIG can be
used to generate Python interfaces for other packages such as Diffpack
and\break Trilinos.

\vspace*{6.5pt}
\section{Using SWIG}
\index{SWIG}

Python and C++ are two very different languages, while Python is
user--friendly and flexible, C++ is very efficient.  To combine the
strengths of the two languages, it has become common to equip C++ (or
FORTRAN/C) libraries with Python interfaces.  Such interfaces must comply
with the \citet{www:python-capi}.  Writing such interfaces, often called
wrapper code, is quite involved.  Therefore, a number of wrapper code
generators have been developed in the recent years, some examples are
\citet{Peterson}, \citet{SIP}, \citet{Siloon}, and \citet{www:swig}.
SWIG has been used to create the DOLFIN Python interface, and will
therefore be the focus in this chapter. SWIG is a mature wrapper code
generator that supports many languages and is extensively documented.

\pagebreak%

\subsection{Basic SWIG}

To get a basic understanding of SWIG, we consider an implementation of an
array class.  Let the array class be defined in \emp{Array.h} as follows:\vspace*{3pt}

\inputcpp{Array.h}

\vspace*{5pt}

\noindent A first attempt to make the
Array accessible in Python using SWIG, is to write a SWIG interface
file \emp{Array\_1.i}.\vspace*{3pt}


\inputswig{Array_1.i}

\vspace*{5pt}

\noindent Here, we specify the name of the Python module: \emp{Array}; the code
that should be inlined in the wrapper code directly (the declarations):
\emp{\#include "Array.h"}; and the code SWIG should parse to create the
wrapper code: \emp{\%include "Array.h"} (definitions). The following
command shows how to run SWIG to produce the wrapper code:\vspace*{3pt}
\begin{bash}
swig -python -c++ -I. -O Array_1.i
\end{bash}

\vspace*{5pt}

\noindent The command generates two files: \emp{Array.py} and
\emp{Array\_}\emp{wrap.cxx}. The file \emp{Array\_}\emp{wrap.cxx}
contains C code that defines the Python interface of Array.  After
\emp{Array\_wrap.cxx} is compiled into a shared library, it can be
imported into Python.  The file \emp{Array.py} is written in pure
Python. It imports the shared library and also adds some functionality
to the wrapped module.  The reader should be able to recognize the Python
class \emp{Array} at the end of the \emp{Array.py} file.

The following \citet{www:distutils} file (\emp{setup.py})
executes the SWIG command above and compiles and links the
source code and the generated wrapper code into a shared library.\vspace*{3pt}


\inputpython{setup.py}

\vspace*{5pt}

\noindent Build and install the
module in the current working directory with the command:\vspace*{3pt}
\begin{bash}
python setup.py install --install-lib=.
\end{bash}

\vspace*{5pt}

\noindent The Python proxy class resembles the C++ class in many ways. Simple
methods such as \emp{dim()} and \emp{norm()} will be wrapped correctly
to Python, since SWIG maps \emp{int} and \emp{double} arguments to the
corresponding Python types through built-in typemaps.  However, a number
of issues appear:
\begin{enumerate}
\item the \emp{operator[]} does not work;

\item the \emp{operator+=} returns a new Python object (with different
\emp{id});

\item printing does not use the \emp{std::ostream \& operator<{}<};

\item the \emp{Array(int n\_, double* a\_);} constructor is not working
properly.
\end{enumerate}
We see that a number of different problems arise even in such a simple
example. Fortunately, these problems are fairly common, and general
solutions can be implemented quite easily. In the following, we will go
through each of the above issues. The example code with the solutions
proposed in the following can be found in \emp{Array\_2.i}.

\vspace*{3pt}
\subsection{The \emp{operator[]}}

In C++, the subscripting \emp{operator[]} is used to implement
both set and get operators. It is possible to distinguish the set
operator from the get operator using \emp{const}, but this is not
required.  In Python, subscripting is performed with the two special
methods: \emp{\_\_setitem\_\_} and \emp{\_\_getitem\_\_}.  Since,
the mapping between the Python operators (\emp{\_\_setitem\_\_} and
\emp{\_\_getitem\_\_}) and the C++ operators \emp{operator[]} may be
ambiguous, SWIG currently ignores these operators.  To implement the
operators properly, also in future versions of SWIG, we ignore both
version of the \emp{operator[]} with\vspace*{3pt}
\begin{swigcode}
%ignore Array::operator[];
\end{swigcode}

\pagebreak

\noindent and extend the interface of the generated C++ code with the auxiliary
\emp{\_\_setitem\_\_} and \emp{\_\_getitem\_\_} methods:\vspace*{2pt}
\begin{swigcode}
%extend Array {
double __getitem__(int i) {
  return (*self)[i];
}

void __setitem__(int i, double v) {
  (*self)[i] = v;
}
...
};
\end{swigcode}

\vspace*{3pt}

\noindent Note that all SWIG directives start with \emp{"\%"}. Furthermore, the
access to the actual instance is provided by the \emp{self} pointer,
which in this case is a C++ pointer that points to an \emp{Array}
instance. The pointer is comparable to the \emp{this} pointer in a C++
class, but only the public attributes are available.

\vspace*{3pt}
\subsection{\emp{operator +=}}

The second problem is related to SWIG and garbage collection in Python.
Python features garbage collection, which means that a user should not
be concerned with the destruction of objects. The mechanism is based
on reference counting; that is, when no more references are pointing
to an object, the object is destroyed. The SWIG generated Python
module consists of a small Python layer that defines the interface
to the underlying C++ object. An instance of a SWIG generated class
therefore keeps a reference to the underlying C++ object. The default
behavior is that the C++ object is destroyed together with the Python
object. This behavior is not consistent with the \emp{operator
+=} returning a new object, which is illustrated by the generated
segmentation fault in the following example (see \emp{segfault\_test.py}):\vspace*{2pt}

\inputpython{segfault_test.py}

\vspace*{3pt}

This script produces the following output:\vspace*{3pt}
\begin{python}
id(a): 3085535980
id(b): 3085535980
id(b): 3085536492
Segmentation fault
\end{python}

\vspace*{3pt}

\noindent The script causes a segmentation fault because the underlying C++ object
is destroyed after the call to \emp{add()}. When the last \emp{a+=a} is
performed the underlying C++ object is already destroyed. This happens
because the SWIG generated \emp{\_\_iadd\_\_} method returns a new
Python object. This is illustrated by the different values obtained from
the \emp{id()} function\footnote{The \emp{id} function returns a unique
integer identifying the object.}. The last two calls to \emp{id(b)} return
different numbers, which means that a new Python object is returned by
the SWIG generated \emp{\_\_iadd\_\_} method. The second \emp{b} object
is local in the \emp{add} function and is therefore deleted together
with the underlying C++ object when \emp{add} has finished.

The memory problem can be solved by extending the interface with an
\emp{\_add} method and implementing our own \emp{\_\_iadd\_\_} method
in terms of \emp{\_add}, using the \emp{\%extend} directive:\vspace*{1.5pt}
\begin{swigcode}
%extend Array {
...
  void _add(const Array& a){
    (*self) += a;
  }

  %pythoncode %{
    def __iadd__(self,a):
      self._add(a)
      return self
  %}
...
};
\end{swigcode}

\vspace*{3pt}

\noindent The above script will now report the same \emp{id} for all objects.
No objects are created or deleted, and segmentation fault is avoided.

\vspace*{3pt}
\subsection{\emp{std::ostream \& operator<{}<}}%>>

In C++, shift operators such as \emp{operator<{}<} are typically used
to implement I/O, while in Python the \emp{\_\_str\_\_} method is used.
Therefore, SWIG ignores the shift operator, as it is likely not to perform
as intended.  However, we can again use the \emp{\%extend} directive to
make this operator available from Python by adding a \emp{\_\_str\_\_}
method.\vspace*{1.5pt}
\begin{swigcode}
%include <std_string.i>

%extend Array {
...
  std::string __str__() {
    std::ostringstream s;
    s << (*self);
    return s.str();
  }
};
\end{swigcode}

\vspace*{3pt}

\noindent This method uses the \emp{operator<{}<} %>> to pipe the stream
representation of the array to a \emp{std::ostringstream} and then
returns a \emp{std::string} representation of the stream.  Note that we
need to include \emp{std\_string.i} in the \emp{Array\_2.i}.  In Python,
we can then call \emp{print} on an instance of \emp{Array}.

\vspace*{3pt}
\subsection{The constructor: \emp{Array(int n\_, double* a\_);}}

The fourth problem is related to pointer handling in C/C++ and SWIG. From
the constructor signature alone, it is not clear whether \emp{double*
a\_} points to a single value or to the first element of an array.
Therefore, SWIG takes a conservative approach and handles pointers as
pointers to single values. In our example \emp{double* a\_} points to
the first element of an array of length \emp{n}, and SWIG erroneously
generates code for passing an \emp{int} and a \emp{double} to the method.

As a remedy, SWIG provides the \emp{typemap} concept to enable mappings
between C/C++ and Python types. The following code, explained in detail
below, demonstrates how to map a \numpy array to the \emp{(int n\_,
double* a\_)} arguments in the constructor.\vspace*{1.5pt}
\begin{swigcode}
%typemap(in) (int n_, double* a_){
  if (!PyArray_Check($input)) {
    PyErr_SetString(PyExc_TypeError, "Not a NumPy array");
    return NULL; ;
  }
  PyArrayObject* pyarray = reinterpret_cast<PyArrayObject*>($input);
  if (!(PyArray_TYPE(pyarray) == NPY_DOUBLE)) {
    PyErr_SetString(PyExc_TypeError, "Not a NumPy array of doubles");
    return NULL; ;
  }
  $1 = PyArray_DIM(pyarray,0);
  $2 = static_cast<double*>(PyArray_DATA(pyarray));
}
\end{swigcode}

\vspace*{3pt}

\noindent The first line specifies that the typemap should be applied to the
input \emp{(in)} arguments of operators, functions, and methods with the
\emp{int n\_,double* a\_} arguments in the signature.  The \$ prefixed
variables are used to map input and output variables in the typemap;
that is, the variables \$1 and \$2 map to the first and second output C
arguments of the typemap, \emp{n\_} and \emp{a\_}, while \emp{\$input}
maps to the Python input.

In the next three lines, we check that the input Python object is a \numpy
array, and raise an exception if not.  Note that any Python C-API function
that returns \emp{NULL} tells the Python interpreter that an exception
has occurred. Python will then raise an error, with the error message set
by the \emp{PyErr\_SetString} function. Next, we cast the Python object
pointer to a \numpy array pointer and check that the data type of the
\numpy array is correct; that is, that it contains doubles. Then, we
acquire the data from the \numpy array and assign the two input variables.

Overloading operators, functions and methods is not possible in Python.
Instead, Python dynamically determines what code to call, a process
which is called dynamic dispatch.  To generate proper wrapper code,
SWIG relies on \emp{\%typecheck} directives to resolve the overloading.
A suitable type check for our example typemap is:\vspace*{1.5pt}
\begin{swigcode}
%typecheck(SWIG_TYPECHECK_DOUBLE_ARRAY) (int n_, double* a_) {
   $1 = PyArray_Check($input) ? 1 : 0;
 }
\end{swigcode}

\vspace*{3pt}

\noindent Here, \emp{SWIG\_TYPECHECK\_DOUBLE\_ARRAY} is a \emp{typedef} for the
priority number assigned for arrays of doubles. The type check should
return 1 if the Python object \emp{\$input} has the correct type, and
0 otherwise.

\vspace*{6pt}
\section{SWIG and the DOLFIN Python interface}

To make the DOLFIN Python interface more \textit{Pythonic}, we have made a number of
specializations, along the lines mentioned above, that we will now
go through. But let us start with the overall picture.  The interface
files resides in the \emp{dolfin/swig} directory, and are organized into
\textit{i)} global files, that apply to the entire \dolfin library,
and \textit{ii)} kernel module files that apply to specific \dolfin
modules. The latter files are divided into \emp{$\ldots$\_pre.i}
and \emp{$\ldots$\_post.i} files, which are applied before and after
the inclusion of the header files of the particular kernel module,
respectively.  The kernel modules, as seen in \emp{kernel\_module.i},
mirror the directory structure of \dolfin: \emp{common}, \emp{parameters},
\emp{la}, \emp{mesh} and so forth. The global interface files are all
included in \emp{dolfin.i}, the main SWIG interface file. The kernel
module interface files are included, together with the C++ header files,
in the automatically generated \emp{kernel\_modules.i} file.

The following sections deal with the main interface file of \emp{dolfin.i}
and address the global interface files. Then we will address some issues
in the module specific interface files.

\vspace*{3pt}
\subsection{\emp{dolfin.i} and the \emp{cpp} module}

The file \emp{dolfin.i} starts by defining the name of the generated
Python module.\vspace*{2pt}
\begin{swigcode}
%module(package="dolfin", directors="1") cpp
\end{swigcode}

\vspace*{3pt}

\noindent This statement tells SWIG to create a module called \emp{cpp} that
resides in the package \dolfin{}. We also enable the use of
directors.  This is required to be able to subclass \dolfin classes
in Python, an issue that will be discussed below.  By naming the generated
extension module \emp{cpp}, and including it in the \dolfin{} Python
package, we hide the generated interface into a submodule of \dolfin{}; the
\emp{dolfin.cpp} module.  The \dolfin{} module then imports the needed
classes and functions from \emp{dolfin.cpp} in the \emp{\_\_init\_\_.py}
file along with additional pure Python classes and functions.

The next two blocks of \emp{dolfin.i} define code that will be inserted
into the SWIG generated C++ wrapper file.\vspace*{2pt}
\begin{swigcode}
%{
#include <dolfin/dolfin.h>
#define PY_ARRAY_UNIQUE_SYMBOL PyDOLFIN
#include <numpy/arrayobject.h>
%}

%init%{
import_array();
%}
\end{swigcode}

\vspace*{3pt}

\noindent SWIG inserts code that resides in a \emp{\%\{$\ldots$\%\}} block,
verbatim at the top of the generated C++ wrapper file. Note that
\emp{\%\{$\ldots$\%\}} is short for \emp{\%header\%\{$\ldots$\%\}}. Hence,
the first block of code is similar to the include statements you
would put in a standard C++ program. The code in the second block,
\emp{\%init\%\{$\ldots$\}\%}, is inserted in the code where the
Python module is initialized. A typical example of such a function is
\emp{import\_array()}, which initializes the \numpy module. SWIG provides
several such statements, each inserting code verbatim into the wrapper
file at different positions.

\vspace*{3pt}
\subsection{Reference counting using \emp{shared\_ptr}}

\index{reference counting}
\index{shared pointer}

In the previous example dealing with \emp{operator+=}, we saw that
it is important to prevent premature destruction of underlying C++
objects. A nice feature of SWIG is that we can declare that a wrapped
class shall store the underlying C++ object \pagebreak using a shared pointer
(\emp{shared\_ptr}), instead of a raw pointer. By doing so, the
underlying C++ object is not explicitly deleted when the reference
count of the Python object reach zero, instead the reference count on
the \emp{shared\_ptr} is\break decreased.

Shared pointers are provided by the \emp{boost\_shared\_ptr.i} file.
This file defines the directive: \emp{\%shared\_ptr}. The directive
must be used for each class we want shared pointers for. In DOLFIN
this is done in the \emp{shared\_ptr\_classes.i} file. Note that when
the \emp{\%shared\_ptr} directive is called, typemaps for passing a
\emp{shared\_ptr} stored object to method that expects a reference or
a pointer is also defined. This means that the typemap pass a
de-referenced \emp{shared\_ptr} to the function. This behavior can
lead to unintentional trouble because the \emp{shared\_ptr} mechanism
is circumvented.

In \dolfin, instances of some crucial classes are stored internally
with \emp{shared\_ptr}s. These classes also uses \emp{shared\_ptr}
in the Python interface. When objects of these classes are passed
as arguments to methods or constructors in C++, two versions are
needed: a \emp{shared\_ptr} and a reference version. The following
code snippet illustrates two constructors of \emp{Function}, each
taking a \emp{FunctionSpace} as an argument\footnote{Instances of
\emp{FunctionSpace} are internally stored using \emp{shared\_ptr} in
the \dolfin C++ library.}:\vspace*{1.5pt}
\begin{c++}
/// Create function on given function space
explicit Function(const FunctionSpace& V);

/// Create function on given function space (shared data)
explicit Function(boost::shared_ptr<const FunctionSpace> V);
\end{c++}

\vspace*{3pt}

\noindent Instances of \emp{FunctionSpace} in DOLFIN are stored
using \emp{shared\_ptr}. Hence, we want SWIG to use the second
constructor. However, SWIG generates de-reference typemaps for the
first constructor. So when a \emp{Function} is instantiated with a
\emp{FunctionSpace}, SWIG will unfortunately pick the first constructor
and the \emp{FunctionSpace} is passed without increasing the reference
count of the \emp{shared\_ptr}.  This undermines the whole concept
of \emp{shared\_ptr}. To prevent this faulty behavior, we ignore the
reference constructor (see \emp{function\_pre.i}):\vspace*{1.5pt}
\begin{swigcode}
%ignore dolfin::Function::Function(const FunctionSpace&);
\end{swigcode}

\vspace*{3pt}
\subsection{Typemaps}
\index{typemap}

Most types in the \emp{kernel\_module.i} file are wrapped nicely with
SWIG. However, as in the \emp{Array} example above, there is need for
typemaps, for instance to handle \numpy arrays. In \emp{dolfin.i} we
include different types of global typemaps. They are called global
because they are defined for the whole DOLFIN library. A typemap is
not global if it is included in the kernel specific interface files,
see below. Here we present some of the global typemaps defined in the
files included in\break \emp{dolfin.i}.

In \emp{typemaps.i} are typemaps and so called SWIG fragments (explained later), defined
for the three primitive types:\emp{int}, \emp{dolfin::uint} and
\emp{double}.

\pagebreak

The simplest typemap is an out-typemap for \emp{dolfin::uint}, a
typedef of \emp{unsigned int}. This typemap is needed since Python
does not have an equivalent of an \emp{unsigned int} type:\vspace*{3pt}
\begin{swigcode}
%typemap(out, fragment=SWIG_From_frag(unsigned int)) unsigned int
{
  // Typemap unsigned int
  $result = SWIG_From(unsigned int)($1);
}
\end{swigcode}

\vspace*{5pt}

\noindent Here we specify that a function returning an \emp{unsigned int} will
use the SWIG provided type\break conversion macro: \emp{SWIG\_From(unsigned
  int)(arg)} to convert the \emp{unsigned int} to a Python
\emp{int}. The macro is not provided by default in SWIG. We therefore
need to specify that SWIG includes the definition of the macro in the
wrapper file by using the \emp{fragment} argument to the \emp{typemap}\break directive.

The next typemap is an in-typemap for \emp{unsigned int}.\vspace*{3pt}
\begin{swigcode}
%typemap(in, fragment="Py_convert_uint") unsigned int
{
  if (!Py_convert_uint($input, $1))
    SWIG_exception(SWIG_TypeError, "expected positive 'int' for argument $argnum");
}
\end{swigcode}
%$ to fool emacs highlight...

\vspace*{5pt}

\noindent This typemap is almost as simple as the corresponding out typemap. It
employs the fragment provided function: \emp{Py\_convert\_uint}, to do
the type check and the Python to C++ conversion. If the conversion is
not successful it will return \emp{false} and a Python exception is
raised. The built in SWIG function, \emp{SWIG\_exception} is used to
raise the Python exception. These predefined SWIG exceptions are
defined in the \emp{exception.i} file, included in
\emp{dolfin.i}. Note that SWIG expands the \emp{\$argnum} variable to
the argument number for which the \emp{dolfin::uint} typemap is used
for. Including this number in the string creates more understandable
error message. The \emp{Py\_convert\_uint} fragment is defined in the
same file and looks like:\vspace*{3pt}
\begin{swigcode}
%fragment("Py_convert_uint", "header", fragment="PyInteger_Check") {
  SWIGINTERNINLINE bool Py_convert_uint(PyObject* in, dolfin::uint& value)
  {
    if (!(PyInteger_Check(in) && PyInt_AS_LONG(in)>=0))
      return false;
    value = static_cast<dolfin::uint>(PyInt_AS_LONG(in));
    return true;
  }
}
\end{swigcode}
%$ to fool emacs highlight...

\vspace*{5pt}

\noindent The first string in the \emp{fragment} declaration is the name of the
\emp{fragment} and the second string defines where the code should be
inserted. Here we insert it into the \emp{"header"} section of the
generated code. This is similar to the code inserted using
\emp{\%\{$\ldots$\%\}} above. Here we also rely on another fragment
provided function: \emp{PyInteger\_Check}, which we provide as a
substitute to the built in \emp{PyInt\_Check} function. The reason for
that is that \emp{PyInt\_Check} in Python2.6 can not be combined with
\numpy \emp{Int}.

In \emp{numpy\_typemaps.i} are typemaps for C-arrays of primitive
types: \emp{double}, \emp{int} and \emp{dolfin::uint} defined. As in
the \emp{Array} example in the previous section, these typemaps let a
user pass a \numpy array of the corresponding type as arguments to
functions, methods, and operators. Instead of writing one typemap for
each primitive type we define a SWIG macro, which instantiates a
typemap for a particular primitive type when it is called. Some of these
typemaps are used frequently and can therefore produce a lot of
code. To avoid code bloat most of the typemap code is placed in the
function \emp{convert\_numpy\_to\_}\-\emp{array\_no\_check(TYPE\_NAME)}, which
is called by the actual typemap. The code is defined within a
\emp{fragment} directive, which means that a typemap can make use of
that code by adding the name of the fragment as an argument in the
typemap definition. The entire macro reads:
\begin{swigcode}
%define UNSAFE_NUMPY_TYPEMAPS(TYPE,TYPE_UPPER,NUMPY_TYPE,TYPE_NAME,DESCR)

%fragment(convert_numpy_to_array_no_check(TYPE_NAME), "header") {
  // Typemap function (Reducing wrapper code size)
  SWIGINTERN bool convert_numpy_to_array_no_check_ ## TYPE_NAME(PyObject* input, TYPE*& ret)
  {
    if (PyArray_ISCONTIGUOUS(xa) && PyArray_TYPE(xa) == NUMPY_TYPE)
    {
      PyArrayObject *xa = reinterpret_cast<PyArrayObject*>(input);
      if ( PyArray_TYPE(xa) == NUMPY_TYPE )
      {
        ret  = static_cast<TYPE*>(PyArray_DATA(xa));
        return true;
      }
    }
    PyErr_SetString(PyExc_TypeError,"contiguous numpy array of 'TYPE_NAME' expected. Make sure that the numpy array is contiguous and uses dtype='DESCR'.");
    return false;
  }
}

// The typecheck
% typecheck(SWIG_TYPECHECK_ ## TYPE_UPPER ## _ARRAY) TYPE * {
    $1 = PyArray_Check($input) ? 1 : 0;
}

// The typemap
%typemap(in, fragment=convert_numpy_to_array_no_check(TYPE_NAME)) TYPE * {
if (!convert_numpy_to_array_no_check_ ## TYPE_NAME($input,$1))
    return NULL;
}
\end{swigcode}
The first line defines the signature of the macro. The macro is called
using 5 arguments:
\begin{enumerate}
\item \emp{TYPE} is the name of the primitive type. Examples are
\emp{dolfin::uint} and \emp{double}.

\item \emp{TYPE\_UPPER} is the name of the type check name that SWIG uses. Examples
are \emp{INT32} and \emp{DOUBLE}.

\item \emp{NUMPY\_TYPE} is the name of the \numpy type. Examples are
\emp{NPY\_UINT} and \emp{NPY\_DOUBLE}.

\item \emp{TYPE\_NAME} is a short type name used in exception string.
Examples are \emp{uint} and \emp{double}.

\item \emp{DESCR} is a description character used in \numpy to describe the
type. Examples are \emp{'I'} and~\emp{'d'}.
\end{enumerate}

\pagebreak

\noindent We can then call the macro to instantiate the typemaps and type checks.
\begin{swigcode}
UNSAFE_NUMPY_TYPEMAPS(dolfin::uint,INT32,NPY_UINT,uint,I)
UNSAFE_NUMPY_TYPEMAPS(double,DOUBLE,NPY_DOUBLE,double,d)
\end{swigcode}
Here, we have instantiated the typemap for a \emp{dolfin::uint} and a
\emp{double} array. The above typemap does not check the length of the
handed \numpy array and is therefore unsafe. Corresponding safe
typemaps can also be found in \emp{numpy\_typemaps.i}. The conversion
function included in the fragment declaration
\begin{swigcode}
SWIGINTERN bool convert_numpy_to_array_no_check_ ## TYPE_NAME(PyObject* input, TYPE*& ret)
\end{swigcode}
takes a pointer to a \emp{PyObject} as input. This function returns
\emp{true} if the conversion is successful and \emp{false}
otherwise. The converted array will be returned by the \emp{TYPE*\&
  ret} argument.  Finally, the \emp{\%apply TYPE* \{TYPE* \_array\}}
directive means that we want the typemap to apply to any argument of
type \emp{TYPE*} with argument name \emp{\_array}. In this way we copy
the already defined typemap to an C-array argument with the name
\emp{\_array}.  Note that \emp{\#\# TYPE\_NAME} is a SWIG macro
directive that will be expanded to the value of the \emp{TYPE\_NAME}
macro argument.

In \emp{std\_vector\_typemaps.i}, several typemaps are defined which
allow users to pass either \texttt{NumPy} arrays or Python lists where
a \emp{std::vector} is expected. One is an in-typemap macro for
passing a \emp{std::vector} of pointers of \dolfin objects to a C++
function and another is an out-typemap macro for passing a
\emp{std::vector} of primitives, using \numpy arrays, to Python. It is
not strictly necessary to add these typemaps as SWIG provides
interface files to handle templated types from
\emp{std::vector}. However, the provided \emp{std::vector}
functionality generates a lot of code and the resulting objects are
not very Pythonic. We have therefore chosen to implement our own
typemaps to handle \emp{std::vector} arguments.

The first typemap we describe enables the use of a Python list of
\dolfin objects instead of a \emp{std:vector} of pointers to such
objects. Since the handed \dolfin objects may and may not be stored
using a \emp{shared\_ptr}, we provide a typemap that works for both
situations. We also create typemaps for arguments where different
combinations of \emp{const} are used. Typically a signature in the
DOLFIN code can look like:
\begin{swigcode}
{const} std::vector<{const} dolfin::TYPE *>
\end{swigcode}

\pagebreak

\noindent where \emp{const} is optional.
To handle the optional \emp{const}s we use nested macros:
\begin{swigcode}
%define IN_TYPEMAPS_STD_VECTOR_OF_POINTERS(TYPE)
// Make SWIG aware of the shared_ptr version of TYPE
%types(SWIG_SHARED_PTR_QNAMESPACE::shared_ptr<TYPE>*);
IN_TYPEMAP_STD_VECTOR_OF_POINTERS(TYPE,const,)
IN_TYPEMAP_STD_VECTOR_OF_POINTERS(TYPE,,const)
IN_TYPEMAP_STD_VECTOR_OF_POINTERS(TYPE,const,const)
%enddef

%define IN_TYPEMAP_STD_VECTOR_OF_POINTERS(TYPE,CONST,CONST_VECTOR)
%typecheck(SWIG_TYPECHECK_POINTER) CONST_VECTOR std::vector<CONST dolfin::TYPE *> &
{
  $1 = PyList_Check($input) ? 1 : 0;
}

%typemap (in) CONST_VECTOR std::vector<CONST dolfin::TYPE *> & (std::vector<CONST dolfin::TYPE *> tmp_vec, SWIG_SHARED_PTR_QNAMESPACE::shared_ptr<dolfin::TYPE> tempshared)
{
  if (!PyList_Check($input)) {
    SWIG_exception(SWIG_TypeError, "list of TYPE expected");
  int size = PyList_Size($input);
  int res = 0;
  PyObject * py_item = 0;
  void * itemp = 0;
  int newmem = 0;
  tmp_vec.reserve(size);
  for (int i = 0; i < size; i++) {
    py_item = PyList_GetItem($input,i);
    res = SWIG_ConvertPtr(py_item, &itemp, $descriptor(dolfin::TYPE *), 0);
    if (SWIG_IsOK(res)) {
      tmp_vec.push_back(reinterpret_cast<dolfin::TYPE *>(itemp));
    }
    else {
      // If failed with normal pointer conversion then
      // try with shared_ptr conversion
      newmem = 0;
      res = SWIG_ConvertPtrAndOwn(py_item, &itemp, $descriptor(SWIG_SHARED_PTR_QNAMESPACE::shared_ptr< dolfin::TYPE > *), 0, &newmem);
      if (SWIG_IsOK(res)) {
        if (itemp) {
          tempshared = *(reinterpret_cast< SWIG_SHARED_PTR_QNAMESPACE::shared_ptr<dolfin::TYPE> * >(itemp));
          tmp_vec.push_back(tempshared.get());
        }
        // If we need to release memory
        if (newmem & SWIG_CAST_NEW_MEMORY) {
          delete reinterpret_cast< SWIG_SHARED_PTR_QNAMESPACE::shared_ptr< dolfin::TYPE > * >(itemp);
        }
      }
      else {
        SWIG_exception(SWIG_TypeError, "list of TYPE expected (Bad conversion)");
      }
    }
  }
  $1 = &tmp_vec;
}
%enddef
\end{swigcode}
In the typemap, we first check that we get a Python list. We then iterate
over the items and try to acquire the specified C++ object by converting
the Python object to the underlying C++ pointer. This is accomplished by:
\begin{swigcode}
res = SWIG_ConvertPtrAndOwn(py_item, &itemp, $descriptor(dolfin::TYPE *), 0, &newmem);
\end{swigcode}
%$ here to fool emacs...
If the conversion is successful we push the C++ pointer to
the \emp{tmp\_vec}. If the conversion fails we try to acquire a
\emp{shared\_ptr} version of the C++ object instead. If neither of the
two conversions succeed we raise an error.

The second typemap defined for \emp{std::vector} arguments is a so
called argout-typemap. This kind of typemap is used to return values
from arguments. In C++ non \emp{const} references or pointers
arguments are commonly used both as input and output of functions. In
Python are output arguments returned by the function. The following
call to the \emp{GenericMatrix::getrow} method illustrates the
difference between C++ and Python. The C++ signature is:
\begin{swigcode}
GenericMatrix::getrow(dolfin::uint row, std::vector<uint>& columns, std::vector<double>& values)
\end{swigcode}
Here, the sparsity pattern associated with row number \emp{row} is
filled into the \emp{columns} and \emp{values} vectors.  In Python, a
corresponding call should look like:
\begin{python}
columns, values = A.getrow(row)
\end{python}
To obtain the desired Python behavior we employ argout-typemaps. The
following macro defines such a typemap:
\begin{swigcode}
%define ARGOUT_TYPEMAP_STD_VECTOR_OF_PRIMITIVES(TYPE, TYPE_UPPER, ARG_NAME, NUMPY_TYPE)
// In typemap removing the argument from the expected in list
%typemap (in,numinputs=0) std::vector<TYPE>& ARG_NAME (std::vector<TYPE> vec_temp)
{
  $1 = &vec_temp;
}

%typemap(argout) std::vector<TYPE> & ARG_NAME
{
  npy_intp size = $1->size();
  PyArrayObject *ret = reinterpret_cast<PyArrayObject*>(PyArray_SimpleNew(1, &size, NUMPY_TYPE));
  TYPE* data = static_cast<TYPE*>(PyArray_DATA(ret));
  for (int i = 0; i < size; ++i)
    data[i] = (*$1)[i];

  // Append the output to $result
  %append_output(PyArray_Return(ret));
}
%enddef
\end{swigcode}
%$ emacs gets confused
The macro begins by defining an in-typemap that removes the output
argument and instantiates the \emp{std::vector} that will be passed as
argument to the C++ function. Then we have the code for the
argout-typemap, which is inserted after the C++ call. Here, the
``returned'' C++ arguments are transformed to Python arguments, by
instantiating a \numpy array \emp{ret} and filling it with the values
from the \emp{std::vector}. Note that here we are forced to copy the
values, or else the return argument would overwrite any previous
created return argument, with memory corruption as result.

\subsection{\dolfin{} header files and Python docstrings}
As mentioned earlier, the file \emp{kernel\_module.i}, generated
by \emp{generate.py}, tells SWIG what parts of DOLFIN that should be
wrapped.  The associated script \emp{generate\_docstrings.py} generates
the Python docstrings extracted from comments in the C++ documentation.
The comments are transformed into SWIG docstring directives like:
\begin{swigcode}
%feature("docstring")  dolfin::Data::ufc_cell "
Return current UFC cell (if available)
";
\end{swigcode}
and saved to a SWIG interface file \emp{docstrings.i}. The
\emp{docstrings.i} file is included from the main \emp{dolfin.i} file.
Note that the \emp{kernel\_module.i} and \emp{docstrings.i} files
are not generated automatically during the build process. This means
that when a header file is added to the \dolfin{} library, one must to
manually run \emp{generate.py} to update the \emp{kernel\_module.i}
and \emp{docstrings.i} files.


\subsection{Specializations of kernel modules}
The DOLFIN SWIG interface file \emp{kernel\_module.i} mirrors the directory
structure of \dolfin{}. As mentioned above, many directories come
with specializations in \emp{$\ldots$\_pre.i} and \emp{$\ldots$\_post.i}
files.  Below, we will highlight some of these specializations.

\paragraph{The \emp{mesh} module.}
The \emp{mesh} module defines the Python interfaces for \emp{Mesh},
\emp{MeshFunction}, \emp{MeshEntity}, and their subclasses. In
\emp{Mesh} the geometrical and topological information is stored in
contiguous C-arrays. These arrays are accessible from Python using
methods that return the underlying data wrapped to \numpy arrays. With
this functionality, for example move a mesh 1 unit in the
x-direction as follows:
\begin{python}
mesh.coordinates()[:,0] += 1
\end{python}
To obtain this functionality we ignore the original C++ methods and
then extend the \emp{Mesh} class with our own versions. The code for
this functionality is found in \emp{mesh\_pre.i}:
\begin{swigcode}
%ignore dolfin::Mesh::coordinates;
%ignore dolfin::Mesh::cells;
...
%extend dolfin::Mesh {
  PyObject* coordinates() {
    return %make_numpy_array(2, double)(self->num_vertices(),
					self->geometry().dim(),
					self->coordinates(), true);
  }

  PyObject* cells() {
    return %make_numpy_array(2, uint)(self->num_cells(), self->topology().dim()+1,
				      self->cells(), false);
  }
}
\end{swigcode}

\pagebreak

\noindent Here, we first ignore the C++ versions of the \emp{coordinates} and
\emp{cells} methods. Then we re-implements them using the
\emp{make\_numpy\_array} macro. This macro takes a pointer to a
C-array, together with the dimension and size of that array. The last
argument is used to set the writable flag of the NumPy array. This
flag is set to \emp{true} for \emp{coordinates} and false for
\emp{cells}.

In a similar fashion, we use the \emp{make\_numpy\_array} macro to wrap
the connectivity information to Python. This is done with the
following SWIG directives in the \emp{mesh\_pre.i} files.\vspace*{3pt}
\begin{swigcode}
%ignore dolfin::MeshConnectivity::operator();
%extend dolfin::MeshConnectivity {
  PyObject* __call__() {
    return %make_numpy_array(1, uint)(self->size(), (*self)(), false);
  }
  ...
}
\end{swigcode}

\vspace*{5pt}

\noindent Here, we extend the C++ extension layer of the
\emp{dolfin::}\emp{MeshConnectivity} class with a \emp{\_\_call\_\_}
method. The method returns all connections between any two types of
topological dimensions in the mesh.

In \emp{mesh\_pre.i} we also declare that it should be possible to
subclass \emp{SubDomain} in Python. This is done using the
\emp{\%director} directive.\vspace*{3pt}
\begin{swigcode}
%feature("director") dolfin::SubDomain;
\end{swigcode}

\vspace*{5pt}

\noindent To avoid code bloat this feature is only included for certain classes.
The following typemap enables seamless integration of \numpy array and
the \emp{Array<double>\&} in the \emp{inside} and \emp{map} methods.\vspace*{3pt}
\begin{swigcode}
%typemap(directorin) const dolfin::Array<double>& {
  $input = %make_numpy_array(1, double)($1_name.size(), $1_name.data().get(), false);
 }
\end{swigcode}%$ To fool emacs

\vspace*{5pt}

\noindent Even if it by concept and name is an \textit{in}-typemap, one can look at
it as an out-typemap (since it is a typemap for a callback function). SWIG
needs to wrap the arguments that the implemented \emp{inside} or \emp{map}
method in Python are called with. The above typemap is inserted in the
\emp{inside} and \emp{map} methods of the SWIG created C++ director class,
which is a subclass of \emp{dolfin::SubDomain}.

\dolfin comes with a \emp{Mesh}\-\emp{Enitity}\-\emp{Iterator} class. This
class lets a user iterate over a given \emp{MeshEntity}: a \emp{cell},
a \emp{vertex} and so forth. The iterators are mapped to Python by the
increment and de-reference operators in \emp{MeshEnitityIterator}. This
enabling is done by renaming these operators in \emp{mesh\_pre.i}:\vspace*{3pt}
\begin{swigcode}
%rename(_increment) dolfin::MeshEntityIterator::operator++;
%rename(_dereference) dolfin::MeshEntityIterator::operator*;
\end{swigcode}

\pagebreak

\noindent In \emp{mesh\_post.i}, the Python iterator
protocol (\emp{\_\_iter\_\_} and \emp{next}) is implemented for the\break
\emp{MeshEnitityIterator} class as
follows:\vspace*{3pt}
\begin{swigcode}
%extend dolfin::MeshEntityIterator {
%pythoncode
%{
def __iter__(self):
  self.first = True
  return self

def next(self):
  self.first = self.first if hasattr(self,"first") else True
  if not self.first:
    self._increment()
  if self.end():
    raise StopIteration
  self.first = False
  return self._dereference()
%}
}
\end{swigcode}

\vspace*{5pt}

\noindent We also rename the iterators to \emp{vertices} for the \emp{VertexIterator}, \emp{cells} for \emp{CellIterator}, and so forth. Iteration over a certain mesh entity in Python is then done by:\vspace*{3pt}
\begin{python}
for cell in cells(mesh):
    ...
\end{python}


\vspace*{3pt}
\paragraph{The \emp{la} module.}
The Python interface of the vector and matrix classes in the \emp{la}
module is heavily specialized, because we want the interface to be
intuitive and integrate nicely with \numpy.  First, all of the implemented
C++ operators are ignored, just like we did for the \emp{operator+=()} in
the \emp{Array} example above. This is done in the \emp{la\_pre.i} file:\vspace*{3pt}
\begin{swigcode}
%ignore dolfin::GenericVector::operator[];
%ignore dolfin::GenericVector::operator*=;
%ignore dolfin::GenericVector::operator/=;
%ignore dolfin::GenericVector::operator+=;
%ignore dolfin::GenericVector::operator-=;

%rename(_assign) dolfin::GenericVector::operator=;
\end{swigcode}

\vspace*{5pt}\enlargethispage{-12pt}

\noindent Here, we only ignore the virtual operators in the base class
\emp{GenericVector}, because SWIG only implements a Python version of a
virtual method in the base class.  Only the base class implementation
is needed since a method call in a derived Python class ends up in
the corresponding Python base class. The base class in Python hands
the call over to the corresponding base class call in C++, which ends
up in the corresponding derived C++ class.  Hence, when we ignore the
above mentioned operators in the base class, we also ignore the same
operators in the derived classes.  Finally, we rename the assignment
operator to \emp{\_assign}. The \emp{\_assign} operator will be used by
the \emp{slice} operator implemented in \emp{la\_post.i}, see below.


The following code snippet from \emp{la\_post.i} shows how the special
method \emp{\_\_mul\_\_} in the Python interface of \emp{GenericVector}
is implemented:
\begin{swigcode}
%extend dolfin::GenericVector {
  void _scale(double a)
  {(*self)*=a;}

  void _vec_mul(const GenericVector& other)
  {(*self)*=other;}

  %pythoncode %{
   ...
    def __mul__(self,other):
        """x.__mul__(y) <==> x*y"""
        if isinstance(other,(int,float)):
            ret = self.copy()
            ret._scale(other)
            return ret
        if isinstance(other, GenericVector):
            ret = self.copy()
            ret._vec_mul(other)
            return ret
        return NotImplemented
    ...
%} }
\end{swigcode}
We first expose \emp{operator*=} to Python by implementing
corresponding hidden methods, the \emp{\_scale} method for scalars and
the \emp{\_vec\_mul} method for other vectors. These methods are then
used in the \emp{\_\_mul\_\_} special method in the Python interface.

Vectors in the DOLFIN Python interface support access and assignment using slices and
\numpy arrays of booleans or integers, and lists of integers. This
is achieved using the \emp{get} and \emp{set} methods in the
\emp{GenericVector}, but is quite technical. In short, a helper class
\emp{Indices} is implemented in \emp{Indices.i}. This class is used in
the \emp{\_get\_vector} and \emp{\_set\_vector} helper functions defined
in the \emp{la\_get\_set\_items.i} file.
\begin{python}
%extend dolfin::GenericVector {
  %pythoncode %{
   ...
    def __getslice__(self, i, j):
        if i == 0 and (j >= len(self) or j == -1):
            return self.copy()
        return self.__getitem__(slice(i, j, 1))

    def __getitem__(self, indices):
        from numpy import ndarray, integer
        from types import SliceType
        if isinstance(indices, (int, integer)):
            return _get_vector_single_item(self, indices)
        elif isinstance(indices, (SliceType, ndarray, list) ):
            return down_cast(_get_vector_sub_vector(self, indices))
        else:
            raise TypeError, "expected an int, slice, list or numpy array of integers"
  ...
%} }
\end{python}
The above code demonstrates the implementation of the slice and index
access in the Python layer of \emp{GenericVector}. When accessing a
vector using a full slice, \emp{v[:]}, \emp{\_\_getslice\_\_} is called
with \emp{i} = 0 and \emp{j} = a-large-number (default in Python).
In this case, we return a copy of the vector. Otherwise, we create a
slice and pass it to \emp{\_\_getitem\_\_}. In the latter method, we
check whether the \emp{indices} argument is a single item or not and
calls upon the correct helper functions.

\section{JIT compilation of \ufl forms, \emp{Expression}s and \emp{SubDomain}s}
The DOLFIN Python interface makes extensive use of just in time (JIT) compilation;
that is code that is compiled, linked and imported into Python using
Instant, see Chapter~\ref{chap:wilbers}. This process is facilitated
by employing the form compilers \ffc or \sfc that translates \ufl code
into \ufc code.  In a similar fashion, DOLFIN enables JIT compilation
of \emp{Expression}s and \emp{SubDomain}s.

We provide two ways of defining an \emp{Expression} in DOLFIN via Python:
subclassing \emp{Expression} directly in Python, and through the compile
function interface. In the first alternative, the \emp{eval} method is
defined in a subclass of Expression:
\begin{python}
class MyExpression(Expression):
    def eval(self, values, x):
        values[0] = 10*exp(-((x[0] - 0.5)**2 + (x[1] - 0.5)** 2) / 0.02)
f = MyExpression()
\end{python}
Here, \emp{f} will be a subclass of both \emp{ufl.Function} and
\emp{cpp.Expression}.  The second alternative is to instantiate the
\emp{Expression} class directly:
\begin{python}
f = Expression("10*exp(-(pow(x[0] - 0.5, 2) + pow(x[1] - 0.5, 2)) / 0.02)")
\end{python}
This example creates a scalar \emp{Expression}. Vector valued and matrix
valued expressions can also be created. As in the first example, \emp{f}
will be a subclass of \emp{ufl.Function}. But it will not inherit from
\emp{cpp.Expression}.  Instead, we create a subclass in C++ that inherit
from \emp{dolfin::Expression} and implements the \emp{eval} method.
The generated code looks like:
\begin{c++}
class Expression_700475d2d88a4982f3042522e5960bc2: public Expression{
public:
  Expression_700475d2d88a4982f3042522e5960bc2():Expression(2){}

  void eval(double* values, const double* x) const{
    values[0] = 10*exp(-(pow(x[0] - 0.5, 2) + pow(x[1] - 0.5, 2)) / 0.02);
  }
};
\end{c++}
The name of the subclass is generated from a hash of the passed
expression string. The code is inserted into \emp{namespace dolfin} and
the appropriate \emp{\#include} is also inserted in the code. Instant is
used to compile and link a Python module from the generated code. The
class made by Instant is imported into Python and used to dynamically
construct a class that inherits the generated class together with
\emp{ufl.Function} and \emp{Expression}.  Dynamic creation of classes
in Python is done using so called meta-classes.  In a similar fashion,
DOLFIN provides functionality to construct C++ code and JIT compile
subclasses of \emp{SubDomain} from Python.


\section{Debugging mixed language applications}
\index{debugging}

Debugging mixed language applications may be more challenging than
debugging single language applications.  The main reason is that
debuggers are written mainly for either compiled languages or scripting
languages. However, as we will show, mixed language applications can be
debugged in much of the same way as compiled languages.

Before starting the debugger, you should make sure that your library,
or the relevant parts of it, is compiled with the debugging on. With
GCC this is done with the \emp{-g} option. The additional debugging
information in the code slows down the performance. Therefore, \dolfin\
is by default not compiled with \emp{-g}.  After compiling the code
with debugging information, you may start Python in \citet{www:gdb},
the GNU Project Debugger, as follows:
\begin{bash}
gdb python
(gdb) run
...
\end{bash}
The problem with GDB is that only one thread is running. This means that
you will not be able to set break points and so on once you have started
the Python interpreter.

However, \citet{www:ddd}, the Data Display Debugger, facilitates running
the debugger and the Python interpreter in two different threads. That
is, you will have two interactive threads, one debugger and the Python
interpreter, during your debugging session. The DDD debugger is started
as:
\begin{bash}
ddd python
\end{bash}
The crucial next step is to start the Python session in a separate
execution window by clicking on \emp{View->Execution Window}.  Then you
start the Python session:
\begin{bash}
(gdb) run
\end{bash}
After importing for your module you may click or search (using the
\emp{Lookup} field) through the source code to set breakpoints, print
variables and so on.

Another useful application for analyzing memory management is
\citet{www:valgrind}. To find memory leaks, do as follows:
\begin{bash}
valgrind --leak-check=full python test_script.py
\end{bash}
Valgrind also provides various profilers for performance testing.

\paragraph{Acknowledgments.}

The authors are very thankful to Johan Jansson who initiated the work
on the DOLFIN Python interface and to Ola Skavhaug and Martin
S. Aln\ae s who have contributed significantly to the development.
Finally, Marie Rognes has improved the language in this chapter
significantly.

\endgroup
