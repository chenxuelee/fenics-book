\bgroup
\setcounter{page}{1}

\nolinenumbers


\def\plseries#1{\noindent \fontfamily{ptm}\fontsize{16}{18}{\selectfont #1\par}}

%\def\plseriesvolume#1{\noindent \fontfamily{ptm}\fontsize{12}{14}{\selectfont #1\par}}

\def\plserieseditor#1{\noindent \fontfamily{ptm}\fontsize{10}{12}{\selectfont #1\par}}

\def\plseriesid#1{\noindent \fontfamily{ptm}\fontsize{9}{11}{\selectfont #1\par}}

\def\pleditor#1{\noindent \fontfamily{ptm}\fontsize{14}{16}{\selectfont #1\par}}

\def\pltitle#1{\noindent \fontfamily{ptm}\fontsize{28}{30}{\selectfont #1\par}}

\def\plsubtitle#1{\noindent \fontfamily{ptm}\fontsize{18}{20}{\selectfont #1\par}}


\noindent{\fontfamily{ptm}\fontsize{16}{18}\selectfont\begin{minipage}{0.9\textwidth}{Lecture Notes\newline in Computational Science\newline and Engineering}\end{minipage}}

\begin{minipage}{0.96\textwidth}{\vspace*{-40pt}\hfill{\fontfamily{ptm}\fontsize{30}{32}\selectfont{84}}}\end{minipage}

\vspace*{-6pt}

\hbox to \textwidth{\hrulefill}

\thispagestyle{empty}

\vspace*{18pt}

{\noindent\fontfamily{ptm}\fontsize{12}{14}\selectfont Editors:\par\vspace*{9pt}
\noindent Timothy J. Barth\par
\noindent Michael Griebel\par
\noindent David E. Keyes\par
\noindent Risto M. Nieminen\par
\noindent Dirk Roose\par
\noindent Tamar Schlick\par}

\vfill

{\noindent \fontfamily{ptm}\fontsize{9.5}{11.5}\selectfont For further volumes:\\
http://www.springer.com/series/3527}

\clearpage

%%%%%%%%%%%%%%%%%%%%%%%%%%%%%%%%%%%%%%%%%%%%%%%%%%%%%%%%%%%%%%%%%
%\onpage{2}
%%%%%%%%%%%%%%%%%%%%%%%%%%%%%%%%%%%%%%%%%%%%%%%%%%%%%%%%%%%%%%%%%

\ \thispagestyle{empty}

\pagebreak

%%%%%%%%%%%%%%%%%%%%%%%%%%%%%%%%%%%%%%%%%%%%%%%%%%%%%%%%%%%%%%%%%%
%\onpage{3}
%%%%%%%%%%%%%%%%%%%%%%%%%%%%%%%%%%%%%%%%%%%%%%%%%%%%%%%%%%%%%%%%%%

\plseries{Anders Logg \enspace{\fontsize{10}{12}\selectfont{\raisebox{1.5pt}{$\bullet$}}}
\enspace Kent-Andre Mardal \enspace{\fontsize{10}{12}\selectfont{\raisebox{1.5pt}{$\bullet$}}}
\enspace Garth Wells}\vspace*{6pt}

\noindent{\fontfamily{ptm}\fontsize{14}{16}\selectfont{Editors}\vspace*{45pt}}

\pltitle{Automated Solution of Differential\\ Equations by the Finite Element\\ Method}\vspace{28pt}

\plsubtitle{The FEniCS Book}\vspace*{36pt}

\thispagestyle{empty}

\vfill\noindent\includegraphics{springer-new.eps}

\clearpage

%%%%%%%%%%%%%%%%%%%%%%%%%%%%%%%%%%%%%%%%%%%%%%%%%%%%%%%%%%%%%%%%%%
%\onpage{4}
%%%%%%%%%%%%%%%%%%%%%%%%%%%%%%%%%%%%%%%%%%%%%%%%%%%%%%%%%%%%%%%%%%
%\address{}

\vspace*{-14pt}
{\noindent \fontfamily{ptm}\fontsize{9}{11}\selectfont {\it Editors}}\vspace*{-1.2pc}
\begin{multicols}{2}
{\noindent \fontfamily{ptm}\fontsize{9}{11}\selectfont{Anders Logg\\
Kent-Andre Mardal\\
Simula Research Laboratory\\
PO Box 134\\
1325 Lysaker\\
Norway\\
logg@simula.no\\
kent-and@simula.no}\par}

\thispagestyle{empty}

\columnbreak

{\noindent \fontfamily{ptm}\fontsize{9}{11}\selectfont{Dr. Garth Wells\\
University of Cambridge\\
Department of Engineering\\
Trumpington Street\\
CB1 1PZ Cambridge\\
United Kingdom\\
gnw20@cam.ac.uk}\par}
\end{multicols}

\vfill

{\noindent \fontfamily{ptm}\fontsize{9}{10}\selectfont

\noindent ISSN 1439-7358

\noindent ISBN 978-3-642-23098-1\hspace*{3pc}e-ISBN 978-3-642-23099-8

\noindent DOI 10.1007/978-3-642-23099-8

\noindent Springer Heidelberg Dordrecht London New York \vspace*{10pt}}

{\noindent \fontfamily{ptm}\fontsize{8}{9}\selectfont

\noindent Library of Congress Control Number: ``PCN applied for''\vspace*{9pt}

%\noindent Mathematics Subject Classification (2010): 65Cxx, 65P99, 65Zxx, 35B27, 65R20, 70Hxx, 70Kxx,\\
%\hphantom{\noindent Mathematics Subject Classification (2010): }70-08, 74-XX, 76S05, 76Txx\vspace*{9pt}

\noindent \copyright\ Springer-Verlag Berlin Heidelberg 2012

\noindent This work is subject to copyright. All rights are reserved,
whether the whole or part of the material is concerned,
specifically the rights of translation, reprinting, reuse of
illustrations, recitation, broadcasting, reproduction on microfilm
or in any other way, and storage in data banks. Duplication of
this publication or parts thereof is permitted only under the
provisions of the German Copyright Law of September 9, 1965, in
its current version, and permission for use must always be
obtained from Springer. Violations are liable to prosecution under
the German Copyright Law. %\vspace*{9pt}

\noindent The use of general descriptive names, registered names,
trademarks, etc. in this publication does not imply, even in the
absence of a specific statement, that such names are exempt from
the relevant protective laws and regulations and therefore free
for general use.\vspace*{9pt}

%\noindent \textit{Cover illustration}: The nudg++ team - Tim Warburton (Rice University), Nigel Nunn, Nico G\"{o}del (Helmut-Schmidt-University, University of the Federal Armed Forces Hamburg)\vspace*{9pt}

%\noindent \textit{Cover design}: deblik, Berlin\vspace*{9pt}

\noindent Printed on acid-free paper\vspace*{9pt}

\noindent Springer is part of Springer Science+Business Media (www.springer.com)\par}

\egroup


% Preface
\chapter*{Preface}

\thispagestyle{empty}

The FEniCS Project set out in 2003 with an idea to automate the
solution of mathematical models based on differential equations.
Initially, the FEniCS Project consisted of two libraries: DOLFIN and
FIAT. Since then, the project has grown and now consists of the core
components DOLFIN, FFC, FIAT, Instant, UFC and UFL. Other FEniCS
components and applications described in this book are SyFi/SFC,
FErari, ASCoT, Unicorn, CBC.Block, CBC.RANS, CBC.Solve and DOLFWAVE.

This book is written by researchers and developers behind the FEniCS
Project. The presentation spans mathematical background, software
design and the use of FEniCS in applications. The mathematical
framework is outlined in Part~I, the implementation of central
components is described in Part~II, while Part~III concerns a wide
range of applications. New users of FEniCS may find the tutorial
included as the opening chapter particularly useful.

Feedback on this book is welcome, and can be given at
\url{https://launchpad.net/fenics-book}. Use the Launchpad system to
file bug reports if you find errors in the text. For more information
about the FEniCS Project, access to the software presented in this
book, documentation, articles and presentations, visit the FEniCS
Project web site at \url{http://fenicsproject.org}. Some of the
chapters in this book are accompanied by supplementary material in the
form of code examples. These code examples can be downloaded from
\url{http://fenicsproject.org/book/}.

The software developed by the FEniCS Project is free for all to use
and modify (licensed under the GNU (L)GPL), and so is this
book. Permission is granted to copy, distribute and/or modify this
book under the terms of the GNU Free Documentation License, Version
1.3 or any later version published by the Free Software Foundation. A
copy of the license is included in the chapter entitled "GNU Free
Documentation License".

\vspace{1em}

\noindent
Anders Logg, Kent-Andre Mardal and Garth N. Wells \\*
\emph{Oslo and Cambridge, November 2011}


\tableofcontents


% Cross references
\index{Dirichlet boundary condition|see{boundary condition}}
\index{Neumann boundary condition|see{boundary condition}}
\index{Robin boundary condition|see{boundary condition}}
\index{Lagrange finite element|see{finite element}}
\index{CG element|see{finite element}}
\index{discontinuous Lagrange element|see{finite element}}
\index{DG element|see{finite element}}
\index{Crouzeix--Raviart element|see{finite element}}
\index{Raviart--Thomas element|see{finite element}}
\index{Brezzi--Douglas--Marini element|see{finite element}}
\index{Mardal--Tai--Winther element|see{finite element}}
\index{Arnold--Winther element|see{finite element}}
\index{N\'ed\'elec element|see{finite element}}
\index{Argyris element|see{finite element}}
\index{Hermite element|see{finite element}}
\index{Morley element|see{finite element}}
\index{Bubble element|see{finite element}}
\index{finite element assembly|see{assembly}}
\index{goal oriented error estimate|see{error estimate}}
\index{a priori error estimate|see{error estimate}}
\index{a posteriori error estimate|see{error estimate}}
\index{JIT|see{just-in-time compilation}}
\index{AD|see{automatic differentiation}}
\index{forward mode AD|see{automatic differentiation}}
\index{reverse mode AD|see{automatic differentiation}}
\index{XFEM|see{extended finite element method}}
\index{SUPG|see{stabilization}}
\index{PDE|see{partial differential equation}}
\index{nonlinear PDE|see{partial differential equation}}
\index{time-dependent PDE|see{partial differential equation}}
\index{XML format|see{file formats}}
\index{IPCS|see{incremental pressure correction}}
\index{Ciarlet finite element definition|see{finite element}}
\index{CSS|see{consistent splitting scheme}}
\index{LBB conditions|see{Ladyzhenskaya--\babuska{}--Brezzi conditions}}
\index{multilinear form|see{form}}
\index{Navier--Stokes|see{incompressible Navier--Stokes equations}}
