\chapter*{Preface}

\thispagestyle{empty}

The FEniCS Project set out in 2003 with the idea to automate the
solution of mathematical models based on differential equations.
Initially, the FEniCS Project consisted of two libraries: DOLFIN and
FIAT. Since then, the project has grown and now consists of the core
components DOLFIN, FFC, FIAT, Instant, UFC and UFL. Other FEniCS
components and applications described in this book are SyFi/SFC,
FErari, AScot, Unicorn, CBC.Block, CBC.RANS, CBC.Solve and DOLFWAVE.

\editornote{Check that we are not missing any
  applications/components.}

This book is written by researchers and developers behind the FEniCS
Project. The presentation spans mathematical background, software
design and the use of FEniCS in applications. The mathematical
framework is outlined in Part~I, the implementation of central
components is described in Part~II, while Part~III concerns a wide
range of applications. New users of FEniCS may find the tutorial
included as the opening chapter particularly useful.

Feedback on this book is welcome, and can be given at
\url{https://launchpad.net/fenics-book}. Use the Launchpad system to
file bug reports if you find errors in the text. For more information
about the FEniCS Project, access to the software presented in this
book, documentation, articles and presentations, visit the FEniCS
Project web site at \url{http://www.fenicsproject.org}. Some of the
chapters in this book are accompanied by supplementary material in the
form of code examples. These code examples can be downloaded from
\url{http://www.fenicsproject.org/book}.

\editornote{GNW/KAM: Comment on code license. AL: Which code license?}

\vspace{1em}

\noindent
Anders Logg, Kent-Andre Mardal and Garth N. Wells \\*
\emph{Oslo and Cambridge, June 2011}
