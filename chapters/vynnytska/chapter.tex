\fenicschapter{Dynamic simulations of convection in the Earth's mantle}
              {Thermochemical convection}
              {Lyudmyla Vynnytska,  Stuart R.~Clark and Marie E.~Rognes}
              {vynnytska}

\newcommand{\erf}{\mathrm{erf}}
\newcommand{\expc}{\mathrm{expc}}
\newcommand{\composition}{\phi}
\newcommand{\test}{\psi}
\newcommand{\triang}{\mathcal{T}}
\newcommand{\jump}[1]{[[#1]]}
\newcommand{\avg}[1]{\{#1\}}
\newcommand{\code}[1]{$\texttt{#1}$}

In this chapter, we model dynamic convection processes in the Earth's
mantle: linking the geodynamical equations, numerical implementation
and Python code tightly together. We start by giving a physical
motivation and description of this challenging problem and continue by
deriving its mathematical formulation. The time-dependent nonlinear
partial differential equations to be solved are the quasi-static
Navier--Stokes equations with depth- and temperature-dependent
viscosities in combination with advection-diffusion equations for the
compositional density variations and the temperature. Next, we present
a numerical algorithm for the simulation of this scenario and an
implementation of this algorithm using the DOLFIN Python
interface. Finally, we discuss the results yielded by a simulation of
this scenario and make some concluding remarks.

%%------------------------------------------------------------------------------

\section{Introduction}

In contrast to the hydrosphere and atmosphere, the Earth's crust and
mantle are primarily solid in nature, allowing the rapid progression
of earthquakes through the elastic media.  However, one of the most
important discoveries of geodynamics is that materials can behave
elastically on certain timescales and viscously on other
timescales. Glaciers work on this principle: solid ice slowly deforms
and flows under the effect of gravity.  While glaciers move on the
order of meters per day, the mantle moves at the speed of a few
centimeters a year~\citep{vanderMeer2010}.  At this rate, a piece of
the Earth's lithosphere, or slab, would take at least a hundred
million years to sink from the Earth's surface to the outer
core~\citep{Jarvis2007}.

Blanketing the outer core, seismologists detect a layer through which
seismic waves are anomalous slow: the D'' (\emph{dee} double prime)
layer.  In some regions, this layer is very thin, overlain by fast
zones that may indicate slabs buried deep within the
mantle~\citep{McNamaraZhong2005}.  Underneath southern Africa and the
Pacific, two prominent seismically anomalous slow regions exist,
seemingly pointing to a hotter or compositionally denser
material~\citep{McNutt1998}. Such heterogeneities have led
geoscientists to speculate on the existence of large chemically
isolated reservoirs in the mantle, perhaps a remnant from the early
Earth's mantle~\citep{Burke2008}.  But how can these chemically
isolated reservoirs survive in a vigorously convecting mantle?
Geodynamicists have tried to answer this question with computer
simulations of thermomechanical convection of a compositionally
heterogeneous mantle, also known as thermochemical
models~\citep{McNamara2010}. The challenge for geodynamicists is: can
the assumptions made in matching the longevity of these reservoirs be
consistent with seismic observations and the physics we know of the
Earth's interior and plate tectonics?

At a primary level, the observations we need to match are simply the
transfer of matter between the Earth's interior and surface. At the
surface of the Earth, tectonic plates are bent and pushed into the
interior. In other places, we see large volcanic terranes created by
material sourced from the Earth's mantle. To model the motion of the
mantle over long timescales, the Navier-Stokes equations are well
established in their ability to replicate this behavior, given the
right assumptions, coupled with a conservation equation for the
thermal energy in the mantle.

%%------------------------------------------------------------------------------

\section{Mathematical statement of the problem}
\label{vynnytska:sec:maths}

We model the problem in a rectangular box $\Omega$, neglecting the
sphericity of the Earth, representing the mantle from the surface to
the core-mantle boundary. The base of the box is covered by a
relatively thin layer of denser material, while the initial
temperature field is set to represent the colder lithosphere along the
top and slab descending on the right hand-side; a hotter layer is
imposed along the bottom and left to represent the hotter D'' layer
and a plume ascending respectively following the boundary layer
arguments of~\citet{KekenEtAl1997}. The idea is simply to create an
initial configuration that drives the convection of the problem.
\begin{figure}[htbp]
  \begin{center}
   \includegraphics[width=0.95\columnwidth]{chapters/vynnytska/figures/layout.png}
    \caption{Initial configuration of the model in the $h$ by
      $\lambda$ rectangular box $\Omega$. Color shows dimensionless
      temperature (for dimensional values, see
      Section \ref{vynnytska:sec:results}. The shaded layer on the
      bottom with height $d$ is the denser layer corresponding to the
      D'' layer.}
  \end{center}
\end{figure}

The viscosity of the mantle, on the order of $10^{22}Pa~s$, is so high
that the inertial forces and compressibility are
negligible~\citep{Ricard2009}. If the assumption of compressibility is
relaxed, the lower mantle can support piles of geochemically isolated
material with sharp edges~\citep{TanGurnis2005}. However, for our
purposes, we assume an incompressible thermal flow driven by
temperature and compositional density variations modeled by the
following nondimensional system of equations:
\begin{align}
  \label{vynnytska:eq:momentum}
  - \Div \sigma^{\prime} - \Grad p
  &=  \left( Rb \, \composition - Ra \, T \right) e,\\
  \label{vynnytska:eq:incompress}
  \Div u & =  0,\\
  \label{vynnytska:eq:energy}
  \frac{\partial T}{\partial t} + u \cdot \Grad T & =  \Delta T.
\end{align}
Here, $\sigma^{\prime}$ is the deviatoric stress tensor, $p$ is the
pressure, $Ra$ and $Rb$ are the thermal and compositional Rayleigh
numbers respectively, $T$ is the temperature, $\composition$ is a
composition field, $u$ is the velocity, and $e$ is a unit vector in
the direction of gravity. The Rayleigh numbers
$Ra$ and $Rb$ are defined in terms of the thermal expansivity
$\alpha$, the acceleration of gravity $g$, the reference density and
viscosity $\rho_0$ and $\eta_0$, the temperature and density contrast
between the two layers $\Delta T$ and $\Delta \rho$, the coefficient
of thermal diffusion $k$, and the depth $h$ as
\begin{equation}
  Ra = \frac{\alpha g \rho_0 \Delta T h^3}{k \eta_0},  \quad
  Rb = \frac{\Delta \rho g h^3}{k \eta_0}.
\end{equation}

The values of the parameters defining the Rayleigh numbers and
dimensionalising the problem are given in
Section \ref{vynnytska:sec:results}. The Rayleigh numbers, $Ra$ and
$Rb$, are defined as equal and as $10^{-6}$ within the ranges for the
Earth \citep{MontagueKelloggManga1998} and such that fluid convection
dominates.

The fluid flow induces transport of the composition
$\composition$. This transport is governed by the equation
\begin{equation}
  \label{vynnytska:eq:trans}
  \frac{\partial \composition}{\partial t} +  u \cdot \Grad \composition = 0.
\end{equation}
However, some chemical diffusivity $k_{ch}$ is also present in the physical
system~\citep{KekenEtAl1997, HansenYuen1988}. Therefore, we substitute
the pure advection equation~\eqref{vynnytska:eq:trans} by an
advection-diffusion equation of the form
\begin{equation}
  \label{vynnytska:eq:transdif}
  \frac{\partial \composition}{\partial t}
  + u  \cdot \Grad \composition =  k_c \Delta \composition .
\end{equation}
This allows us to take chemical diffusivity into account. Here, $k_c =
k_{ch}/k_{th}$.

It remains to specify the constitutive relationship relating the
deviatoric stress tensor $\sigma^{\prime}$ to the other
variables. Here, we consider the case of a Newtonian rheology with a
depth- and temperature-dependent viscosity
$\eta$~\citep{BlankenbachBusse1989}. The stress-strain relationship is
then described by the equations
\begin{align}
  \sigma^{\prime} &= 2 \eta \dot{\varepsilon} (u), \\
  \dot{\varepsilon}(u) &= \frac{1}{2} \left (\Grad u + \Grad u^T \right),  \\
  \eta &= \eta(T, x_2)
  = \eta_0 \exp \left( -b T/\Delta T + c (1 - x_2)/ h \right) .
\end{align}
Here, $\dot {\varepsilon}$ is the strain rate tensor, $\eta_0$ is a
base viscosity, $x_2$ is the second spatial coordinate, and $b$ and
$c$ are given additional parameters.

For the current scenario, we will consider the following boundary
conditions. For the velocity $u$ and the deviatoric stress tensor
$\sigma^{\prime}$, we apply no slip conditions on the bottom boundary
($x_2 = 0$), and free slip and reflective symmetry on the remaining
boundary $\Gamma = \partial \Omega \backslash \{x: x_2 = 0 \}$:
\begin{align}
  \label{vynnytska:eq:bcs}
  u |_{x_2 = 0} = 0, \quad
  u_n|_{\Gamma}  =  \sigma^{\prime}_{n \tau} |_{\Gamma} = 0,
\end{align}
where $n$ is the outward normal and $\tau$ is the tangent vector on
the boundary $\partial \Omega$. For the temperature, we fix its value
on the top and bottom boundary and apply symmetry conditions (or no
heat exchange) on the left and right boundaries:
\begin{align}
  T |_{x_2 = 0} = \Delta T, \quad T |_{x_2 = h}  = 0, \quad
  \partial_{x_1} T |_{x_1 = 0}  = \partial_{x_1} T |_{x_1 = \lambda} = 0.
\end{align}
For the composition $\composition$, we set
\begin{align}
  \composition |_{x_2 = 0} = 1, \quad \partial_{n} \composition|_{\Gamma} = 0,
\end{align}
i.e. we fix composition on the bottom of the box $\Omega$ and no flux
conditions on the remainder of the boundary. This condition can be
viewed as a consequence of the no outflow conditions for the velocity.

The initial temperature field is given by $T_0$.
Below~\eqref{vynnytska:eq:ic1} is an analytical expression based on
the boundary layer theory~\citep{KekenEtAl1997} that take into account
value for $Ra$ with the input from upper $T_u$, lower $T_l$, right
$T_r$ and left $T_s$ parts of the domain:
\begin{subequations}
  \label{vynnytska:eq:ic1}
  \begin{align}
    T_0 &= T_u + T_l + T_r + T_s - 1.5, \\
    q &= \frac{\lambda^{7/3}}{\left(1 + \lambda^4 \right)^{2/3}} \left( \frac{Ra}{2 \sqrt{\pi}}\right)^{2/3}, \quad Q  = 2\sqrt{\frac{\lambda}{\pi q}},   \\
    T_u &= 0.5 \erf \left( \frac{1-x_2}{2} \sqrt{\frac{q}{x_1}  } \right), \\
    T_l &= 1 - 0.5 \erf \left( \frac{x_2}{2} \sqrt{\frac{q}{\lambda - x_1}  } \right), \\
    T_r &= 0.5 + \frac{Q}{2\sqrt{\pi}} \sqrt{\frac{q}{x_{2} + 1} } \expc \left( - \frac{x_1^2 q}{4 x_2 + 4} \right), \\
    T_s &= 0.5 - \frac{Q}{2\sqrt{\pi}} \sqrt{\frac{q}{2 - x_{2}} } \expc \left( - \frac{ \left(\lambda - x_1 \right)^2  q}{8 - 4 x_2} \right).
  \end{align}
\end{subequations}
In order to keep the initial temperature distribution in the range
between $0$ and $1$ we performed additional correction. According to
eqs.~\eqref{vynnytska:eq:ic1} there are two peaks: positive in the top
right and negative in the bottom left corners.  We mapped all values
below $0$ to $0$ and above $1$ to $1$.

The initial composition $\composition$ is a step function equal to one
on the bottom layer and to zero on the top layer.

%%------------------------------------------------------------------------------

\section{Numerical method}
In this section, we present a numerical solution method for the
thermochemical convection model established in the previous. Instead
of solving the fully coupled system of nonlinear time-dependent
partial differential equations, we consider a predictor-corrector
based splitting scheme~\citep{BergKekenYuen1993,
  HansenEbel1988}. Therefore, we first present the numerical methods
for the solution of each separate equation before formulating the full
algorithm. Special attention must be paid to the discretization of the
energy~\eqref{vynnytska:eq:energy} and transport
equations~\eqref{vynnytska:eq:trans} due to the interface between the
compositionally distinct layers. For the Stokes
equations~\eqref{vynnytska:eq:momentum}~--~\eqref{vynnytska:eq:incompress},
we use a mixed finite element formulation, thus obtaining solutions
for the velocity and pressure simultaneously.

\subsection{Mixed finite element formulation of the Stokes equations}

Let $\triang_h$ be a mesh partitioning of the domain $\Omega$. Let
$V_h$ and $Q_h$ be finite dimensional spaces, defined relative to the
partition $\triang_h$, for the velocity and pressure fields
respectively. The standard discrete mixed finite element formulation
with independent approximation of the (continuous) velocity field $u$
and the pressure field $p$ for the incompressible Stokes
equations~\eqref{vynnytska:eq:momentum}~--~\eqref{vynnytska:eq:incompress}
with the boundary conditions given by~\eqref{vynnytska:eq:bcs} reads
as follows~\citep{ZienkiewiczTaylor2000}. For a given temperature
$T_h$ and composition $\composition_h$, find $u_h \in V_h$ and $p_h
\in Q_h$ such that
\begin{equation}
  \label{vynnytska:eq:mixed}
  a_{IS} ( (u_h, p_h), (v_h, q_h) ) = L_{IS} ( (v_h, q_h) )
\end{equation}
for all $v_h \in V_h$ and all $q_h \in Q_h$ where
\begin{align}
  a_{IS} ( (u_h, p_h), (v_h, q_h) )
  & =
  \int_{\Omega} 2 \eta \dot {\varepsilon}(u_h)  \cdot \dot {\varepsilon}(v_h)
  + p_h \Div v_h + q_h \Div u_h \dx, \\
  L_{IS} ( (v_h, q_h) ) &=
  \int_{\Omega} ( Ra T_h - Rb \composition_{h} ) e \cdot v_h  \dx.
\end{align}
In the subsequent simulations, we use the lowest order Taylor--Hood
elements for the velocity and the pressure; that is, the combination
of continuous piecewise quadratic vector fields for $V_h$ and
continuous piecewise linears for $Q_h$~\citep{TaylorHood1973}.

\subsection{Discontinuous Galerkin formulation of advection--diffusion equations}

The energy and transport equations~\eqref{vynnytska:eq:energy}
and~\eqref{vynnytska:eq:transdif} have the same structure from the
mathematical point of view. The equations both model the time
evolution of advection-diffusion processes. The numerical analysis of
these can therefore be performed by the same numerical
scheme. However, the numerical treatment of advection dominated
advection--diffusion equations is nontrivial. There exists a large
body of research on the development of efficient computational schemes
for such kinds of
problems~\citep{Lin2006},~\citep{ZienkiewiczTaylor2000}.  Within the
finite element setting, there are two main approaches: Petrov-Galerkin
approximation and discontinuous Galerkin methods. Here, we prefer a
discontinuous Galerkin method due to a straightforward and apparent
reason, namely its potential for dealing effectively with
discontinuous property fields. In the following, we describe an
upwinded discontinuous Galerkin formulation for the
equation~\eqref{vynnytska:eq:transdif} for the compositional field
$\composition$. This formulation also applies for the energy
equation~\eqref{vynnytska:eq:energy} with $k_c = 1$. For the sake of
clarity, we consider the discretization of~\eqref{vynnytska:eq:trans}
separately first.

Using full upwind numerical flux and taking into consideration that
the normal component of velocity is equal to zero on the boundary, the
spatial discontinuous finite element discretization
of~\eqref{vynnytska:eq:trans} with a given $u_h$ reads
as~\citep{PietroLoForteParolini2006}: find $\composition_h \in P_h$
such that
\begin{equation}
  \sum_{T \in \Omega_h} \int_T \frac{\partial \composition_h}{\partial t} \test_h
  \dx + a_{A} (u_h; \composition_h, \test_h) = 0,
\end{equation}
for all $\test_h \in P_h$. Here
\begin{equation}
  \label{vynnytska:eq:dgadv}
   a_{A} (u_h; \composition_h, \test_h )
   =
   - \sum_{T \in \triang_h} \int_T \composition_h u_h \cdot \Grad \test_h \dx
   + \sum_{e \in \Gamma_i} \int_{e} \left (
   u_h \cdot \jump{\test_h} \avg{\composition_h} + \frac{1}{2}
   | u_h \cdot n | \jump{\test_h} \jump{\composition_h} \right ) \ds,
\end{equation}
wherein $\Gamma_i$ denotes the interior edges of $\triang_h$. The jump
$\jump{\cdot}$ and average $\avg{\cdot}$ operators on an internal edge
shared by elements $T^{+}$ and $T^{-}$ with outward normals $n^+$ and
$n^-$ respectively, are defined for generic scalar fields $\alpha$ and
vector fields $\beta$ as
\begin{align*}
  \jump{\alpha}&= \alpha^{+} n^{+} + \alpha^{-} n^{-}, \quad
  \jump{\beta}  = \beta^{+} \cdot n^{+} + \beta^{-} \cdot n^{-}, \\
  \avg{\alpha} &= \frac{1}{2} \left( \alpha^{+} + \alpha^{-} \right), \quad
  \avg{\beta}   = \frac{1}{2} \left( \beta^{+} + \beta^{-} \right), \\
  \alpha^{\pm}  &= \alpha |_{T^{\pm}}, \quad
  \beta^{\pm}    = \beta |_{T^{\pm}}.
\end{align*}

We now turn to consider the diffusive term
of~\eqref{vynnytska:eq:transdif} separately. Its standard variational
form for a symmetric discontinuous Galerkin discretization with a
stabilization penalty term is given by~\citep{Arnold1982,
  KulkarniRovasTortorelli2007}
\begin{equation}
  \label{vynnytska:eq:dgdiff}
  \begin{split}
    a_D(\composition_h, \test_h)
    =
    \sum_{T \in \triang_h} \int_T k_c \Grad \composition_h \cdot \Grad \test_h \dx
    + \sum_{e \in \Gamma_i \cup \Gamma_e}
    \int_{e} - \avg{k_c \Grad \composition_h} \cdot \jump{\test_h} \ds \\
    + \sum_{e \in \Gamma_i} \int_{e} \left(
    - \{ k_c \Grad \test_h\} \cdot \jump{\composition_h}
    + \frac{\alpha k_c}{h_T} \jump{\composition_h} \cdot \jump{\test_h}
    \right) \ds
  \end{split}
\end{equation}
where $\alpha$ is a sufficiently large constant, and $h_T$ is the size
of element $T$. Here $\Gamma_e$ denotes the facets on the (exterior)
boundary of the domain. These exterior terms drop when applying the
boundary conditions for the compositional field and are included here
merely for the sake of completeness.
Combining~\eqref{vynnytska:eq:dgadv} and~\eqref{vynnytska:eq:dgdiff},
we obtain the following spatially discrete variational formulation of
the transport equation~\eqref{vynnytska:eq:transdif}: find
$\composition_h \in P_h$ such that
\begin{equation}
  \label{vynnytska:eq:semi_composition}
  \sum_{T \in \triang_h} \int_T \frac{\partial \composition_h}{\partial t}\test_h\dx
  + a_{A} (u_h; \composition_h, \test_h) + a_D(\composition_h, \test_h) = 0.
\end{equation}
for all $\test_h \in P_h$ where $P_h$ is a finite element space of
discontinuous piecewise polynomial fields, and correspondingly for the
temperature $T_h$. In the subsequent simulations, we will let $P_h$ be
the space of (discontinuous) piecewise linears defined relative to the
partition $\triang_h$.

\subsection{A decoupling predictor-corrector scheme}

Instead of solving the fully coupled (nonlinear) system of equations
defined
by~\eqref{vynnytska:eq:momentum}~--~\eqref{vynnytska:eq:incompress},
\eqref{vynnytska:eq:energy}, and \eqref{vynnytska:eq:transdif}, we
turn to the use of a splitting scheme. In particular, we consider a
predictor--corrector scheme~\citep{BergKekenYuen1993, HansenEbel1988}
for the temperature $T$ in combination with a filtering algorithm for
the composition $\composition$. The filtering algorithm is aimed at
correcting property fields from numerical diffusion and dispersion
errors, and is motivated and described in detail
by~\citet{LenardicKaula1993}.

Before outlining the algorithm, we make some comments on the temporal
discretization of~\eqref{vynnytska:eq:semi_composition} and the
corresponding equation for the
temperature. Rewriting~\eqref{vynnytska:eq:semi_composition} as
\begin{equation}
  \frac{\partial r}{\partial t} = W,
\end{equation}
the common $\theta$-scheme for the simulation of $r$'s evolution from
time $k-1$ to time $k$ with time step $\Delta t_k$ reads:
\begin{align}
  \label{vynnytska:eq:thetascheme}
  \frac{r^k - r^{k-1}}{\Delta t_k} = \theta W^k + (1 - \theta) W^{k-1}.
\end{align}
The choice $\theta = 1$ corresponds to the implicit Euler scheme while
$\theta = 0.5$ corresponds to the Crank-Nicholson scheme. The
predictor--corrector scheme draws on the Crank-Nicholson scheme in
using a two-step procedure. Taking the energy equation for the
temperature as an example, assuming that the temperature at the
previous time $T_h^{k-1}$ and the previous velocity $u_h^{k-1}$ are
known, the predictor step computes a predicted temperature $T_h^{pr}
\in P_h$ solving the implicit Euler equations for all $\test_h \in
P_h$:
\begin{equation}
  \label{vynnytska:eq:predictor}
  \sum_{T \in \triang_h}
  \int_T \frac{T_h^{pr} - T_h^{k-1}}{\Delta t_k} \test_h \dx
  + a_{A} (u^{k-1}; T_h^{pr}, \test_h) + a_D (T_h^{pr}, \test_h) = 0
\end{equation}
 where $a_A$ and $a_D$ are defined
by~\eqref{vynnytska:eq:dgadv} and~\eqref{vynnytska:eq:dgdiff}
respectively. Later, the corrector step computes the corrected
temperature $T_h^k$ by a Crank-Nicholson scheme
\begin{equation}
  \label{vynnytska:eq:corrector}
  \begin{split}
    \sum_{T \in \triang_h}
    \int_T \frac{T_h^k - T_h^{k-1}}{\Delta t_k} \test_h \dx
    &+ 0.5 \left( a_{A} (u_h^{pr}; T_h^{k}, \test_h)
    + a_D (T_h^{k}, \test_h) \right) \\
    &+ 0.5 \left( a_{A} (u_h^{k-1}; T_h^{k-1}, \test_h)
    + a_D (T_h^{k-1}, \test_h) \right) = 0
  \end{split}
\end{equation}
but using a predicted velocity $u_h^{pr}$, which will be further
specified in the detailed algorithm, see
Algorithm~\ref{vynnytska:alg:algorithm}.
\begin{algorithm}
  \begin{tabbing}
    Initialize temperature $T^0$ and composition $\phi^0$. \\
    Compute initial velocity $v^0$ by
    solving~\eqref{vynnytska:eq:mixed} with $T^0$ and $\phi^0$. \\
    Compute time step $\Delta t_1$ from $v^0$ according
    to~\eqref{vynnytska:eq:timestep}. \\
    \textbf{for}  {$k = 1, \dots, n$}. \\
    \tab (1) Solve~\eqref{vynnytska:eq:predictor} to obtain $T_h^{pr}$. \\
    \tab (2) Solve (the composition equivalent
    of)~\eqref{vynnytska:eq:predictor} to obtain $\composition_h^{pr}$. \\
    \tab (3) Filter predicted composition $\composition_h^{pr}$ to
    obtain $\composition_h^{k}$. \\
    \tab (4) Solve~\eqref{vynnytska:eq:mixed} with $T_h^{pr}$ and
    $\composition_h^{k}$ as input to obtain $u_h^{pr}$. \\
    \tab (5) Solve~\eqref{vynnytska:eq:corrector} to obtain $T_h^{k}$. \\
    \tab (4) Solve~\eqref{vynnytska:eq:mixed} with $T_h^{k}$ and
    $\composition_h^{k}$ as input to obtain $u_h^{k}$. \\
    \tab (6) Compute new time step $\Delta t_{k+1}$ according
    to~\eqref{vynnytska:eq:timestep}. \\
    \textbf{end for}
  \end{tabbing}
  \caption{A predictor--corrector algorithm}
  \label{vynnytska:alg:algorithm}
\end{algorithm}

We shall here consider a variable time step in order to satisfy a
CFL-type stability condition. In particular, we define each time step
$\Delta t_k$ by the formula
\begin{align}
   \label{vynnytska:eq:timestep}
   \Delta t =  C_{CFL} h_{min} / \max{|u^{k-1}|}
\end{align}
where $h_{\min}$ is the minimal cell size of the mesh and $C_{CFL}$ is
a chosen positive number.

We end this section by outlining the filtering algorithm used for the
composition. The basic idea of this algorithm is to ensure that
$\composition$ remains in its initial boundaries, i.e. $0 \leq
\composition \leq 1$, and to minimize dispersion error.  We refer the
reader to~\citet{LenardicKaula1993} for the detailed explanation and
here give the outline of the algorithm for a discrete property field
$\composition = \{\composition_i \}_{i}$
\begin{algorithm}
  \begin{tabbing}
  (1) Compute initial sum $S_0$ of all values of $\composition$. \\
  (2) Find minimal value $\composition_{\min}$ below 0. \\
  (3) Find maximal value $\composition_{\max}$ above 1. \\
  (4) Assign to $\composition_i \leq | \composition_{\min} |$ value 0. \\
  (5) Assign to $\composition_i \geq 2 - \composition_{\max} $ value 1. \\
  (6) Compute sum $S_1$ of all values of $\composition$. \\
  (7) Compute the number $num$ of $0 < \composition_j < 1$. \\
  (8) Add $dist = (S_1 - S_0)/num$ to all $ 0 < \composition_j < 1$.
  \end{tabbing}
  \caption{A property filtering algorithm}
  \label{vynnytska:alg:filtering}
\end{algorithm}

%%------------------------------------------------------------------------------

\section{Implementation}

In this section, we discuss the main structure of the implementation
of the numerical scheme described above and in particular in
Algorithm~\ref{vynnytska:alg:algorithm}. We also focus on the
structure of the code for the definition of the separate variational
problems. The complete code is available from $\dots$.

\subsection{Main algorithm}
\begin{figure}
  \begin{center}
    \begin{python}
# Functions at previous time step (and initial conditions)
(phi_, T_, u_, P) = compute_initial_conditions()

# Containers for storage
velocity_series = TimeSeries("bin/velocity")
temperature_series = TimeSeries("bin/temperature")
composition_series = TimeSeries("bin/composition")

# Solver for the Stokes systems
solver = KrylovSolver("tfqmr", "amg_ml")
...
while (t <= finish):

    # Solve for predicted temperature
    (a, L) = energy(mesh, Constant(dt), u_, T_)
    eq = VariationalProblem(a, L, T_bcs)
    eq.parameters["solver"]["linear_solver"] = "gmres"
    eq.solve(T_pr)

    # Solve for predicted phi
    (a, L) = transport(mesh, Constant(dt), u_, phi_)
    eq = VariationalProblem(a, L, bc)
    eq.parameters["solver"]["linear_solver"] = "gmres"
    eq.solve(phi_pr)

    # Filter predicted phi (in place)
    filter_properties(phi_pr)
    phi.assign(phi_pr)

    # Solve for predicted velocity
    H = Ra*T_pr - Rb*phi_pr
    eta = viscosity(T_pr)
    (a, L, precond) = stokes(mesh, eta, H*g)
    (A, b) = assemble_system(a, L, bcs)
    solver.set_operators(A, P)
    solver.solve(velocity_pressure.vector(), b)
    u_pr.assign(velocity_pressure.split()[0])

    # Solve for corrected temperature T
    (a, L) = energy_correction(mesh, Constant(dt), u_pr, u_, T_)
    eq = VariationalProblem(a, L, T_bcs)
    eq.parameters["solver"]["linear_solver"] = "gmres"
    eq.solve(T)

    # Solve for corrected velocity
    H = Ra*T - Rb*phi
    eta = viscosity(T)
    (a, L, precond) = stokes(mesh, eta, H*g)
    (A, b) = assemble_system(a, L, bcs)
    solver.set_operators(A, P)
    solver.solve(velocity_pressure.vector(), b)
    u.assign(velocity_pressure.split()[0])

    # Store stuff
    composition_series.store(phi.vector(), t)
    temperature_series.store(T.vector(), t)
    velocity_series.store(u.vector(), t)

    # Define dt based on CFL condition
    dt = compute_timestep(u)

    # Move to new timestep, including updating functions
    phi_.assign(phi)
    T_.assign(T)
    u_.assign(u)
    t += dt
    \end{python}
    \caption{Abridged code for the main predictor-corrector algorithm,
      see Algorithm~\ref{vynnytska:alg:algorithm}. The initialization
      of the mesh \code{mesh}, the viscous and chemical parameters and
      the boundary conditions are omitted. The solution fields are
      consistently named \code{T}, \code{u}, and \code{phi} for
      solutions at the current time; \code{T\_}, \code{u\_}, and
      \code{phi\_} for fields at the previous time; and \code{T\_pr},
      \code{u\_pr}, and \code{phi\_pr} for predictor fields. These
      \code{Functions} are all initialized (but not included here).

      First, the initial conditions are constructed. This involves the
      solution of a Stokes system for the initial velocity \code{u\_},
      and in particular the construction of a preconditioner matrix
      \code{P}. Since the matrix is to be reused, the initial
      condition computation also return this matrix. Next,
      \code{TimeSeries} objects are initialized for easy storage of
      the solutions at each time. The \code{KrylovSolver} can also be
      reused in each iteration and is therefore created outside the
      loop.

      The contents of the loop follow steps (1) -- (6) of
      Algorithm~\ref{vynnytska:alg:algorithm}. First, the predicted
      temperature \code{T\_pr} must be computed. This is done is 4
      sub-steps: the forms for the variational problem are created;
      the forms are passed together with the boundary conditions to a
      \code{VariationalProblem}; the type of linear solver is set for
      the problem; and the problem is solved. Next, the same steps are
      performed for the predicted composition \code{phi\_pr}. The
      predicted composition is then filtered (in place) and also
      assigned to \code{phi}.

      Using the predicted temperature and filtered composition, the
      source term and viscosity for the Stokes equations are defined,
      and then the variational Stokes form is created. In order to
      retain the symmetry of the matrix under the application of
      Dirichlet boundary conditions, the linear system is assembled
      and solved explicitly. This consists of calling the method
      \code{assemble\_system}, updating the \code{KrylovSolver} with
      the current operators \code{A} and \code{P}, and applying the
      solver to the right-hand side vector. The procedure is repeated
      for the corrected temperature and again for the corrected Stokes
      system.

      Finally, the solution fields at the current time are stored, the
      new time step \code{dt} is computed based on the current velocity
      and the current solutions are assigned to the previous time as
      we step forward in time.}
    \label{vynnytska:fig:mainalgorithm}
  \end{center}
\end{figure}

The main body of the implementation consists of the temporal loop
defined in Algorithm~\ref{vynnytska:alg:algorithm}. An abridged
version of this code is listed in
Figure~\ref{vynnytska:fig:mainalgorithm}, and explained in detail in
the corresponding caption.  As the filtering step is straightforward
to implement based on the algorithm described in
Algorithm~\ref{vynnytska:alg:filtering}, we will not comment any
further on this aspect. Below, we make some general observations and
comments.
\begin{itemize}
\item
  In each iteration, five variational problems are solved. First, a
  predicted temperature is computed based on the velocity and the
  temperature from the previous time step. Next, a tentative
  composition is computed based on the velocity and the composition
  from the previous time step. This composition is then filtered using
  the filtering algorithm as described in
  Algorithm~\ref{vynnytska:alg:filtering}. With the filtered
  composition and the predicted temperature, we solve for a predicted
  velocity and pressure. We next solve for a corrected temperature by
  solving the Crank-Nicholson system with the predicted velocity, and
  the velocity and the temperature from the previous time step. The
  Stokes system is then solved again, this time with the filtered
  composition and the corrected temperature as input to yield the
  corrected velocity and pressure at this time step.
\item
  The advection-diffusion problems for the predicted composition, and
  the predicted and corrected temperature depend on the velocity at
  the previous and current current time. Analogously, the Stokes
  equations for the predicted and corrected velocity depend on the
  predicted and corrected composition and temperatures through the
  viscosity and the source terms. The linear systems of equations
  therefore have to be assembled (in addition to solved) at each
  time. For simplicity, we therefore generate the variational forms
  describing the equations and define a new \code{VariationalProblem}
  for each of these problems at each time. The compute time used for
  this is insignificant in comparison with the time required for the
  solution of the linear systems.
\item
  The linear systems resulting from the equations for the composition
  and the temperature are positive definite but not symmetric. These
  are therefore solved iteratively using a standard generalized
  minimal residual solver ("gmres"). For the simulations considered in
  the subsequent section, these solvers converge in $4 - 10$
  iterations.  On the other hand, the linear systems resulting from
  the Stokes equations are symmetric but
  indefinite. Non-preconditioned iterative solvers typically fail to
  converge for such systems, while direct solvers are prohibitively
  (memory) expensive. These systems therefore require
  preconditioning. Following Chapter~\ref{chap:mardal-4}, we here take
  advantage of a standard Stokes preconditioner. Although the
  viscosities vary, we use the same preconditioner at each
  time. Hence, we can assemble the preconditioner matrix outside the
  loop and reuse it and the Krylov solver in each iteration.
\item
  The time step \code{dt} is computed adaptively in each iteration of
  the temporal loop using the
  formula~\eqref{vynnytska:eq:timestep}. The minimal mesh size is
  easily extracted using \code{mesh.hmin()} and the maximal value for
  the velocity is extracted as the maximal degree of freedom from the
  \code{numpy} array of degrees of freedom. Since the time step hence
  varies in each iteration and with the mesh size, it is convenient to
  use a \code{Constant} for its representation in order to avoid
  recompilation of the variational forms at each iteration.
\item
  The solutions for the composition, the temperature and the velocity
  are stored at each time using the \code{TimeSeries} class. This
  class allows for easy storage and retrieval of meshes and solution
  vectors. Moreover, it naturally encourages a decoupling of the
  simulation and the post-processing of the simulation data. This can
  be highly advantageous especially for computations with significant
  run times.
\end{itemize}
In the subsections below, we discuss the definition of each of the
variational forms and problems and an implementational structure for
these.

\subsection{Variational formulation of the Stokes equations}

The mixed variational formulation for the Stokes equations is
classical and listed in Figure~\ref{vynnytska:fig:stokes}. The
definition of the bilinear and linear forms rely only on the mesh, the
viscosity $\eta$ and a source vector field $f$, which in our case
takes the form $f = (Ra T_h - Rb \composition_h) e$
cf~\eqref{vynnytska:eq:mixed}. In particular, as the rheology
considered is Newtonian and the viscosity thus does not depend on the
velocity, the actual definition of the viscosity can be treated
separately. Since the preconditioner for this system relies on the
same function spaces and basis functions, we define the form for the
preconditioner together with the variational forms describing the
differential equation.
\begin{figure}
  \begin{python}
def stokes(mesh, eta, f):

    # Define spatial discretization (Taylor--Hood)
    V = VectorFunctionSpace(mesh, "CG", 2)
    Q = FunctionSpace(mesh, "CG", 1)
    W = V * Q

    # Define basis functions
    (u, p) = TrialFunctions(W)
    (v, q) = TestFunctions(W)

    # Define equation F((u, p), (v, q)) = 0
    F = (2.0*eta*inner(sym(grad(u)), sym(grad(v)))*dx
         + div(v)*p*dx
         + div(u)*q*dx
         + inner(f, v)*dx)

    # Define form for preconditioner
    precond = inner(grad(u), grad(v))*dx + p*q*dx

    # Return right and left hand side and preconditioner
    return (lhs(F), rhs(F), precond)
  \end{python}
  \caption{Definition of variational forms for the Stokes equations
    and the corresponding preconditioner. The Taylor--Hood mixed
    finite element space is defined by combining Lagrange vector
    elements of polynomial order 2 with Lagrange elements of order
    1. The equation is phrased in the style $F(\cdot, \cdot) = 0$, and
    the form for the preconditioner is defined using the same basis
    functions. The left- and right-hand side forms are extracted using
    the UFL functions \code{lhs} and \code{rhs}.}
  \label{vynnytska:fig:stokes}
\end{figure}

\subsection{Variational formulation of advection-diffusion equations}

In the implementation of the variational forms for the
advection-diffusion problems, we emphasized the following
points. First, the variational forms for the predictor step of both
the temperature and the composition are the same (modulo a different
value of the diffusivity and possibly penalty constant).  Second, the
predictor and the corrector steps for the temperature involve the same
mathematical building blocks. Third, discontinuous Galerkin methods
often involve quite a number of terms and the combined forms may
easily become intractable. In view of these aspects, a minimal (as in
highly reused) code close to mathematical syntax seemed preferable.

To this end, the implementation mimics the structure defined
by~\eqref{vynnytska:eq:dgadv},~\eqref{vynnytska:eq:dgdiff}, and
\eqref{vynnytska:eq:thetascheme}. The pure advection form $a_A$ and
the pure diffusion form $a_D$ are defined through separate python
functions. The code for these are listed in
Figures~\ref{vynnytska:fig:advection}
and~\ref{vynnytska:fig:diffusion} and explained in their captions. The
implementation of the weak forms for the predictor equation for the
composition and the predictor and corrector equations for the
temperature then build on these basic functions. The code for the
corrector equation is included and discussed
in~Figure~\ref{vynnytska:fig:temperaturecorrection}. The code for the
predictor equations is similar and simpler and therefore not discussed
here.

\begin{figure}
  \begin{center}
    \begin{python}
def advection(phi, psi, u, n, theta=1.0):

    # Define |u * n|
    un = abs(dot(u('+'), n('+')))

    # Contributions from cells
    a_cell = - theta*dot(u*phi, grad(psi))*dx

    # Contribution from interior facets
    a_int = theta*(dot(u('+'), jump(psi, n))*avg(phi)
                   + 0.5*un*dot(jump(phi, n), jump(psi, n)))*dS

    return a_cell + a_int
    \end{python}
    \caption{Definition of an upwinded discontinuous Galerkin
      formulation of the advection term. The input consists of
      \code{phi} and \code{psi} (typically functions or basis
      functions), a given velocity \code{u}, a facet normal \code{n}
      and an optional scalar multiplier \code{theta}. The absolute
      value of the normal component of \code{u} is defined and the
      contributions from the cell integrals and facet integrals are
      defined in accordance with~\eqref{vynnytska:eq:dgadv}. The sum
      of the contributions is returned.}
    \label{vynnytska:fig:advection}
  \end{center}
\end{figure}
\begin{figure}
  \begin{center}
    \begin{python}
def diffusion(phi, psi, k_c, alpha, n, h, theta=1.0):

    # Contribution from the cells
    a_cell = theta*k_c*dot(grad(phi), grad(psi))*dx

    # Contribution from the interior facets
    a_int = theta*(k_c('+')*alpha('+')/h('+')*dot(jump(phi, n), jump(psi, n))
                   - k_c('+')*dot(avg(grad(psi)), jump(phi, n))
                   - k_c('+')*dot(jump(psi, n), avg(grad(phi))))*dS

    return a_cell + a_int
    \end{python}
    \caption{Definition of a discontinuous Galerkin formulation of the
      diffusion term. The input consists of \code{phi} and \code{psi}
      (typically functions or basis functions), the diffusivity
      constant \code{k\_c}, a stabilization parameter \code{alpha}, a
      facet normal \code{n}, the cell size \code{h}, and an optional
      scalar multiplier \code{theta}. The contributions from the cell
      integrals and facet integrals are defined
      following~\eqref{vynnytska:eq:dgdiff}. The sum of the
      contributions is returned.}
    \label{vynnytska:fig:diffusion}
  \end{center}
\end{figure}

\begin{figure}
  \begin{center}
    \begin{python}
def energy_correction(mesh, dt, u, u_, T_):

    # Define function space and test and trial functions
    P = FunctionSpace(mesh, "DG", 1)
    T = TrialFunction(P)
    psi = TestFunction(P)

    # Diffusivity constant
    k_c = Constant(1.0)

    # Constants associated with DG scheme
    alpha = Constant(50.0)
    h = CellSize(mesh)
    n = FacetNormal(mesh)

    # Define discrete time derivative operator
    Dt =  lambda T: (T - T_)/dt

    # Add syntactical sugar for advection and diffusion terms
    a_A = lambda u, T, psi: advection(T, psi, u, n, theta=0.5)
    a_D = lambda u, T, psi: diffusion(T, psi, k_c, alpha, n, h, theta=0.5)

    # Define form
    F = Dt(T)*psi*dx + a_A(u, T, psi) + a_A(u_, T_, psi) \
        + a_D(u, T, psi) + a_D(u_, T_, psi)

    return (lhs(F), rhs(F))
    \end{python}
    \caption{Definition of variational forms for one correction step
      for the temperature, see~\eqref{vynnytska:eq:corrector}. The
      input is the mesh, the time step \code{dt}, two given velocities
      \code{u} and \code{u\_}, and (typically) a previous temperature
      \code{T\_}. We define the function space of (discontinuous)
      piecewise linears, the unknown temperature \code{T} and the test
      function \code{psi}. The diffusivity constant \code{k\_c} is in
      this case $1.0$. We define the penalty parameter \code{alpha}
      and also the cell normal \code{n} and mesh size \code{h} to be
      used in the advection and diffusion forms.

      We use \code{lambda} functions to reduce the number of input
      arguments to the advection and diffusion functions. This is in
      part merely syntactical, but does also increase readability and
      facilitates debugging. The input to the functions \code{a\_A}
      and \code{a\_D} thus directly corresponds to the arguments of
      $a_A$ and $a_D$.

      The equation is again phrased in the style $F(\cdot, \cdot) =
      0$. The reader is encouraged to compare the code with the
      mathematical formulation of the equation,
      cf.~\eqref{vynnytska:eq:corrector}. Finally, the left- and
      right-hand side forms are extracted using the UFL functions
      \code{lhs} and \code{rhs}.}
    \label{vynnytska:fig:temperaturecorrection}
  \end{center}
\end{figure}

%%------------------------------------------------------------------------------

\section{Results}
\label{vynnytska:sec:results}

The equations presented in Section \ref{vynnytska:sec:maths} are
dimensionalised using the physical parameters in
Table \ref{vynnytska:table:variables}.
\begin{table}[htbp]
\caption{Specification of parameters and parameter values.}
  \begin{tabular}{l c r r}
    Parameter & & Dimensional  & Dimensionless  \\
              &  & value & value\\
    \toprule
    Box height & $h$ & $3000km$ & $1.0$\\
    Box length & $\lambda$ & $6000km$ & $2.0$ \\
    Boundary layer thickness & $d$ & $150km$ & $0.15$ \\
    Acceleration due to gravity & $g$ & $10m/s^2$ & $1.0$ \\
    Thermal contrast & $\Delta T$ & $3000K$ & $1.0$ \\
    Thermal expansivity & $\alpha$ & $2 \times 10^{-5}K^{-1}$ & \\
    Thermal diffusivity & $k_{th}$ & $10^{-6}m^{2}/s$ & \\
    Chemical diffusivity & $k_{ch}$ & $10^{-10}m^{2}/s$ & \\
    Reference density & $\rho_{0}$ & $3100kg/m^{3}$ & $0.0$ \\
    Density contrast & $\Delta \rho $ & $185kg/m^{3}$ & $1.0$ \\
    Reference viscosity & $\eta_{0}$ & $5 \times 10^{22} Pa \cdot s$ & $1.0$ \\
    Thermal Rayleigh number & $Ra$ & $1 \times 10^{6}$ & $ 1 \times 10^{6}$\\
    Chemical Rayleigh number & $Rb$ & $1 \times 10^{6}$ & $1 \times 10^{6}$\\
    Velocity & $u_s$ & $6 \times 10^{-13} m/s$& $1.0$ \\
    Time & $t_s$ & $5 \times 10^{18} s$& $1.0$ \\
    \bottomrule
\end{tabular}
\label{vynnytska:table:variables}
\end{table}

Scaling parameters for time, $t_s$, and velocity, $u_s$, are obtained
as follows:
\begin{align}
 \label{vynnytska:eq:velsc}
 t_s = \frac{\rho_0 \eta_0}{h g}, \quad t_s = \frac{h}{u_s}.
\end{align}
The vigor of convection in the model domain is determined by the
kinematic energy of the fluid, given by:
\begin{align}
\label{vynnytska:eq:KE}
E_{K} = \frac{1}{2}\int_{\Omega}\rho \Vert \bm{u} \Vert^{2}dx.
\end{align}
Since the variation in density is small, $\rho$ can be taken out of
the integral, and by defining the root-mean square velocity, $u_{rms}$
by:
\begin{align}
\label{vynnytska:eq:u_rms}
u_{rms} =\left( \frac{1}{\lambda h} \int_{\Omega} \Vert \bm{u} \Vert^{2} d\bm{x}  \right)^{1/2},
\end{align}
we have the relationship:
\begin{align}
  \label{vynnytska:eq:KEu_rms}
  E_{K} = \frac{1}{2} \rho \lambda h u_{rms}^{2}.
\end{align}
Thus, $u_{rms}$ is a scaled measure of the kinematic energy of
convection. For the discussion of the results, we will use Figure
\ref{vynnytska:fig:rms_velocity} to describe the local turning points,
$A$ to $M$, of the root-mean square velocity and refer to the driving
forces in the model to explain these turning points. See the captions
to Figures \ref{vynnytska:fig:BG} and \ref{vynnytska:fig:HM} for
details.


\begin{figure}[htbp]
  \begin{center}
   \includegraphics[width=0.95\columnwidth]{chapters/vynnytska/figures/rms_velocity.png}
    \caption{Dimensionless root mean square velocity over
    dimensionless time. $A$ to $M$ are labels used to refer to stages
    in the model.}
  \end{center}
\label{vynnytska:fig:rms_velocity}
\end{figure}

\begin{figure}[htbp]
\begin{center}
\begin{tabular}{c c l}
Temperature & Viscosity \\
\includegraphics[width=0.45\columnwidth]{chapters/vynnytska/figures/tmB.png} &
\includegraphics[width=0.45\columnwidth]{chapters/vynnytska/figures/visB.png} & $B$ \\
\includegraphics[width=0.45\columnwidth]{chapters/vynnytska/figures/tmC.png} &
\includegraphics[width=0.45\columnwidth]{chapters/vynnytska/figures/visC.png} & $C$ \\
\includegraphics[width=0.45\columnwidth]{chapters/vynnytska/figures/tmD.png} &
\includegraphics[width=0.45\columnwidth]{chapters/vynnytska/figures/visD.png} & $D$ \\
\includegraphics[width=0.45\columnwidth]{chapters/vynnytska/figures/tmE.png} &
\includegraphics[width=0.45\columnwidth]{chapters/vynnytska/figures/visE.png} & $E$ \\
\includegraphics[width=0.45\columnwidth]{chapters/vynnytska/figures/tmF.png} &
\includegraphics[width=0.45\columnwidth]{chapters/vynnytska/figures/visF.png} & $F$ \\
\includegraphics[width=0.45\columnwidth]{chapters/vynnytska/figures/tmG.png} &
\includegraphics[width=0.45\columnwidth]{chapters/vynnytska/figures/visG.png} & $G$ \\
\includegraphics[width=0.45\columnwidth]{chapters/vynnytska/figures/tmleg.png} &
\includegraphics[width=0.45\columnwidth]{chapters/vynnytska/figures/visleg.png} &
\end{tabular}
\end{center}
\label{vynnytska:fig:BG}
\caption{Dimensionless temperature and viscosity for turning points
  $A$ to $G$, with composition barrier shown in white. The initial
  condition drives the system's high velocity from point $A$, but as
  the cold surface material (slab) reaches the deeper mantle, there is
  a retardation of the flow, towards $B$. The rising hot material
  (plume) during the stage $A$ to $B$ drives lateral flow of the
  surface causing cold material to build up until $B$. From $B$ to $C$
  this cold material begins to rapidly sink, increasing $u_{rms}$
  until it is slowed by the increasing viscosity at $C$. However, the
  slab's downward motion has created a thermal instability at the base
  of the model, which rises between $C$ and $D$ increasing $u_{rms}$
  again. The pace of the material slows as the plume necks between $D$
  and $E$ and the compositional density of the remaining material
  prevents it from rising further. A small plume rises from the left
  side of the base, increasing the $u_{rms}$ briefly. Between $E$ and
  $G$ short-lived plumes and slabs are generated from the bottom and
  top boundary layers, causing small instabilities in the root-mean
  square velocity.}
\end{figure}
\begin{figure}[htbp]
\begin{center}
\begin{tabular}{c c l}
Temperature & Viscosity \\
\includegraphics[width=0.45\columnwidth]{chapters/vynnytska/figures/tmH.png} &
\includegraphics[width=0.45\columnwidth]{chapters/vynnytska/figures/visH.png} & $H$ \\
\includegraphics[width=0.45\columnwidth]{chapters/vynnytska/figures/tmI.png} &
\includegraphics[width=0.45\columnwidth]{chapters/vynnytska/figures/visI.png} & $I$ \\
\includegraphics[width=0.45\columnwidth]{chapters/vynnytska/figures/tmJ.png} &
\includegraphics[width=0.45\columnwidth]{chapters/vynnytska/figures/visJ.png} & $J$ \\
\includegraphics[width=0.45\columnwidth]{chapters/vynnytska/figures/tmK.png} &
\includegraphics[width=0.45\columnwidth]{chapters/vynnytska/figures/visK.png} & $K$ \\
\includegraphics[width=0.45\columnwidth]{chapters/vynnytska/figures/tmL.png} &
\includegraphics[width=0.45\columnwidth]{chapters/vynnytska/figures/visL.png} & $L$ \\
\includegraphics[width=0.45\columnwidth]{chapters/vynnytska/figures/tmM.png} &
\includegraphics[width=0.45\columnwidth]{chapters/vynnytska/figures/visM.png} & $M$ \\
\includegraphics[width=0.45\columnwidth]{chapters/vynnytska/figures/tmleg.png} &
\includegraphics[width=0.45\columnwidth]{chapters/vynnytska/figures/visleg.png} &
\end{tabular}
\end{center}
\label{vynnytska:fig:HM}
\caption{Continued from \ref{vynnytska:fig:BG}, temperature and
  viscosity for points $H$ to $M$. During the stage $G$-$H$, a second
  slab forms and the two downwellings merge increasing the root-mean
  square velocity rapidly. From $H$, the downwelling is impeded by the
  higher viscosity at depth, reducing $u_{rms}$ until $I$. Then a
  plume arising from the bottom left rises more rapidly through the
  lower viscosities of the upper mantle until $J$. Between $J$ and
  $K$, no new up- or downwellings occur, retarding the root-mean
  square velocity. From $K$ to $L$ a new plume increases the kinetic
  energy in the model, then pushes material laterally from underneath
  the top boundary layer until a slab begins to sink at the edge from
  $M$, increasing the velocity again.}
\end{figure}

%%------------------------------------------------------------------------------

\section{Conclusion}

The results presented above show the mantle convecting with two
distinct chemical layers: plumes arise from atop piles on the bottom
denser layer, but are not compositionally distinct. The location of
these piles is initially set by the thermally dense slabs. Slabs
collapsing into the mantle drive the largest changes in the system
energy, while plumes drive smaller increases because of the
composition counteracting the thermal buoyancy. The upwellings and
downwellings react: slabs rapidly sinking cause upwellings to form;
the lateral motion of upwellings at the top pushes and thickens the
top layer, causing it to become unstable and sink. As the system
evolves, colder slab material builds up at the bottom, increasing the
viscosity of the lower mantle, while the reverse happens in the upper
mantle.

The code required for this simulation has been included almost in its
entirety. We can conclude that the amount of code required to
implement such a problem within the FEniCS framework is quite
small. Moreover, the code for the variational problems closely matches
the mathematical formulation of the problems, and thus the complexity
of the code scales with the complexity of the numerical algorithm.
The numerical simulations presented here are spatially two-dimensional
and serve as a simplified model. Realistic three-dimensional
simulations would require taking advantage of the parallel, and
possibly more sophisticated adaptive, features of the FEniCS
project. However, such would not require significant additional
problem-specific implementational effort.

We would like to stress that compositional variation which is
presented in the considered problem requires solving
equation~\eqref{vynnytska:eq:transdif}.  Solving this equation is
challenging for numerical schemes as requires tracking of sharp
interfaces.  Field, markers and marker chain approaches are proposed
in the literature to trace compositional
discontinuities~\cite{IsmailZadehTackley2010}. We used field approach
to represent compositional heterogeneity which is easily implemented by
the means of DG FEM. Additionally, filtering scheme is implemented in
order to minimize numerical dispersion error which is the main
disadvantage of the field approach.

%%------------------------------------------------------------------------------

