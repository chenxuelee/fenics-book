\fenicschapter{Finite element assembly}
              {Finite element assembly}
              {Anders Logg, Kent-Andre Mardal and Garth N. Wells}
              {logg-3}

The finite element method may be viewed as a method for forming a
discrete linear system $AU = b$ or nonlinear system $b(U) = 0$
corresponding to the discretization of the variational form of a
differential equation. A central part of the implementation of finite
element methods is therefore the computation of matrices and vectors
from variational forms. In this chapter, we describe the standard
algorithm for computing the discrete operator (tensor) $A$ defined in
Chapter~\ref{chap:kirby-5}. This algorithm is known as \emph{finite
element assembly}. We also discuss efficiency aspects of the
standard algorithm and extensions to matrix-free methods.

%------------------------------------------------------------------------------
\section{Assembly algorithm}

As seen in Chapter~\ref{chap:kirby-5}, the discrete operator of a
multilinear form~$a : V_{\rho} \times \cdots \times V_2 \times V_1
\rightarrow \R$ of arity~$\rho$ is the rank~$\rho$ tensor~$A$ defined
by
\begin{equation}
  A_I = a(\phi^{\rho}_{I_{\rho}}, \ldots, \phi^2_{I_2}, \phi^1_{I_1}),
\end{equation}
where $I = (I_1, I_2, \ldots, I_{\rho})$ is a multi-index of
length~$\rho$ and $\{\phi^j_k\}_{k=1}^{N_j}$ is a basis for $V_{j,h}
\subset V_j$, $j = 1,2,\ldots,\rho$. The discrete operator is a
typically sparse tensor of rank~$\rho$ and dimension $N_1 \times N_2
\times \cdots \times N_{\rho}$.

A straightforward algorithm to compute the tensor~$A$ is to iterate
over all its entries and compute them one by one as outlined in
Algorithm~\ref{alg:assembly,naive}. This algorithm has two major
drawbacks and is rarely used in practice. First, it does not take into
account that most entries of the sparse tensor~$A$ may be
zero. Second, it does not take into account that each entry is
typically a sum of contributions (integrals) from the set of cells
that form the support of the basis functions $\phi^1_{I_1},
\phi^2_{I_2}, \ldots, \phi^{\rho}_{I_\rho}$. As a result, each cell of
the mesh must be visited multiple times when computing the local
contribution to different entries of the tensor~$A$.  For this reason,
the tensor~$A$ is usually computed by iterating over the cells of the
mesh and adding the contribution from each local cell to the global
tensor~$A$. To see how the tensor~$A$ can be decomposed as a sum of
local contributions, we recall the definition of the cell tensor~$A_T$
from Chapter~\ref{chap:kirby-5}:
\begin{equation}
  A_{T,i} = a_T(\phi^{T,\rho}_{i_{\rho}}, \ldots, \phi^{T,2}_{i_2},
  \phi^{T,1}_{i_1}),
\end{equation}
where $A_{T,i}$ is the $i$th multi-index of the rank~$\rho$ tensor
$A_T$, $a_T$ is the local contribution to the multilinear form from a
cell~$T\in\mathcal{T}_h$ and $\{\phi^{T,j}_k\}_{k=1}^{n_j}$ is the
local finite element basis for $V_{j,h}$ on~$T$. We assume here that
the multilinear form is expressed as an integral over the
domain~$\Omega$ so that it may be naturally decomposed as a sum of
local contributions. If the form contains contributions from facet or
boundary integrals, one may similarly decompose the multilinear form
into local contributions from facets.

\begin{algorithm}
  \begin{tabbing}
    \textbf{for} {$I_1 = 1,2,\ldots,N_1$}\\
    \tab \textbf{for} {$I_2 = 1,2,\ldots,N_2$}\\
    \tab \tab \textbf{for} \ldots \\
    \tab \tab \tab $A_I = a(\phi^{\rho}_{I_{\rho}}, \ldots, \phi^2_{I_2}, \phi^1_{I_1})$
  \end{tabbing}
  \caption{Straightforward (naive) ``assembly'' algorithm.}
  \label{alg:assembly,naive}
\end{algorithm}

To formulate the general assembly algorithm, let $\iota_T^j : [1,n_j]
\rightarrow [1,N_j]$ denote the local-to-global mapping introduced in
Chapter~\ref{chap:kirby-7} for each discrete function space $V_{j,h}$,
$j=1,2,\ldots,\rho$, and define for each $T \in \mathcal{T}_h$ the
collective local-to-global mapping $\iota_T : \mathcal{I}_T
\rightarrow \mathcal{I}$ by
\begin{equation}
  \iota_T(i) =
  (\iota_T^1(i_1),\iota_T^2(i_2),\ldots,\iota_T^{\rho}(i_{\rho}))
  \quad \foralls i \in \mathcal{I}_T,
\end{equation}
where $\mathcal{I}_T$ is the index set
\begin{equation}
  \mathcal{I}_T =
  \prod_{j=1}^{\rho}[1,n_j] = \{(1,1,\ldots,1), (1,1,\ldots,2), \ldots,
  (n_1, n_2, \ldots, n_{\rho} - 1),
  (n_1, n_2, \ldots, n_{\rho})\}.
\end{equation}
That is, $\iota_T$ maps a tuple of local degrees of freedom to a tuple
of global degrees of freedom. Furthermore, let $\mathcal{T}_I \subset
\mathcal{T}_h$ denote the subset of cells of the mesh on which
$\{\phi_{i_j}^j\}_{j=1}^{\rho}$ are all nonzero. We note that
$\iota_T$ is invertible if $T \in \mathcal{T}_I$.  We may now compute
the tensor~$A$ by summing local contributions from the cells of the
mesh:
\begin{equation}
  \begin{split}
  A_I
  &=
  \sum_{T\in\mathcal{T}_h}
  a_T(\phi_{I_{\rho}}^{\rho}, \ldots, \phi_{I_2}^2, \phi_{I_1}^1)
  =
  \sum_{T\in\mathcal{T}_I}
  a_T(\phi_{I_{\rho}}^{\rho}, \ldots, \phi_{I_2}^2, \phi_{I_1}^1) \\
  &=
  \sum_{T\in\mathcal{T}_I}
  a_T(\phi_{(\iota_T^{\rho})^{-1}(I_{\rho})}^{T,{\rho}},
      \ldots,
      \phi_{(\iota_T^2)^{-1}(I_2)}^{T,2},
      \phi_{(\iota_T^1)^{-1}(I_1)}^{T,1})
  =
  \sum_{T\in\mathcal{T}_I}
  A_{T,{\iota_T^{-1}(I)}}.
  \end{split}
\end{equation}
This computation may be carried out efficiently by a single iteration
over all cells~$T \in \mathcal{T}_h$. On each cell~$T$, the cell
tensor~$A_T$ is computed and then added to the global tensor~$A$ as
outlined in Algorithm~\ref{alg:assembly} and illustrated in
Figure~\ref{fig:insertion}.
\begin{algorithm}
  \begin{tabbing}
    $A = 0$\\
    \textbf{for}  {$T \in \mathcal{T}_h$}\\
    \tab (1) Compute $\iota_T$ \\
    \tab (2) Compute $A_T$ \\ \\
    \tab (3) Add $A_T$ to $A$ according to $\iota_T$: \\
    \tab \textbf{for} $i \in \mathcal{I}_T$ \\
    \tab \tab $A_{\iota_T(i)} \stackrel{+}{=} A_{T,i}$ \\
    \tab \textbf{end for} \\
    \textbf{end for}
  \end{tabbing}
  \caption{Assembly algorithm}
  \label{alg:assembly}
\end{algorithm}

\begin{figure}
  \center\fenicsfig{logg-3}{insertion}{\largefig}
    \caption{Adding the entries of the cell tensor~$A_T$ to the global
      tensor~$A$ using the local-to-global mapping $\iota_T$,
      illustrated here for the assembly of a rank two tensor (matrix)
      with piecewise linear elements on triangles. On each element
      $T$, a $3 \times 3$ element matrix~$A_T$ is computed and its
      entries are added to the global matrix. The entries of the first
      row are added to row $\iota^1_T(1)$ of the global matrix in the
      columns given by $\iota^2_T(1)$, $\iota^2_T(2)$ and
      $\iota^2_T(3)$, respectively. The entries of the second row are
      added to row $\iota^1_T(2)$ of the global matrix etc.}
    \label{fig:insertion}
\end{figure}

%------------------------------------------------------------------------------
\section{Implementation}

In FEniCS, the assembly algorithm (Algorithm~\ref{alg:assembly}) is
implemented as part of DOLFIN (see
Figure~\ref{fig:assembly,code}). For the steps (1), (2) and (3) of the
assembly algorithm, DOLFIN relies on external code. For steps (1) and
(2), DOLFIN calls code generated by a form compiler such as FFC or
SyFi. In particular, DOLFIN calls the two functions
\emp{tabulate\_dofs} and \emp{tabulate\_tensor} through the UFC
interface for steps (1) and (2), respectively. Step (3) is carried out
through the DOLFIN \emp{GenericTensor::add} interface and maps to the
corresponding operation in one of a number of linear algebra backends,
such as \emp{MatSetValues} for PETSc and \emp{SumIntoGlobalValues}
for Trilinos/Epetra.

\begin{figure}
  \begin{c++}
for (CellIterator cell(mesh); !cell.end(); ++cell)
{
  ...

  // Get local-to-global dofmap for each dimension
  for (uint i = 0; i < form_rank; ++i)
    dofs[i] = &(dofmaps[i]->cell_dofs(cell->index()));

  // Tabulate cell tensor
  integral->tabulate_tensor(ufc.A.get(),
                            ufc.w,
                            ufc.cell);

  // Add entries to global tensor
  A.add(ufc.A.get(), dofs);
}
\end{c++}
  \caption{Actual implementation (excerpt) of the assembly algorithm
    (Algorithm~\ref{alg:assembly}) in DOLFIN (from \emp{Assembler.cpp}
    in DOLFIN~1.0).}
  \label{fig:assembly,code}
\end{figure}

In typical assembly implementations, the computation of the
cell tensor~$A_T$ is the most costly operation of the assembly
algorithm. For \dolfin{}, however, as a result of optimized algorithms
for the computation of $A_T$ being generated by form compilers (see
Chapters~\ref{chap:oelgaard-2} and~\ref{chap:kirby-8}), adding entries of
the local tensor $A_T$ to appropriate positions in the global tensor~$A$
often constitutes a significant portion of the total assembly time. This
operation is costly since the addition of a value to an arbitrary entry
of a sparse tensor is not a trivial operation, even when the layout of
the sparse matrix has been initialized.  In the standard case when $A$
is a sparse matrix (a rank two tensor), the linear algebra backend stores
the sparse matrix in \emph{compressed row storage} (CRS) format or some
other sparse format. For each given entry, the linear algebra backend
must search along a row~$I$ to find the position to store the value for
a given column~$J$. As a result, the speed of assembly in FEniCS for
sparse matrices is currently limited by the speed of insertion into a
sparse linear algebra data structure for many problems. An additional
cost is associated with the initialization of a sparse matrix, which
involves the computation of a sparsity pattern. For most linear algebra
libraries, it is necessary to initialize the layout of a sparse matrix
before inserting entries in order to achieve tolerable insertion speed.
Computation of the sparsity pattern is a moderately costly operation,
but which in the case of nonlinear problems is usually amortized over time.

Algorithm~\ref{alg:assembly} may be easily extended to assembly over
the facets of a mesh. Assembly over facets is necessary both for
handling variational forms that contain integrals over the boundary of
a mesh (the exterior facets), to account for Neumann boundary
conditions, and forms that contain integrals over the interior facets
of a mesh as part of a discontinuous Galerkin formulation. For this
reason, DOLFIN implements three different assembly algorithms. These
are assembly over cells, exterior facets and interior facets.

%------------------------------------------------------------------------------
\section{Symmetric application of boundary conditions}

For symmetric problems, it is useful to be able to apply Dirichlet
boundary conditions in a fashion that preserves the symmetry of the
matrix, since that allows the use of solution algorithms which are
limited to symmetric matrices, such as the conjugate gradient method
and Cholesky decomposition. The symmetric application of boundary
conditions may be handled by modifying the cell tensors $A_T$ before
assembling into the global tensor~$A$. Assembly with symmetric
application of boundary conditions is implemented in DOLFIN in the
class \emp{SystemAssembler}.

To explain the symmetric assembly algorithm, consider the global
system $AU=b$ and the corresponding element matrix $A_T$ and element
vector~$b_T$. If a global index $I$ is associated with a Dirichlet
boundary condition, $U_I=D_I$, then this condition can be enforced by
setting $A_{II} = 1$, $A_{IJ} = 0$ for $I \neq J$, and $b_I =
D_I$. This approach is applied when calling the DOLFIN function
\emp{DirichetBC::apply}. However, to preserve symmetry of the matrix,
we can perform a partial Gaussian elimination to obtain $A_{JI} =
A_{IJ} = 0$ for $I \ne J$. This is achieved by subtracting the $I$th
row multiplied by $A_{JI}$ from the $J$th equation, locally. These
partial Gaussian eliminations are performed on the linear systems at
the element level. The local linear systems are then added to the
global matrix. As a result, the Dirichlet condition is added multiple
times to the global vector, one time for each cell, which is
compensated for by the addition of one multiple times to the
corresponding diagonal entry of~$A$. This is summarized in
Algorithm~\ref{alg:assembly:sym}. Alternatively, one may choose to
eliminate degrees of freedom corresponding to Dirichlet boundary
conditions from the linear system (since these values are known). The
values then end up in the right-hand side of the linear system. The
described algorithm does not eliminate the degrees of freedom
associated with a Dirichlet boundary condition. Instead, these degrees
of freedom are retained to preserve the dimension of the linear system
so that it always matches the total number of degrees of freedom for
the solution (including known Dirichlet values).

\begin{algorithm}
  \begin{tabbing}
    $A = 0$ and $b = 0$\\\
    \textbf{for}  {$T \in \mathcal{T}_h$}\\
    \tab (1) Compute $\iota^A_T$ and $\iota^b_T$  \\
    \tab (2) Compute $A_T$ and $b_T$ \\
    \tab (3) Apply Dirichlet boundary conditions to $A_T$ and $b_T$ \\
    \tab (4) Perform partial Gaussian elimination on $A_T$ and $b_T$ to preserve symmetry \\
    \tab (5) Add $A_T$ and $b_T$ to $A$ and $b$ according to $\iota^A_T$ and $\iota^b_T$, respectively: \\
    \tab \textbf{for} $(i,j) \in \mathcal{I}^A_T$ \\
    \tab \tab $A_{\iota^{A,1}_T(i), \iota^{A,2}_T(j)} \stackrel{+}{=} A_{T,ij}$ \\
    \tab \textbf{end for} \\
    \tab \textbf{for} $i \in \mathcal{I}^b_T$  \\
    \tab \tab $b_{\iota^b_T(i)} \stackrel{+}{=} b^{\top}_i$  \\
    \tab \textbf{end for} \\
    \textbf{end for}
  \end{tabbing}
  \caption{Symmetric assembly algorithm ($\rho = 2$)}
  \label{alg:assembly:sym}
\end{algorithm}

%------------------------------------------------------------------------------
\section{Parallel assembly}

The assembly algorithms remain unchanged in a distributed\footnote{By
  distributed assembly, we refer here to assembly in parallel on a
  distributed memory parallel architecture, running multiple processes
  that cannot access the same memory, but must pass data as messages
  between processes.} parallel environment if the linear algebra
backend supports distributed matrices and insertion for both on- and
off-process matrix entries, and if the mesh data structure supports
distributed meshes. Both PETSc
\citep{BalayBuschelmanGroppEtAl2001,BalayBuschelmanEijkhoutEtAl2004}
and Trilinos/Epetra \citep{HerouxBartlettHowleEtAl2005} support
distributed matrices and vectors. Efficient parallel assembly relies
on appropriately partitioned meshes and properly distributed
degree-of-freedom maps to minimize inter-process communication.  It is
not generally possible to produce an effective degree-of-freedom map
using only a form compiler, since the degree-of-freedom map should
reflect the partitioning of the mesh. Instead, one may use a
degree-of-freedom map generated by a form compiler to construct a
suitable map at run-time.  \dolfin{} supports distributes meshes and
computes distributed degree of freedom maps for distributed assembly.

Multi-threaded\footnote{By multi-threaded assembly, we refer here to
  assembly in parallel on a shared memory parallel architecture,
  running multiple threads that may access the same memory.} assembly
  is outwardly simpler than distributed assembly and is attractive
  given the rapid growth in multi-core architectures. The assembly
  code can be easily modified, using for example OpenMP, to
  parallelize the assembly loop over cells. Multi-threaded assembly
  requires extra care so that multiple threads don't write to the same
  memory location (when multiple threads attempt to write to the same
  memory location, this is know as \emph{race condition}.
  Multi-threaded assembly has recently been implemented in DOLFIN
  (from version 1.0) based on coloring the cells of the mesh so that
  no two neighboring cells (cells that share a common vertex in the
  case of Lagrange elements) have the same color. One may then iterate
  over the colors of the mesh, and for each color use OpenMP to
  parallelize the assembly loop. This ensures that no two cells will
  write data from the same location (in the mesh), or write data to
  the same location (in the global tensor).

%------------------------------------------------------------------------------
\section{Matrix-free methods}

A feature of Krylov subspace methods and some other iterative methods
for linear systems of the form $AU = b$ is that they rely only on the
\emph{action} of the matrix operator $A$ on vectors and do not require
direct manipulation of~$A$. This is in contrast with direct linear
solvers. Therefore, if the action of $A$ on an arbitrary vector $v$
can be computed, then a Krylov solver can be used to solve the system
$AU = b$ without needing to assemble~$A$. This matrix-free approach
may be attractive for problem types that are well-suited to Krylov
solvers and for which the assembly of $A$ is costly (in terms of CPU
time and/or memory).  A disadvantage of matrix-free methods is that
the preconditioners that are most commonly used to improve the
convergence properties and robustness of Krylov solvers do involve
manipulations of $A$; hence these cannot be applied in a matrix-free
approach. For the purpose of assembly, a matrix-free approach replaces
the assembly of the matrix $A$ with repeated assembly of a vector
$Av$, which is the action of $A$ on the given vector~$v$. A key
element in the efficient application of such methods is the rapid
assembly of vectors. The cost of insertion into a dense vectors is
relatively low, compared to insertion into a sparse matrix. The
computation of the cell tensor is therefore the dominant
cost. Assembly of the action of a linear or linearized operator is
supported in FEniCS.
