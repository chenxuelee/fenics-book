\fenicschapter{Saddle Point Stability}
              {Saddle Point Stability}
              {Marie E. Rognes}
              {rognes}

The stability of finite element approximations for abstract saddle point problems
has been an active research field for the last four decades. The well-known Babuska-
Brezzi conditions provide stability for saddle point problems of the form
\begin{equation}
  a(v, u) + b(v, p) + b(u, q) = \langle f, v \rangle + \langle g, q \rangle
  \quad \forall (v, q) \in V \times Q,
\end{equation}
where $a$, $b$ are bilinear forms and $V$, $Q$ Hilbert spaces. For a
choice of discrete spaces $V_h \subset V$ and $Q_h \subset Q$, the
corresponding discrete Babuska-Brezzi conditions guarantee stability.

However, there are finite element spaces used in practice, with
success1, that do not satisfy the stability conditions in general. The
element spaces may satisfy the conditions of certain classes of
meshes. Or, there are only a few spurious modes to be filtered out
before the method is stable.

The task of determining the stability of a given set of finite
element spaces for a given set of equations has mainly been a manual
task. However, the flexibility of the FEniCS components has made
automation feasible.

For each set of discrete spaces, the discrete Brezzi conditions can be
equivalently formulated in terms of an eigenvalue problem. For
instance, [...]  where $B$ is the element matrix associated with the
form $b$ and $M$, $N$ are the matrices induced by the inner-products
on $V$ and $Q$ respectively. Hence, the stability of a set of finite
element spaces on a type of meshes can be tested numerically by
solving a series of eigenvalue problems.

A small library FEAST (Finite Element Automatic Stability Tester) has
been built on top of the core FEniCS components, providing automated
functionality for the testing of finite element spaces for a given
equation on given meshes. With some additional input, convergence
rates and in particular optimal choices of element (in some measure
such as error per degrees of freedom) can be determined.

In this note, the functionality provided by FEAST is explained and
results for equations such as the Stokes equations, Darcy flow and
mixed elasticity are demon- strated.
