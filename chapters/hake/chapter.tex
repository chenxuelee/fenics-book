\fenicschapter{Simulations of a Hybrid Stochastic and Deterministic Model of Ca2+ Dynamics in the Dyadic Cleft}
              {Simulations of a Hybrid Stochastic and Deterministic Model of Ca2+ Dynamics}
              {Johan Hake}

\editornote{[hake]}

In this study we present a coupled stochastic and deterministic model of Ca2+
dynamics in the dyadic cleft and a simulator that solves the model. The simulator
is implemented in python using NumPy, PyDOLFIN and FFC.

Ca2+ is one of the most important cellular messengers we have in our body.
Among other things, Ca2+ controls the contraction of cells in the heart. The elec-
tric signal that traverse the heart during a beat, triggers Ca2+ inow, which then
triggers further Ca2+ release from intracellular Ca2+ storage, through a process
called Ca2+ induced Ca2+ release. The elevated Ca2+ then makes the heart con-
tract. The Ca2+ induced Ca2+ release is tightly controlled in a signalling micro
domain called the dyadic cleft. In this small domain Ca2+ channels are opened and
closed stochastically. We have used Markov chains to model this process, and we
present a modied Gillespie method that is used to solve these models. Inside the
domain Ca2+ ions diuse triggering further Ca2+ release. The diusion is inuenced
by an electric eld, which is induced by charges that reside on the cell membrane.
The eld acts as an eective buer slowing the diusion. This electro-diusion is
modelled using a partial dierential equation (PDE), which is discretized using the
nite element method, and stabilized using the Streamline upwind Petrov-Galerkin
method. We present the discretization and stabilization method and how we have
used FEniCS to implement them. The stochastic Markov chain model and the
deterministic PDE is coupled as some of the rates in the Markov chain models
are Ca2+ dependent and the Ca2+ diusion is dependent on a channel being open
or close. We present a method which solves this coupled system using a hybrid
approach.
