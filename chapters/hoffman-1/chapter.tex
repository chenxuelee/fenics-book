\fenicschapter{Turbulent Flow and Fluid--Structure Interaction with Unicorn}
              {Turbulent Flow and Fluid--Structure Interaction with Unicorn}
              {Johan Hoffman, Johan Jansson, Niclas Jansson, Claes Johnson and Murtazo Nazarov}

\section{Introduction}

For many problems involving a fluid and a structure, decoupling the
computation of the two is not possible for accurate modeling of the
phenomenon at hand, instead the full fluid-structure interaction (FSI)
problem has to be solved together as a coupled problem. This includes
a multitude of important problems in biology, medicine and industry,
such as the simulation of insect or bird flight, the human
cardiovascular and respiratory systems, the human speech organ, the
paper making process, acoustic noise generation in exhaust systems,
airplane wing flutter, wind induced vibrations in bridges and wave
loads on offshore structures. Common for many of these problems is
that for various reasons they are very hard or impossible to
investigate experimentally, and thus reliable computational simulation
would open up for detailed study and new insights, as well as for new
important design tools for construction.

Computational methods used today are characterized by a high
computational cost, and a lack of generality and reliability. In
particular, major open challenges of computational FSI include: (i)
robustness of the fluid- structure coupling, (ii) for high Reynolds
numbers the computation of turbulent fluid flow, and (iii) efficiency
and reliability of the computations in the form of adaptive methods
and quantitative error estimation.

The FEniCS project aims towards the goals of generality, efficiency,
and simplicity, concerning mathematical methodology, implementation,
and application.  The Unicorn project is a realization of this effort
in the field of continuum mechanics, that we here expose to a range of
challenging problems that traditionally demand a number of specialized
methods and codes, including some problems which have been considered
unsolvable with state of the art methods.  The basis of Unicorn is an
adaptive finite element method and a unified continuum formulation,
which offer new possibilities for mathematical modeling of high
Reynolds number turbulent flow, gas dynamics and fluid-structure
interaction.

Unicorn, which is based on the DOLFIN/FFC/FIAT suite, is today central
in a number of applied research projects, characterized by large
problems, complex geometry and constitutive models, and a need for
fast results with quantitative error control.  We here present some
key elements of Unicorn and the underlying theory, and illustrate how
this opens for a number of breakthroughs in applied research.

\section{Continuum models}

Continuum mechanics is based on conservation laws for mass, momentum
and energy, together with constitutive laws for the stresses. A
Newtonian fluid is characterized by a linear relation between the
viscous stress and the strain, together with a fluid pressure,
resulting in the Navier-Stokes equations.  Many common fluids,
including water and air at subsonic velocities, can be modeled as
incompressible fluids, where the pressure acts as a Langrangian
multiplier enforcing a divergence free velocity. In models of gas
dynamics the pressure is given from the internal energy, with an ideal
gas corresponding to a linear relation.  Solids and non-Newtonian
fluids can be described by arbitrary complex laws relating the stress
to displacements, strains and internal energies.

Newtonian fluids are characterized by two important non-dimensional
numbers: the Reynolds number $Re$, measuring the importance of viscous
effects, and the Mach number $M$, measuring the compressibility
effects by relating the fluid velocity to the speed of sound. High
$Re$ flow is characterized by partly turbulent flow, and high $M$ flow
by shocks and contact discontinuities, all phenomena associated with
complex flow on a range of scales. The Euler equations corresponds to
the limit of inviscid flow where $Re \rightarrow \infty$, and
incompressible flow corresponds to the limit of $M\rightarrow 0$.

\section{Mathematical framework}

The mathematical theory for the Navier-Stokes equations is incomplete,
without any proof of existence or uniqueness, formulated as one of the
Clay Institute \$1 million problems. What is available if the proof of
existence of weak solutions by Leray from 1934, but this proof does
not extend to the inviscid case of the Euler equations. No general
uniqueness result is available for weal solutions, which limits the
usefulness of the concept.

In \cite{} a new mathematical framework of weak (output) uniqueness is
introduced, characterizing well-posedness of weak solutions with
respect to functionals $M$ of the solution $u$, thereby circumventing
the problem of non-existence of classical solutions. This framework
extends to the Euler equations, and also to compressible flow, where
the basic result takes the form
\begin{equation}
\vert M(u) - M(U) \vert \leq S (\Vert R(u)\Vert_{-1} + \Vert R(U)\Vert_{-1}) 
\end{equation}
with $\Vert \cdot \Vert _{-1}$ a weak norm measuring residuals $R$ of two weak solutions $u$ and 
$U$, and with $S$ a stability factor given by a duality argument connecting local errors to output errors 
in $M$. 

\section{Computational method}

Computational methods in fluid mechanics are typically very
specialized; for a certain range of $Re$ or $M$, or for a particular
class of geometries. In particular, there is a wide range of
turbulence models and shock capturing techniques.  The goal of Unicorn
is to design one method with one implementation, capable of modeling
general geometry and the whole range of parameters $Re$ and $M$. The
mathematical framework of well-posedness allows for a general
foundation for Newtonian fluid mechanics, and the General Galerkin
(G2) finite element method offers a robust algorithm to compute weak
solutions \cite{g2incomprcompr}.

Adaptive G2 methods are based on a posteriori error estimates of the form: 
\begin{equation}
\vert M(u) - M(U) \vert \leq \sum_K {\cal E}_K
\end{equation}
with $ {\cal E}_K$ a local error indicator for cell $K$.

Parallel mesh refinement...

[Unicorn implementation]

\section{Boundary conditions}

1/2 page

friction bc, turbulent boundary conditions

[Unicorn implementation]

\section{Geometry modeling}

1/2 page

projection to exact geometry

[Unicorn implementation]

\section{Fluid-structure interaction}

1 page

Challenges: stablity coupling, weak strong, monolithic

different software, methods

mesh algorithms, smoothing, Madlib

Unified continuum fluid-structure interaction

[Unicorn implementation]

\section{Applications}

\subsection{Turbulent flow separation}

1 pages

drag crisis, cylinders and spheres, etc. 

\subsection{Flight aerodynamics}

2 page 

naca aoa

\subsection{Vehicle aerodynamics} 

1 page 

Volvo

\subsection{Biomedical flow}

1 page 

ALE heart model 

\subsection{Aeroacoustics}

1 page 

Swenox mixer - aerodynamic sources 

\subsection{Gas flow}

2 pages 

compressible flow 

\section{References}

1 page
