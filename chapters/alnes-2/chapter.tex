\fenicschapter{UFC: A Low-Level Interface at the Core of FEniCS}
              {UFC: A Low-Level Interface at the Core of FEniCS}
              {Martin S. Aln\ae{}s, Anders Logg, and Kent-Andre Mardal}

\editornote{[alnes-2]}

When combining handwritten libraries with automatically generated
code like we do in FEniCS, it is important to have clear boundaries
between the two.  This is best done by having the generated code
implement a fixed interface, such that the library and generated code
can be as independent as possible.  Such an interface is specified in
the project Unified Form-assembly Code (UFC) for finite elements and
discrete variational forms. This interface consists of a small set of
abstract classes in a single header file, which is well documented.
The details of the UFC interface should rarely be visible to the
end-user, but can be important for developers and technical users to
understand how FEniCS projects fit together. In this chapter we
discuss the main design ideas behind the UFC interface, including
current limitations and possible future improvements.
