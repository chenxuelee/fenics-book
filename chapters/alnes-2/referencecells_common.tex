\index{reference cells}

The following five reference cells are covered by the UFC
specification: the reference \emph{interval}, the reference
\emph{triangle}, the reference \emph{quadrilateral}, the reference
\emph{tetrahedron}, and the reference \emph{hexahedron}.  The UFC
specification assumes that each cell in a finite element mesh is
always isomorphic to one of the reference cells. The UFC reference
cells are listed in the table below.

\vspace{0.5cm}
\begin{center}
  \begin{tabular}{lccc}
    \toprule
    Reference cell & Dimension & \#Vertices & \#Facets \\
    \hline
    The reference interval      & 1 & 2 & 2 \\
    The reference triangle      & 2 & 3 & 3 \\
    The reference quadrilateral & 2 & 4 & 4 \\
    The reference tetrahedron   & 3 & 4 & 4 \\
    The reference hexahedron    & 3 & 8 & 6 \\
    \bottomrule
  \end{tabular}
\end{center}

\paragraph{The reference interval.}
\index{interval}

The reference interval and the coordinates of its two vertices are
shown in the figure and table below.

\newcolumntype{T}{>{\centering\arraybackslash}m{6cm}}

\begin{center}
  \begin{tabular}{TT}
    \fenicsfig{alnes-2}{interval}{\smallfig}
    &
    \begin{tabular}{cc}
      \toprule
      Vertex & Coordinates \\
      \hline
      $v_0$ & $x = 0$ \\
      $v_1$ & $x = 1$ \\
      \bottomrule
    \end{tabular}
  \end{tabular}
\end{center}

\paragraph{The reference triangle.}
\index{triangle}

The reference triangle and the coordinates of its three vertices are
shown in figure and table below.

\begin{center}
  \begin{tabular}{TT}
    \fenicsfig{alnes-2}{triangle}{\smallfig}
    &
    \begin{tabular}{cc}
      \toprule
      Vertex & Coordinates \\
      \hline
      $v_0$ & $x = (0, 0)$ \\
      $v_1$ & $x = (1, 0)$ \\
      $v_2$ & $x = (0, 1)$ \\
      \bottomrule
    \end{tabular}
  \end{tabular}
\end{center}

\paragraph{The reference quadrilateral.}
\index{quadrilateral}

The reference quadrilateral and the coordinates of its four vertices
are shown in the figure and table below.

\begin{center}
  \begin{tabular}{TT}
    \fenicsfig{alnes-2}{quadrilateral}{\smallfig}
    &
    \begin{tabular}{cc}
      \toprule
      Vertex & Coordinates \\
      \hline
      $v_0$ & $x = (0, 0)$ \\
      $v_1$ & $x = (1, 0)$ \\
      $v_2$ & $x = (1, 1)$ \\
      $v_3$ & $x = (0, 1)$ \\
      \bottomrule
    \end{tabular}
  \end{tabular}
\end{center}

\paragraph{The reference tetrahedron.}
\index{tetrahedron}

The reference tetrahedron and the coordinates of its four vertices are
shown in the figure and table below.

\begin{center}
  \begin{tabular}{TT}
    \fenicsfig{alnes-2}{tetrahedron}{\smallfig}
    &
    \begin{tabular}{cc}
      \toprule
      Vertex & Coordinates \\
      \hline
      $v_0$ & $x = (0, 0, 0)$ \\
      $v_1$ & $x = (1, 0, 0)$ \\
      $v_2$ & $x = (0, 1, 0)$ \\
      $v_3$ & $x = (0, 0, 1)$ \\
      \bottomrule
    \end{tabular}
  \end{tabular}
\end{center}

\paragraph{The reference hexahedron.}
\index{hexahedron}

The reference hexahedron and the coordinates of its eight vertices are
shown in the figure and table below.

\begin{center}
  \begin{tabular}{TT}
    \fenicsfig{alnes-2}{hexahedron}{\smallfig}
    &
    \begin{tabular}{cc}
      \toprule
      Vertex & Coordinate \\
      \hline
      $v_0$ & $x = (0, 0, 0)$ \\
      $v_1$ & $x = (1, 0, 0)$ \\
      $v_2$ & $x = (1, 1, 0)$ \\
      $v_3$ & $x = (0, 1, 0)$ \\
      $v_4$ & $x = (0, 0, 1)$ \\
      $v_5$ & $x = (1, 0, 1)$ \\
      $v_6$ & $x = (1, 1, 1)$ \\
      $v_7$ & $x = (0, 1, 1)$ \\
      \bottomrule
    \end{tabular}
  \end{tabular}
\end{center}
