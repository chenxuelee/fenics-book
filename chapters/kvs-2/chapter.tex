\fenicschapter{Simulating the Hemodynamics of the Circle of Willis}
              {Simulating the Hemodynamics of the Circle of Willis}
              {Kristian Valen-Sendstad, Kent-Andre Mardal and Anders Logg}
              {kvs-2}

Stroke is a leading cause of death in the western world. Stroke has
different causes but around 5-10\% is the result of a so-called
subarachnoid hemorrhage caused by the rupture of an aneurysm. These
aneurysms are usually found in our near the circle of Willis, which is
an arterial network at the base of the brain.  In this chapter we will
employ FEniCS solvers to simulate the hemodynamics in several examples
ranging from simple time-dependent flow in pipes to the blood flow in
patient-specific anatomies.
