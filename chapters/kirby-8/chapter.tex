\fenicschapter{Tensor representation of finite element variational forms}
              {Tensor representation of finite element variational forms}
              {Robert C. Kirby and Anders Logg}
              {kirby-8}

In Chapter~\ref{chap:logg-3}, we saw that an important step in the
assembly of matrices and vectors for the discretization of finite
element variational problems is the evaluation of the cell (element)
tensor $A_T$ defined by
\begin{equation}
  A_{T,i}
  = a_T(\phi^{T,\rho}_{i_{\rho}}, \ldots, \phi^{T,2}_{i_2}, \phi^{T,1}_{i_1}).
\end{equation}
Here, $a_T$ is the local contribution to a multilinear form $a:
V_{\rho} \times \cdots \times V_2 \times V_1$,
$i=(i_1,i_2,\dots,i_\rho)$ is a multi-index of length~$\rho$, and
$\{\phi^{T,j}_k\}_{k=1}^{n_j}$ is a basis for the local finite element
space of $V_{j,h} \subset V_j$ on a local cell~$T$ for $j =
1,2,\ldots,\rho$. In this chapter, we describe how the cell
tensor~$A_T$ can be computed efficiently by an approach referred to as
\emph{tensor representation}.

%------------------------------------------------------------------------------
\section{Tensor representation for Poisson equation}

We first describe how one may express the cell tensor for Poisson's
equation as a special tensor contraction and explain below how this
may be generalized to other variational forms. For Poisson's equation,
the cell tensor (matrix) $A_T$ is defined by
\begin{equation}
  A_{T,i} =
  \int_T
  \nabla \phi_{i_1}^{T,1} \cdot
  \nabla \phi_{i_2}^{T,2} \dx
  =
  \int_T
  \sum_{\beta=1}^d
  \frac{\partial \phi_{i_1}^{T,1}}{\partial x_{\beta}}
  \frac{\partial \phi_{i_2}^{T,2}}{\partial x_{\beta}} \dx.
\end{equation}
Let $F_T : \hat{T} \rightarrow T$ be an affine map from a
reference cell~$\hat{T}$ to the current cell~$T$ as illustrated in
Figure~\ref{fig:affinemap}. Using this affine map, we make a change of
variables to obtain
\begin{equation}
  A_{T,i} =
  \int_{\hat{T}}
  \sum_{\beta=1}^d
  \sum_{\alpha_1=1}^d
  \frac{\partial \hat{x}_{\alpha_1}}{\partial x_{\beta}}
  \frac{\partial \hat{\phi}^1_{i_1}}{\partial \hat{x}_{\alpha_1}}
  \sum_{\alpha_2=1}^d
  \frac{\partial \hat{x}_{\alpha_2}}{\partial x_{\beta}}
  \frac{\partial \hat{\phi}^2_{i_2}}{\partial \hat{x}_{\alpha_2}}
  \det F_T'
  \dxhat.
\end{equation}
Here, $\hat{\phi}_i^j = \phi_i^{T,j} \circ F_T$ denotes the basis function
on the reference cell~$\hat{T}$ corresponding to the basis function
$\phi_i^{T,j}$ on the current cell~$T$. Since $F_T$ is affine, the
derivatives $\partial \hat{x} / \partial x$ and the determinant~$\det F_T'$
are constant. We thus obtain
\begin{equation}
  A_{T,i} =
  \det F_T'
  \sum_{\alpha_1=1}^d
  \sum_{\alpha_2=1}^d
  \sum_{\beta=1}^d
  \frac{\partial \hat{x}_{\alpha_1}}{\partial x_{\beta}}
  \frac{\partial \hat{x}_{\alpha_2}}{\partial x_{\beta}}
  \int_{\hat{T}}
  \frac{\partial \hat{\phi}^1_{i_1}}{\partial \hat{x}_{\alpha_1}}
  \frac{\partial \hat{\phi}^2_{i_2}}{\partial \hat{x}_{\alpha_2}}
  \dxhat
  =
  \sum_{\alpha_1=1}^d
  \sum_{\alpha_2=1}^d
  A^0_{i\alpha} G_T^{\alpha},
\end{equation}
where
\begin{equation} \label{eq:poisson,AandG}
  \begin{split}
    A^0_{i\alpha}
    &=
    \int_{\hat{T}}
    \frac{\partial \hat{\phi}^1_{i_1}}{\partial \hat{x}_{\alpha_1}}
    \frac{\partial \hat{\phi}^2_{i_2}}{\partial \hat{x}_{\alpha_2}}
    \dxhat, \\
    G_T^{\alpha}
    &=
    \det F_T'
    \sum_{\beta=1}^d
    \frac{\partial \hat{x}_{\alpha_1}}{\partial x_{\beta}}
    \frac{\partial \hat{x}_{\alpha_2}}{\partial x_{\beta}}.
  \end{split}
\end{equation}
We refer to the tensor~$A^0$ as the \emph{reference tensor} and to the
tensor~$G_T$ as the \emph{geometry tensor}. We may thus express the
computation of the cell tensor~$A_T$ for Poisson's equation as the
tensor contraction
\begin{equation}
  A_T = A^0 : G_T.
\end{equation}

\begin{figure}
  \begin{center}
    \def\svgwidth{\largefig}
    \import{chapters/kirby-8/pdf/}{affinemap.pdf_tex}
    \caption{The (affine) map $F_T$ from a reference cell $\hat{T}$
      to a cell $T \in \mathcal{T}_h$.}
    \label{fig:affinemap}
  \end{center}
\end{figure}

This tensor contraction may be computed efficiently by precomputing
the entries of the reference tensor~$A^0$. This is possible since the
reference tensor is constant and does not depend on the cell~$T$ or
the mesh~$\mathcal{T}_h = \{T\}$. On each cell~$T \in \mathcal{T}_h$,
the cell tensor may thus be computed by first computing the geometry
tensor~$G_T$ and then contracting it with the precomputed reference
tensor. In Chapter~\ref{chap:logg-1}, we describe the FEniCS Form
Compiler~(FFC) which precomputes the reference tensor~$A^0$ at
compile-time and generates code for computing the tensor contraction.

For Poisson's equation in two space dimensions, the tensor contraction
involves contracting the $2 \times 2$ geometry tensor~$G_T$ with each
corresponding block of the $3 \times 3 \times 2 \times 2$ reference
tensor~$A^0$ to form the entries of the $3 \times 3$ cell tensor
$A^T$. Each of these entries may thus be computed in only four
multiply-add pairs (plus the cost of computing the geometry
tensor). This brings a considerable speedup compared to evaluation by
run-time quadrature, in particular for higher-order elements. In
Chapter~\ref{chap:kirby-4}, we discuss how this may be improved
further by examining the structure of the reference tensor~$A^0$ to
find a reduced-arithmetic computation for the tensor contraction.

%------------------------------------------------------------------------------
\section{A representation theorem}

In~\citet{KirbyLogg2006}, it was proved that the cell tensor for any
affinely mapped monomial multilinear form may be expressed as a tensor
contraction $A_T = A^0 : G_T$; that is,
\begin{equation}
  A_{T,i} = \sum_{\alpha} A^0_{i\alpha} G_T^{\alpha}.
\end{equation}
More precisely, the cell tensor may be expressed as a sum of tensor
contractions:
\begin{equation} \label{eq:tensorcontraction}
  A_T = \sum_k A^{0,k} : G_{T,k}.
\end{equation}
By a monomial multilinear form, we here mean a multilinear form that
can be expressed as a sum of monomials, where each monomial is a
product of coefficients, trial/test functions and their derivatives.
This class covers all forms that may be expressed by addition,
multiplication and differentiation. Early versions of the form
compiler FFC implemented a simple form language that was limited to
these three operations. This simple form language is now replaced by
the new and more expressive UFL form language

The representation theorem was later extended to Piola-mapped elements
in~\citet{RognesKirbyLogg2009}, and in~\citet{OelgaardLoggWells2008}
it was demonstrated how the tensor representation may be computed for
discontinuous Galerkin methods.

The ranks of the reference and geometry tensors are determined by the
multilinear form~$a$, in particular by the number of coefficients and
derivatives of the form. Since the rank of the cell tensor $A_T$ is
equal to the arity~$\rho$ of the multilinear form~$a$, the rank of the
reference tensor~$A^0$ must be $|i\alpha| = \rho + |\alpha|$, where
$|\alpha|$ is the rank of the geometry tensor. For Poisson's equation,
we have $|i\alpha| = 4$ and $|\alpha| = 2$. In Tables~\ref{tab:mass}
and~\ref{tab:advection}, we demonstrate how the tensor representation
may be computed for the bilinear forms $a(u, v) = \inner{u}{v}$ (mass
matrix) and $a(w; u, v) = \inner{w \cdot \nabla u}{v}$ (advection).

\begin{table}
  \begin{center}
    \begin{tabular}{|rcl|c|}
      \hline
      $a(u, v)$ &$=$& $\inner{u}{v}$ & rank \\
      \hline
      \hline
      &&&\\[-2ex]
      $A^0_{i\alpha}$ &$=$& $\int_{\hat{T}} \hat{\phi}_{i_1}^{1} \hat{\phi}_{i_2}^{2} \dxhat$
      & $|i\alpha| = 2$ \\[1ex]
      \hline
      &&&\\[-2ex]
      $G_T^{\alpha}$ &$=$& $\det F_T'$
      & $|\alpha| = 0$ \\[1ex]
      \hline
    \end{tabular}
    \caption{Tensor representation $A_T = A^0 : G_T$ of the cell
      tensor~$A_T$ for the bilinear form associated with a mass
      matrix.}
    \label{tab:mass}
  \end{center}
\end{table}

% AL: Note to editor: don't touch the [!ht] below! It works together
% with the vspace{1cm} below to make sure the caption does not run
% into the text.
\begin{table}[!ht]
  \begin{center}
    \begin{tabular}{|rcl|c|}
      \hline
      $a(w; u, v)$ &$=$& $\inner{w \cdot \nabla u}{v}$ & rank \\
      \hline
      \hline
      &&&\\[-2ex]
      $A^0_{i\alpha}$ &$=$&
      $\sum_{\beta=1}^d
      \int_{\hat{T}}
      \frac{\partial \hat{\phi}^2_{i_2}[\beta]}{\partial \hat{x}_{\alpha_3}}
      \hat{\phi}^3_{\alpha_1}[\alpha_2]
      \hat{\phi}^1_{i_1}[\beta]
      \dxhat$
      & $|i\alpha| = 5$ \\[1ex]
      \hline
      &&&\\[-2ex]
      $G_T^{\alpha}$ &$=$&
      $w^T_{\alpha_1} \det F_T'
      \frac{\partial \hat{x}_{\alpha_3}}{\partial x_{\alpha_2}}$
      & $|\alpha| = 3$ \\[1ex]
      \hline
    \end{tabular}
    \caption{Tensor representation $A_T = A^0 : G_T$ of the cell
      tensor~$A_T$ for the bilinear form associated with advection $w
      \cdot \nabla u$. It is assumed that the velocity field $w$ may
      be interpolated into a local finite element space with expansion
      coefficients~$w^T_{\alpha_1}$.  Note that $w$ is a vector-valued
      function, the components of which are referenced by $w[\beta]$.}
    \label{tab:advection}
  \end{center}
\end{table}
\vspace{1cm}

%------------------------------------------------------------------------------
\section{Extensions and limitations}

The tensor contraction~\eqref{eq:tensorcontraction} assumes that the
map~$F_T$ from the reference cell is affine, allowing the transforms
$\partial \hat{x} / \partial x$ and the determinant $\det F_K'$ to be pulled
out of the integral. If the map is non-affine (sometimes called a
``higher-order'' map), one may expand it in the basis functions of the
corresponding finite element space and pull the coefficients outside
the integral, as done for the advection term from
Table~\ref{tab:advection}. Alternatively, one may evaluate the cell
tensor by quadrature and express the summation over quadrature points
as a tensor contraction as explained in~\citet{KirbyLogg2006}.  As
noted above, the tensor contraction readily extends to basis functions
mapped by Piola transforms.

One limitation of this approach is it requires each basis function on
a cell \( T \) to be the image of a single reference element basis
function under an affine Piola transformation. While this covers a
wide range of commonly used elements, it does not include certain
kinds of elements with derivative-based degrees of freedom such as the
Hermite and Argyris elements. Let \( \mathcal{F}_T \) be the mapping
of the reference element function space to the function space over the
cell $T$, such as the affine map or Piola transform. Then the physical
element basis functions can be expressed as a linear combination of
the transformed reference element basis functions,
\begin{equation}
  \phi^T_i = \sum_{j=1}^n M_{T,{ij}} \mathcal{F}_T \left( \hat{\phi}_j \right).
\end{equation}
The structure of this matrix \( M_T \) depends on the kinds of degrees
freedom, and the values typically vary for each \( T \) based on the
cell geometry.  Frequently, the matrix $M_T$ is sparse. Given \( M_T
\), the tensor-contraction framework may be extended to handle these
more general elements.  As before, one may compute the reference
tensor \( A^0 \) by mapping the reference element basis functions. But
in addition, the tensor contraction \( A^0 : G_T \) must be corrected
by acting on it with the matrix \( M_T \). This is currently not
implemented in the form compiler FFC and thus FEniCS does not support
Hermite and Argyris elements.

For many simple variational forms, such as those for Poisson's
equation, the mass matrix and the advection term discussed above, the
tensor contraction~\eqref{eq:tensorcontraction} leads to significant
speedups over numerical quadrature, sometimes as much as several
orders of magnitude.  However, as the complexity of a form increases,
the relative efficiency of quadrature also increases. In simple terms,
the complexity of a form can be measured as the number of derivatives
and the number of coefficients appearing in a form. For each
derivative and coefficient, the rank of the reference tensor $A^0$
increases by one. Thus, for Poisson's equation, the rank is $2 + 2 =
4$ since the form has two derivatives and for the mass matrix, the
rank is $2 + 0$ since there are neither derivatives nor
coefficients. For the advection term, the rank is $2 + 2 + 1 = 5$
since the form has one derivative, one coefficient, and also an inner
product $w \cdot \nabla$. Since the size of the reference tensor~$A^0$
grows exponentially with its rank, the tensor contraction may become
very costly for forms of high complexity. In these cases, quadrature
is more efficient. Quadrature may sometimes also be the only available
option as the tensor contraction is not directly applicable to forms
that are not expressed as simple sums of products of coefficients,
trial/test functions and their derivatives. For this reason, it is
important to be able to choose between both approaches; tensor
representation may sometimes be the most efficient approach whereas in
other cases quadrature is more efficient or even the only possible
alternative.  Such trade-offs are discussed in
Chapter~\ref{chap:oelgaard-2} and Chapter~\ref{chap:kirby-3}.
