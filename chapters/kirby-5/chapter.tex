\fenicschapter{Finite element variational forms}
              {Finite element variational forms}
              {Robert C. Kirby and Anders Logg}
              {kirby-5}

Much of the FEniCS software is devoted to the formulation of
variational forms (UFL), the discretization of variational forms
(FIAT, FFC, SyFi) and the assembly of the corresponding discrete
operators (UFC, DOLFIN). This chapter summarizes the notation for
variational forms used throughout FEniCS.

%------------------------------------------------------------------------------
\section{Background}

In Chapter~\ref{chap:kirby-7}, we introduced the following canonical
variational problem: Find $u \in V$ such that
\begin{equation} \label{eq:kirby-5:canonical}
  a(u, v) = L(v) \quad \foralls v \in \hat{V},
\end{equation}
where $V$ is a given trial space and $\hat{V}$ is a given test space.
The bilinear form
\begin{equation}
  a: V \times \hat{V} \rightarrow \R
\end{equation}
maps a pair of trial and test functions to a real number and is linear
in both arguments. Similarly, the linear form $L: \hat{V} \rightarrow
\R$ maps a given test function to a real number. We also considered
the discretization of nonlinear variational problems: Find $u \in V$
such that
\begin{equation}
  F(u; v) = 0 \quad \foralls v \in \hat{V}.
\end{equation}
Here, $F: V \times \hat{V} \rightarrow \R$ again maps a pair of
functions to a real number. The semilinear form $F$ is nonlinear in
the function~$u$ but linear in the test function~$v$. Alternatively,
we may consider the mapping
\begin{equation}
  L_u \equiv F(u; \cdot) : \hat{V} \rightarrow \R,
\end{equation}
and note that $L_u$ is a linear form on~$\hat{V}$ for any fixed value
of~$u$. In Chapter~\ref{chap:kirby-7}, we also considered the
estimation of the error in a given functional $\mathcal{M} : V
\rightarrow \R$. Here, the possibly nonlinear functional $\mathcal{M}$
maps a given function~$u$ to a real number $\mathcal{M}(u)$.

In all these examples, the central concept is that of a form that maps
a given tuple of functions to a real number. We shall refer to these
as \emph{multilinear forms}. Below, we formalize the concept of a
multilinear form, discuss the discretization of multilinear forms, and
related concepts such as the \emph{action}, \emph{derivative} and
\emph{adjoint} of a multilinear form.

%------------------------------------------------------------------------------
\section{Multilinear forms}

A form is a mapping from the product of a given sequence
$\{V_j\}_{j=1}^{\rho}$ of function spaces to a real number,
\begin{equation}
  a : V_{\rho} \times \cdots \times V_2 \times V_1 \rightarrow \R.
\end{equation}
If the form~$a$ is linear in each of its arguments, we say that the
form is \emph{multilinear}. The number of arguments~$\rho$ of the form
is the \emph{arity} of the form. Note that the spaces are numbered
from right to left. As we shall see below in
Section~\ref{sec:kirby-5:discretizing}, this is practical when we
consider the discretization of multilinear forms.

Forms may often be parametrized over one or more
\emph{coefficients}. A typical example is the right-hand side $L$ of
the canonical variational problem~(\ref{eq:kirby-5:canonical}), which
is a linear form parametrized over a given coefficient~$f$. We shall
use the notation $a(f; v) \equiv L_f(v) \equiv L(v)$ and refer to the
test function~$v$ as an \emph{argument} and to the function~$f$ as a
\emph{coefficient}. In general, we shall refer to forms which are
linear in each argument (but possibly nonlinear in its coefficients)
as multilinear forms. Such a multilinear form is a mapping from the
product of a sequence of argument spaces and coefficient spaces:
\begin{equation}
  \begin{split}
    a &: W_1 \times W_2 \times \cdots \times W_n \, \times \,
         V_{\rho} \times \cdots \times V_2 \times V_1 \rightarrow \R, \\
    a &\mapsto a(w_1, w_2, \ldots, w_n; v_{\rho}, \ldots, v_2, v_1);
  \end{split}
\end{equation}
The argument spaces $\{V_j\}_{j=1}^{\rho}$ and coefficient spaces
$\{W_j\}_{j=1}^n$ may all be the same space but they typically differ,
such as when Dirichlet boundary conditions are imposed on one or more of
the spaces, or when the multilinear form arises from the discretization
of a mixed problem such as in Section~\ref{sec:kirby-7:mixed}.

In finite element applications, the arity of a form is typically $\rho
= 2$, in which case the form is said to be bilinear, or $\rho = 1$, in
which case the form is said to be linear. In the special case of $\rho
= 0$, we shall refer to the multilinear form as a
\emph{functional}. It may sometimes also be of interest to consider
forms of higher arity ($\rho > 2$). Below, we give examples of some
multilinear forms of different arity.

\subsection{Examples}

\paragraph{Poisson's equation.}

Consider Poisson's equation with variable conductivity~$\kappa =
\kappa(x)$,
\begin{equation}
  -\nabla \cdot (\kappa \nabla u) = f.
\end{equation}
Assuming Dirichlet boundary conditions on the boundary
$\partial\Omega$, the corresponding canonical
variational problem is defined in terms of a pair of multilinear
forms, $a(\kappa; u, v) = \int_{\Omega} \kappa \nabla u \cdot \nabla
v \dx$ and $L(v) = \int_{\Omega} f v \dx$. Here, $a$ is a bilinear
form ($\rho = 2$) and $L$ is a linear form ($\rho = 1$). Both forms
have one coefficient ($n = 1$) and the coefficients are $\kappa$ and
$f$ respectively:
\begin{equation}
  \begin{split}
    a &= a(\kappa; u, v), \\
    L &= L(f; v).
  \end{split}
\end{equation}
We usually drop the coefficients from the notation and use the
short-hand notation $a = a(u, v)$ and $L = L(v)$.

\paragraph{The incompressible Navier--Stokes equations.}

The incompressible Navier--Stokes equations for the velocity~$u$ and
pressure~$p$ of an incompressible fluid read:
\begin{equation}
  \begin{split}
    \rho(\dot{u} + \nabla u \, u) - \nabla \cdot \sigma(u, p) &= f, \\
    \nabla \cdot u &= 0,
  \end{split}
\end{equation}
where the stress tensor~$\sigma$ is given by $\sigma(u, p) = 2 \mu
\epsilon(u) - p I$, $\epsilon$ is the symmetric gradient; that is,
$\epsilon(u) = \frac{1}{2}(\nabla u + (\nabla u)^{\top})$, $\rho$ is
the fluid density and $f$ is a body force.
Consider here the form obtained by
integrating the nonlinear term $\nabla u , u$ against a test
function $v$:
\begin{equation}
  a(u; v) = \int_{\Omega} \nabla u \, u \cdot v \dx.
\end{equation}
This is a linear form ($\rho = 1$) with one coefficient ($n = 1$). We
may linearize around a fixed velocity~$\bar{u}$ to obtain
\begin{equation}
  a(u; v) = a(\bar{u}; v) + a'(\bar{u}; v) \delta u + \mathcal{O}(\delta u^2),
\end{equation}
where $u = \bar{u} + \delta u$. The linearized operator $a'$ is here
given by
\begin{equation}
  a'(\bar{u}; \delta u, v) \equiv a'(v; \bar{u}) \delta u =
  \int_{\Omega}
  \nabla \delta u \cdot \bar{u} \cdot v +
  \nabla \bar{u} \cdot \delta u \cdot v \dx.
\end{equation}
This is a bilinear form ($\rho = 2$) with one coefficient ($n = 1)$.
We may also consider the \emph{trilinear} form
\begin{equation}
  a(w, u, v) = \int_{\Omega} \nabla u , w \cdot v \dx.
\end{equation}
This trilinear form may be assembled into a rank three tensor and
applied to a given vector of expansion coefficients for $w$ to obtain
a rank two tensor (a matrix) corresponding to the bilinear form $a(w;
u, v)$. This may be useful in an iterative fixed point method for the
solution of the Navier--Stokes equations, in which case $w$ is a given
(frozen) value for the convective velocity obtained from a previous
iteration. This is rarely done in practice due to the cost of
assembling the global rank three tensor. However, the corresponding
local rank three tensor may be contracted with the local expansion
coefficients for~$w$ on each local cell to compute the matrix
corresponding to~$a(w; u, v)$.

\paragraph{Lift and drag.}

When solving the Navier--Stokes equations, it may be of interest to
compute the lift and drag of some object immersed in the fluid. The
lift and drag are given by the $z$- and $x$-components of the force
generated on the object (for a flow in the $x$-direction):
\begin{equation}
  \begin{split}
    L_{\mathrm{lift}}(u, p;) &= \int_{\Gamma} \sigma(u, p) n \cdot e_z \ds, \\
    L_{\mathrm{drag}}(u, p;) &= \int_{\Gamma} \sigma(u, p) n \cdot e_x \ds.
  \end{split}
\end{equation}
Here, $\Gamma$ is the boundary of the body, $n$ is the outward
unit normal of $\Gamma$ and $e_x$, $e_z$ are unit vectors
in the $x$- and $z$-directions respectively. The arity of both forms
is $\rho = 0$ and both forms have two coefficients.

\begin{figure}
  \centering
  \fenicsfig{kirby-5}{lift_drag}{\largefig}
  \caption{The lift and drag of an object, here a NACA 63A409
      airfoil, are the integrals of the vertical and horizontal
      components respectively of the stress $\sigma \cdot n$ over the
      surface~$\Gamma$ of the object. At each point, the product of
      the stress tensor~$\sigma$ and the outward unit normal
      vector~$n$ gives the force per unit area acting on the
      surface.}
\end{figure}

\subsection{Canonical form}

FEniCS automatically handles the representation and evaluation of a
large class of multilinear forms, but not all. FEniCS is currently
limited to forms that may be expressed as a sum of integrals over the
cells (the domain), the exterior facets (the boundary) and the
interior facets of a given mesh. In particular, FEniCS handles forms
that may be expressed as the following canonical form:
\begin{equation} \label{eq:kirby-5:integrals}
  a(w_1, w_2, \ldots, w_n; v_{\rho}, \ldots, v_2, v_1)
  =
  \sum_{k=1}^{n_c}   \int_{\Omega_k} I^c_k \dx +
  \sum_{k=1}^{n_f}   \int_{\Gamma_k} I^f_k \ds +
  \sum_{k=1}^{n_f^0} \int_{\Gamma_k^0} I^{f,0}_k \dS.
\end{equation}
Here, each $\Omega_k$ denotes a union of mesh cells covering a subset
of the computational domain~$\Omega$. Similarly, each $\Gamma_k$
denotes a subset of the facets on the boundary of the mesh, and
$\Gamma_k^0$ denotes a subset of the interior facets of the
mesh. The latter is of particular interest for the formulation of
discontinuous Galerkin methods that typically involve integrals across
cell boundaries (interior facets). The contribution from each subset
is an integral over the subset of some integrand. Thus, the
contribution from the $k$th subset of cells is an integral over
$\Omega_k$ of the integrand $I^c_k$ etc.

One may consider extensions of~(\ref{eq:kirby-5:integrals}) that
involve point values or integrals over subsets of individual cells
(cut cells) or facets. Such extensions are currently not supported by
FEniCS but may be added in the future.

\section{Discretizing multilinear forms}
\label{sec:kirby-5:discretizing}

As we saw in Chapter~\ref{chap:kirby-7}, one may obtain the finite
element approximation $u_h = \sum_{j=1}^N U_j \phi_j \approx u$ of the
canonical variational problem~(\ref{eq:kirby-5:canonical}) by solving a
linear system $AU=b$, where
\begin{equation}
  \begin{split}
    A_{ij} &= a(\phi_j, \hat{\phi}_i), \quad i,j = 1,2,\ldots,N, \\
    b_i &= L(\hat{\phi}_i), \quad i = 1,2,\ldots,N.
  \end{split}
\end{equation}
Here, $A$ and $b$ are the discrete operators corresponding to the
bilinear and linear forms $a$ and $L$ for the given bases of the trial
and test spaces. Note that the discrete operator is defined as the
transpose of the multilinear form applied to the basis functions to
account for the fact that in a bilinear form $a(u, v)$, the trial
function~$u$ is associated with the columns of the matrix $A$,
while the test function~$v$ is associated with the rows (the
equations) of the matrix $A$.

In general, we may discretize a multilinear form~$a$ of arity~$\rho$
to obtain a tensor~$A$ of rank~$\rho$. The discrete operator $A$ is
defined by
\begin{equation}
  A_i = a(w_1, w_2, \ldots, w_n;
  \phi^{\rho}_{i_{\rho}}, \ldots, \phi^2_{i_2}, \phi^1_{i_1}),
\end{equation}
where $i = (i_1, i_2, \ldots, i_{\rho})$ is a multi-index of
length~$\rho$ and $\{\phi^j_k\}_{k=1}^{N_j}$ is a basis for $V_{j,h}
\subset V_j$, $j = 1,2,\ldots,\rho$. The discrete operator is a
typically sparse tensor of rank~$\rho$ and dimension $N_1 \times N_2
\times \cdots \times N_{\rho}$.

The discrete operator~$A$ may be computed efficiently using an
algorithm known as~\emph{assembly}, which is the topic of the next
chapter. As we shall see then, an important tool is the \emph{cell
  tensor} obtained as the discretization of the bilinear form on a
local cell of the mesh. In particular, consider the discretization of
a multilinear form that may be expressed as a sum of local
contributions from each cell~$T$ of a mesh~$\mathcal{T}_h = \{T\}$,
\begin{equation}
  a(w_1, w_2, \ldots, w_n; v_{\rho}, \ldots, v_2, v_1)
  = \sum_{T\in\mathcal{T}_h}
  a_T(w_1, w_2, \ldots, w_n; v_{\rho}, \ldots, v_2, v_1).
\end{equation}
Discretizing~$a_T$ using the local finite element
basis~$\{\phi^{T,j}_k\}_{k=1}^{n_j}$ on~$T$ for $j = 1,
2, \ldots, \rho$, we obtain the cell tensor
\begin{equation} \label{eq:kirby-5:celltensor}
  A_{T,i}
  = a_T(w_1, w_2, \ldots, w_n;
        \phi^{T,\rho}_{i_{\rho}}, \ldots, \phi^{T,2}_{i_2}, \phi^{T,1}_{i_1}).
\end{equation}
The cell tensor~$A_T$ is a typically dense tensor of rank~$\rho$ and
dimension $n_1 \times n_2 \times \cdots \times n_{\rho}$. The discrete
operator $A$ may be obtained by appropriately summing the
contributions from each cell tensor $A_T$. We return to this in detail
below in Chapter~\ref{chap:logg-3}.

\begin{figure}
  \centering
  \fenicsfig{kirby-5}{element_tensors}{\largefig}
  \caption{The cell tensor~$A_T$, exterior facet tensor $A_S$, and
    interior facet tensor $A_{S,0}$ on a mesh are obtained by
    discretizing the local contribution to a multilinear form on a
    cell, exterior facet or interior facet respectively. By
    assembling the local contributions from all cell and facet
    tensors, one obtains the global discrete operator $A$ that
    discretizes the multilinear form.}
  \label{fig:kirby-5:tensors}
\end{figure}

If $\Omega_k \subset \Omega$, the discrete operator $A$ may be
obtained by summing the contributions only from the cells covered by
$\Omega_k$. One may similarly define the exterior and interior facet
tensors $A_S$ and $A_{S,0}$ as the contributions from a facet on the
boundary or in the interior of the mesh. The exterior facet tensor
$A_S$ is defined as in~(\ref{eq:kirby-5:celltensor}) by replacing the
domain of integration $T$ by a facet $S$. The dimension of $A_S$ is
generally the same as that of $A_T$. The interior facet tensor
$A_{S,0}$ is defined slightly differently by considering the basis on
a \emph{macro element} consisting of the two elements sharing the
common facet~$S$ as depicted in Figure~\ref{fig:kirby-5:tensors}. For
details, we refer to~\citet{OelgaardLoggWells2008}.

\section{The action of a multilinear form}

Consider the bilinear form
\begin{equation}
  a(u, v) = \int_{\Omega} \nabla u \cdot \nabla v \dx,
\end{equation}
obtained from the discretization of the left-hand side of Poisson's
equation. Here, $u$ and $v$ are a pair of trial and test functions.
Alternatively, we may consider~$v$ to be a test function and $u$ to be
a given solution to obtain a \emph{linear} form parametrized over the
coefficient~$u$,
\begin{equation}
  (\mathcal{A} a) (u; v) = \int_{\Omega} \nabla u \cdot \nabla v \dx.
\end{equation}
We refer to the linear form $\mathcal{A}a$ as the \emph{action} of the
bilinear form~$a$. In general, the action of a $\rho$-linear form with
$n$ coefficients is a $(\rho-1)$-linear form with $n+1$ coefficients.
In particular, the action of a bilinear form is a linear form, and the
action of a linear form is a functional.

The action of a bilinear form plays an important role in the
definition of matrix-free methods for solving differential
equations. Consider the solution of a variational problem of the
canonical form~(\ref{eq:kirby-5:canonical}) by a Krylov subspace
method such as GMRES (Generalized Minimal RESidual
method) \citep{SaadSchultz1986} or CG (Conjugate Gradient
method) \citep{HestenesStiefel1952}. Krylov methods approximate the
solution $U\in\R^N$ of the linear system $AU=b$ by finding an
approximation for $U$ in the subspace of $\R^N$ spanned by the vectors
$b, Ab, A^2b, \ldots, A^kb$ for some $k \ll N$. These vectors may be
computed by repeated application of the discrete operator $A$ defined
as above by
\begin{equation}
  A_{ij} = a(\phi^2_j, \phi^1_i).
\end{equation}
For any given vector $U \in \R^N$, it follows that
\begin{equation}
  (A U)_i = \sum_{j=1}^N A_{ij} U_j
  = \sum_{j=1}^N a(\phi^2_j, \phi^1_i) U_j
  = a\left(\sum_{j=1}^N U_j \phi^2_j, \phi^1_i \right)
  = a(u_h, \phi^1_i) = (\mathcal{A} a) (u_h; \phi^1_i),
\end{equation}
where $u_h = \sum_{j=1}^N U_j \phi^2_j$ is the finite element
approximation corresponding to the coefficient vector~$U$.  In other
words, the application of the matrix~$A$ on a given vector~$U$ is
given by the action of the bilinear form evaluated at the
corresponding finite element approximation:
\begin{equation}
  (A U)_i = (\mathcal{A} a) (u_h; \phi^1_i).
\end{equation}
The variational problem $(\ref{eq:kirby-5:canonical})$ may thus be
solved by repeated evaluation (assembly) of a linear form (the action
$\mathcal{A} a$ of the bilinear form~$a$) as an alternative to first
computing (assembling) the matrix~$A$ and then repeatedly computing
matrix--vector products with~$A$. Which approach is more efficient
depends on how efficiently the action may be computed compared to
matrix assembly, as well as on available preconditioners. For a
further discussion on the action of multilinear forms, we refer
to~\citet{BagheriScott2004}.

Computing the action of a multilinear form is supported by the UFL
form language by calling the \emp{action} function:
\begin{python}
  a  = inner(grad(u), grad(v))*dx
  Aa = action(a)
\end{python}

\section{The derivative of a multilinear form}

When discretizing nonlinear variational problems, it may be of
interest to compute the derivative of a multilinear form with respect
to one or more of its coefficients. Consider the nonlinear variational
problem to find $u \in V$ such that
\begin{equation} \label{eq:kirby-5:nonlinear}
  a(u; v) = 0 \quad \foralls v \in \hat{V}.
\end{equation}
To solve this problem by Newton's method, we linearize around a fixed
value~$\bar{u}$ to obtain
\begin{equation}
  0 = a(u; v) \approx a(\bar{u}; v) + a'(\bar{u}; v) \delta u.
\end{equation}
Given an approximate solution $\bar{u}$ of the nonlinear variational
problem~(\ref{eq:kirby-5:nonlinear}), we may then hope to improve the
approximation by solving the following linear variational problem:
Find $\delta u \in V$ such that
\begin{equation}
  a'(\bar{u}; \delta u, v) \equiv a'(\bar{u}; v) \delta u = -a(\bar{u}; v)
  \quad \foralls v \in \hat{V}.
\end{equation}
Here, $a'$ is a bilinear form with two arguments $\delta u$ and $v$,
and one coefficient $\bar{u}$, and $-a$ is a linear form with one
argument $v$ and one coefficient $\bar{u}$.

When there is more than one coefficient, we use the notation
$\mathcal{D}_w$ to denote the derivative with respect to a specific
coefficient~$w$. In general, the derivative $\mathcal{D}$ of a
$\rho$-linear form with $n > 0$ coefficients is a $(\rho+1)$-linear
form with $n$ coefficients. To solve the variational
problem~(\ref{eq:kirby-5:nonlinear}) using a matrix-free Newton
method, we would thus need to repeatedly evaluate the linear form
$(\mathcal{A}\mathcal{D}_u a) (\bar{u}_h, \delta u_h; v)$ for a given
finite element approximation~$\bar{u}_h$ and increment $\delta u_h$.

Note that one may equivalently consider the application of Newton's
method to the nonlinear discrete system of equations obtained by a
direct application of the finite element method to the variational
problem~(\ref{eq:kirby-5:nonlinear}) as discussed in
Chapter~\ref{chap:kirby-7}.

Computing the derivative of a multilinear form is supported by the UFL
form language by calling the \emp{derivative} function:
\begin{python}
  a  = inner(grad(u)*u, v)*dx
  Da = derivative(a, u)
\end{python}

\section{The adjoint of a bilinear form}

The adjoint $a^*$ of a bilinear form~$a$ is the form obtained by
interchanging the two arguments,
\begin{equation}
  a^*(v, w) = a(w, v) \quad \foralls v \in V^1 \quad \foralls w \in V^2.
\end{equation}
The adjoint of a bilinear form plays an important role in the error
analysis of finite element methods as we saw in
Chapter~\ref{chap:kirby-7} and as will be discussed further in
Chapter~\ref{chap:selim} where we consider the linearized adjoint
problem (the dual problem) of the general nonlinear variational
problem~(\ref{eq:kirby-5:nonlinear}). The dual problem takes the form
\begin{equation}
  (\mathcal{D}_u a)^* (u; z, v) = \mathcal{D}_u \mathcal{M} (u; v) \quad \foralls v \in V,
\end{equation}
or simply
\begin{equation}
  a'^* (z, v) = \mathcal{M}' (v) \quad \foralls v \in V,
\end{equation}
where $(\mathcal{D}_u a)^*$ is a bilinear form, $\mathcal{D}_u
\mathcal{M}$ is a linear form (the derivative of the
functional~$\mathcal{M}$), and $z$ is the solution of the dual
problem.

Computing the adjoint of a multilinear form is supported by the UFL
form language by calling the \emp{adjoint} function:
\begin{python}
  a = div(u)*p*dx
  a_star = adjoint(a)
\end{python}

\section{A note on the order of trial and test functions}

It is common in the literature to consider bilinear forms where the
trial function~$u$ is the first argument, and the test function~$v$ is
the second argument:
\begin{equation}
  a = a(u, v).
\end{equation}
With this notation, one is lead to define the discrete operator~$A$ as
\begin{equation}
  A_{ij} = a(\phi_j, \phi_i),
\end{equation}
that is, a transpose must be introduced to account for the fact that
the order of trial and test functions does not match the order of rows
and columns in a matrix. Alternatively, one may change the order of
trial and test functions and write $a = a(v, u)$ and avoid taking the
transpose in the definition of the discrete operator $A_{ij} =
a(\phi_i, \phi_j)$. This is practical in the definition and
implementation of software systems such as FEniCS for the general
treatment of variational forms.

In this book and throughout the code and documentation of the FEniCS
Project, we have adopted the following compromise. Variational forms
are expressed using the conventional \emph{order} of trial and test
functions; that is,
\begin{equation}
  a = a(u, v),
\end{equation}
but using an unconventional \emph{numbering} of trial and test
functions. Thus, $v$ is the first argument of the bilinear form and
$u$ is the second argument. This ensures that one may express finite
element variational problems in the conventional notation, but at the
same time allows the implementation to use a more practical numbering
scheme.
