\fenicschapter{Image-Based Computational Hemodynamics}
              {Image-Based Computational Hemodynamics}
              {Luca Antiga}
              {antiga}

The physiopathology of the cardiovascular system has been observed to  
be tightly linked to the local in-vivo hemodynamic environment. For  
this reason, numerical simulation of patient-specific hemodynamics is  
gaining ground in the vascular research community, and it is expected  
to start playing a role in future clinical environments.
For the non-invasive characterization of local hemodynamics on the  
basis of information drawn from medical images, robust workflows from  
images to the definition and the discretization of computational  
domains for numerical analysis are required.
In this chapter, we present a framework for image analysis, surface  
modeling, geometric characterization and mesh generation provided as  
part of the Vascular Modeling Toolkit (VMTK), an open-source effort.
Starting from a brief introduction of the theoretical bases of which  
VMTK is based, we provide an operative description of the steps  
required to generate a computational mesh from a medical imaging data  
set. Particular attention will be devoted to the integration of the  
Vascular Modeling Toolkit with FEniCS. All aspects covered in this  
chapter are documented with examples and accompanied by code and  
data, which allow to concretely introduce the reader to the field of  
patient-specific computational hemodynamics.



