\fenicschapter{UFL: A High-Level Language for Discrete Variational Forms}
              {UFL: A High-Level Language for Discrete Variational Forms}
              {Martin S. Aln\ae{}s, Anders Logg, and Kent-Andre Mardal}

\editornote{[alnes-1]}

One of the key steps for an end-user implementing a simulator
application in the FEniCS framework is defining the PDEs to
solve. This includes choosing finite elements and defining the
discrete variational form. A major design goal for this part of the
FEniCS framework is to stay as close to the mathematical formulation
as possible. The Unified Form Language (UFL) specifies a high-level
Domain Specific Language (DSL) for this purpose. Some key fea- tures
of UFL are support for arbitrary nested hierarchies of mixed elements,
tensor algebra with implicit summation over repeated indices, and
automatic functional differentiation to compute Jacobian matrices for
nonlinear problems. The present chapter will explain these features
in more detail and provide many examples of what UFL can do.
