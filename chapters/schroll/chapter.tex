\fenicschapter{Calibration of Depositional Models}
              {Calibration of Depositional Models}
              {Hans Joachim Schroll}

\editornote{[schroll]}

\section{Issues in Dual Lithology Sedimentation}

Different types of forward computer models are being used by
sedimentologists and geomorphologists to simulate the process of
sedimentary deposition over geological time periods. The models can be
used to predict the presence of reservoir rocks and stratigraphic
traps at a variety of scales. State-of-the-art advanced numerical
software provides accurate approximations to the mathe- matical model,
which commonly is expressed in terms of a nonlinear diffusion
dominated PDE system. The potential of todays simulation software in
indus- trial applications is limited however, due to major
uncertainties in crucial ma- terial parameters that combine a number
of physical phenomena and therefore are difficult to
quantify. Examples of such parameters are diffusive transport
coefficients.

The idea in this contribution is to calibrate uncertain transport
coefficients to direct observable data, like well measurements from a
specific basin. In this approach the forward evolution process,
mapping data to observations, is reversed to determine the data, i.e.,
transport coefficients. Mathematical tools and numerical algorithms
are applied to automatically calibrate geological models to actual
observations --- a critical but so far missing link in forward
depositional modelling.

Automatic calibration, in combination with stochastic modelling, will
boost the applicability and impact of modern numerical simulations in
industrial ap- plications.
