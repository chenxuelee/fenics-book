\fenicschapter{Boussinesq Surface Wave Equations}
              {Boussinesq Surface Wave Equations}
              {N. Lopes, P. Pereira, and L. Trabucho}

\editornote{[lopes]}

The FEniCS project, via DOLFIN and FFC, provides a good support for
the implemen- tation of several large scale industrial models. We
implement a solver for an improved Boussinesq type system that models
the evolution of surface water waves in a variable depth seabed. This
type of models is used, for instance in harbour simulation.

There are several Boussinesq models and the most widely used are the
ones based on wave Elevation and horizontal Velocities formulation
(BEV), see for instance: [Wal99], [LW04] and the references therein.

In this work we use the model proposed by Zhao, Teng and Cheng in
[ZTC04]. This model uses as unknowns the wave Elevation and the
velocity Potential (BEP), reducing the number of system equations and
the spatial dierentiation order, when compared to the (BEV) models.

The model considered is described by the following set of equations:
...  where $h$ stands for the water depth, $g$ the gravity, $\eta$
represents the wave elevation and $\phi$ the velocity potential. We
consider different examples with standard boundary conditions: total
reective walls, corresponding to the standard zero Neumann conditions;
incident wave boundaries, corresponding to time dependent Dirichlet
conditions.

As we consider several types of sponge layers near the wave breaking
zone, we add some extra damping terms to the above equations.

We improve the (ZTC) model with an internal wave generation
method, proposed in [WKS99] for the (BEV) models by considering also
time-dependent source functions.

The numerical treatment of the model is made using Lagrange P1 and P2
elements for the discretization of the spacial variables. For the time
integration we consider several Runge-Kutta and Predictor-Corrector
algorithms.

Several standard benchmarks were done, for instance the Gaussian wave
evolution in a rectangular basin, as in [LW04], showing good agreement
with the improved (BEV) models. Some new test cases are considered,
essentially wave evolution and interaction in variable seadepth domain
with obstacles.
