\fenicschapter{Finite Element Analysis of Electromagnetic Waveguides}
              {Finite Element Analysis of Electromagnetic Waveguides}
              {Evan Lezar and David B. Davidson}

\editornote{[lezar]}

Waveguides are some of the most common structures in microwave
engineering and are especially useful in applications where high power
and low loss are es- sential [1]. Waveguide analysis was one of the
earliest applications of the finite element method in
electromagnetics, and it remains an important topic in microwave
engineering. This chapter considers the use of FEnICS in the
dispersion and cutoff analysis of these structures as well as the
analysis of waveguide obsta- cles and discontinuities.Both two and
three dimensional analysis is considered.

Two-dimensional waveguide analysis focuses on the eigenmodes in an
infinitely long waveguide of constant cross-section which can be
homogeneously or inho- mogeneously filled. For the former, the fields
are either transverse electric or transverse magnetic, which
simplifies the analysis; for the latter, the modes are generally of a
hybrid nature. The analysis results which are required are firstly,
modal cutoff frequencies, and secondly, a dispersion analysis
($\omega$--$\beta$ diagrams).  The simplest case of a hollow
rectangular waveguide is considered first. This problem is ideal for
testing and validation as analytical solutions are readily
available.Furthermore, convergence studies can be undertaken. Other
waveg- uide configurations to be considered are a dielectric loaded
waveguide as well as a boxed microstrip line. Both of these support
hybrid modes, and in the latter case, complex modes can also occur
under certain conditions. (Complex modes have complex propagation
constants even in lossless media, and occur in complex conjugate
pairs) [2]. The question of open waveguides may also be considered.

Obstacle and discontinuity analysis is a more challenging class of
problems from a computational viewpoint, as they are inherently three
dimensional. These are driven problems, and typically involve
determining the scattering (S) parameters for a particular guide
configuration with various approaches possible for intro- ducing the
source. Configurations such as a Magic-T (a power divider/combiner)
and a waveguide containing a capacitive iris [3] will be
considered. More ad- vanced applications, such as a waveguide filter
comprising a number of metallic septa, will also be addressed.
