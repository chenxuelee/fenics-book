\fenicschapter{Applications in solid mechanics}
              {Applications in solid mechanics}
              {Applications in solid mechanics}
              {Kristian B. {\O}lgaard and Garth N. Wells}
              {oelgaard-1}

%------------------------------------------------------------------------------
Problems in solid mechanics constitute perhaps the largest field of
application of finite element methods. The vast majority of solid
mechanics problems involve the standard momentum balance equation,
posed in a Lagrangian setting, with different models distinguished by the
choice of nonlinear or linearized kinematics, and the constitutive model
for determining the stress. For some common models, the constitutive
relationships are rather complex. This chapter addresses a number of
canonical solid mechanics models in the context of automated modeling,
and focuses on some pertinent issues that arise due to the nature of the
constitutive models. The solution of equations with second-order time
derivatives, which characterizes dynamic problems, is also considered.

%------------------------------------------------------------------------------
\section{Background}

We present in this chapter the solution of a collection of common
solid mechanics problems using automated code generation techniques.
For users familiar with traditional development techniques for
solid mechanics problems, it is often not evident how the automation
techniques established with the FEniCS Project should be applied to
solid mechanics problems.  The traditional development approach to solid
mechanics problems, and traditional finite element codes, places a strong
emphasis on the implementation of constitutive models at the quadrature
point level. Automated methods, on the other hand, tend to stress more
heavily the governing balance equations.  Widely used finite element
codes for solid mechanics applications provide application programming
interfaces (APIs) for users to implement their own constitutive models.
The interface supplies kinematic and history data, and the user code
computes the stress tensor, and when required also the linearization
of the stress. Users of such libraries will typically not be exposed to
code development other than via the constitutive model API.

The purpose of this chapter is to illustrate how problems of relevance
in solid mechanics can be solved using automation tools.  We consider
the common problems of linearized elasticity, plasticity, hyperelasticity
and elastic wave propagation.  Topics that we address via these problems
include `off-line' computation of stress updates, linearization of
problems with off-line stress updates, automatic differentiation and time
stepping for problems with second-order time derivatives. The presentation
starts with the relevant governing equations and some constitutive models,
followed by a summary of a commonly used time stepping method.  We then
address the important issue of solution and linearization of problems
in which the governing equation is expressed in terms of the stress
tensor (rather than explicitly in terms of the displacement field, or
derivatives of the displacement field), and the stress tensor is computed
via a separate algorithm.  These topics are then followed by a number
of examples that demonstrate implementation approaches.  Finally,
two future extensions of the \fenics framework that are particular
interesting with respect to solid mechanics problems are summarized.

This chapter does not set out to provide a comprehensive treatment of
solid mechanics problems. It addresses a number of the most frequently
encountered issues when applying automated techniques to solid mechanics
problems. It should be clear from the considered examples how a wider
range of common solid mechanics problems can be tackled using automated
modeling.

%------------------------------------------------------------------------------
\section{Governing equations}
%------------------------------------------------------------------------------
\subsection{Preliminaries}

We will consider problems posed on a polygonal domain $\Omega \subset
\mathbb{R}^{d}$, where $1 \le d \le 3$. The boundary of $\Omega$,
denoted by $\partial \Omega$, is decomposed into regions $\Gamma_{D}$
and $\Gamma_{N}$ such that $\Gamma_{D} \cup \Gamma_{N} = \partial \Omega$
and $\Gamma_{D} \cap \Gamma_{N} = \emptyset$.  The outward unit normal
vector on $\partial \Omega$ will be denoted by~$n$.  For time-dependent
problems, we will consider a time interval of interest~$I = (0, T]$
and let superimposed dots denote time derivatives.  We will use $\Omega$
to denote the current configuration of a solid body; that is, the domain
$\Omega$ depends on the displacement field. It is sometimes convenient
to also define a reference domain $\Omega_{0} \subset \mathbb{R}^{d}$
that remains fixed.  For convenience, we will consider cases in which
$\Omega$ and $\Omega_{0}$ coincide at time $t = 0$.  To indicate
boundaries, outward unit normal vectors, and other quantities relative
to $\Omega_{0}$, the subscript `$0$' will be used.  When considering
linearized kinematics, the domains $\Omega$ and $\Omega_{0}$ are both
fixed and coincide at all times~$t$.  A triangulation of the domain
$\Omega$ will be denoted by~$\mathcal{T}_h$, and a triangulation of the
domain $\Omega_{0}$ will be denoted by~$\mathcal{T}_{0}$.  A finite
element cell will be denoted by $T \in \mathcal{T}_h$.

The governing equations for the different models will be formulated in
the common framework of: find $u \in V$ such that
%
\begin{equation}
  F(u; w) = 0 \quad \foralls w \in V,
  \label{eq:oelgaard-1:gov_generic}
\end{equation}
%
where $F: V \times V \rightarrow \mathbb{R}$ is linear in $w$ and $V$
is a suitable function space. If $F$ is also linear in $u$, then $F$
can be expressed as
%
\begin{equation}
  F(u; w) := a(u, w) - L(w),
\end{equation}
%
where $a: V \times V \rightarrow \mathbb{R}$ is linear in $u$ and in $w$,
and $L: V \rightarrow \mathbb{R}$ is linear in~$w$.  For this case, the
problem can be cast in the canonical setting of: find $u \in V$ such that
%
\begin{equation}
  a(u, w) = L(w) \quad \foralls w \in V.
\label{eq:oelgaard-1:linear_weak_form}
\end{equation}
%
For nonlinear problems, a Newton method is typically employed to
solve~\eqref{eq:oelgaard-1:gov_generic}.  Linearizing $F$ about $u =
u_{0}$ leads to a bilinear form,
%
\begin{equation}
  a(du, w) := D F_{du}\brac{u_{0}; w}
     = \left. \frac{d F\brac{u_{0} + \epsilon du; w}}{d \epsilon} \right|_{\epsilon = 0},
\label{eq:oelgaard-1:gov_jacobian}
\end{equation}
%
and a linear form is given by:
%
\begin{equation}
  L(w) := F(u_{0}, w).
\label{eq:oelgaard-1:gov_residual}
\end{equation}
%
Using the definitions of $a$ and $L$ in~\eqref{eq:oelgaard-1:gov_jacobian}
and~\eqref{eq:oelgaard-1:gov_residual}, respectively,
a Newton step involves solving a problem of the type
in~\eqref{eq:oelgaard-1:linear_weak_form}, followed by the correction
$u_{0} \leftarrow u_{0} - du$. The process is repeated until
\eqref{eq:oelgaard-1:gov_generic} is satisfied to within a specified
tolerance.

%------------------------------------------------------------------------------
\subsection{Balance of momentum}
%
The standard balance of linear momentum problem for the body $\Omega$
reads:
%
\begin{align}
  \rho \ddot{u} - \nabla \cdot \sigma  &= b \quad \text{in} \;
    \Omega \times I,
  \label{eq:oelgaard-1:gov_balance_momentum}
\\
  u &= g \quad \text{on} \; \Gamma_{D} \times I,
\\
  \sigma  n &= h \quad \text{on} \; \Gamma_{N} \times I,
  \label{eq:oelgaard-1:gov_neumann_bc}
\\
  u\brac{x, 0} &= u_{0} \quad \text{in} \; \Omega,
  \label{eq:oelgaard-1:gov_initial_condition_0}
\\
  \Dot{u}(x, 0) &= v_{0} \quad \text{in} \; \Omega,
  \label{eq:oelgaard-1:gov_initial_condition_1}
\end{align}
%
where $\rho: \Omega \times I \rightarrow \mathbb{R}$ is the mass density,
$u: \Omega \times I \rightarrow \mathbb{R}^{d}$ is the displacement field,
$\sigma: \Omega \times I \rightarrow \mathbb{R}^{d} \times \mathbb{R}^{d}$
is the symmetric Cauchy stress tensor, $b: \Omega \times I \rightarrow
\mathbb{R}^{d}$ is a body force, $g: \Omega \times I \rightarrow
\mathbb{R}^{d}$ is a prescribed boundary displacement, $h: \Omega \times
I \rightarrow \mathbb{R}^{d}$ is a prescribed boundary traction, $u_{0}:
\Omega \rightarrow \mathbb{R}^{d}$ is the initial displacement and $v_{0}:
\Omega \rightarrow \mathbb{R}^{d}$ is the initial velocity.
To complete the boundary value problem, a constitutive model that relates
$\sigma$ to $u$ is required.

To develop finite element models, it is necessary to cast the momentum
balance equation in a weak form by multiplying the balance equation
\eqref{eq:oelgaard-1:gov_balance_momentum} by a weight function $w$
and integrating.  It is possible to formulate a space-time method by
considering a weight function that depends on space and time, and then
integrating over $\Omega \times I$. However, it is far more common in
solid mechanics applications to consider a weight function that depends
on spatial position only and to apply finite difference methods to deal
with time derivatives.  Following this approach, at a time $t \in I$
we multiply \eqref{eq:oelgaard-1:gov_balance_momentum} by a function
$w$ ($w$ is assumed to satisfy $w = 0$ on $\Gamma_{D}$) and integrate
over~$\Omega$:
%
\begin{equation}
  \int_{\Omega} \rho \ddot{u} \cdot w \dx
- \int_{\Omega} \left( \nabla \cdot \sigma \right) \cdot w \dx
- \int_{\Omega} b \cdot w \dx = 0.
\end{equation}
%
Applying integration by parts, using the divergence theorem and inserting
the boundary condition \eqref{eq:oelgaard-1:gov_neumann_bc}, we obtain:
%
\begin{equation}
F := \int_{\Omega} \rho \ddot{u} \cdot w \dx
  + \int_{\Omega} \sigma : \nabla w \dx
  - \int_{\Gamma_{N}} h \cdot w \ds
  - \int_{\Omega} b \cdot w \dx = 0.
\label{eqn:oelgaard-1:F}
\end{equation}

In this section, the momentum balance equation has been presented
on the current configuration~$\Omega$. It can also be posed on the
fixed reference domain $\Omega_{0}$ via a pull-back operation.  For the
particular presentation that we will use in this chapter for geometrically
nonlinear models details of the pull-back will not be needed.
%------------------------------------------------------------------------------
\subsection{Potential energy minimization}
\label{sec:oelgaard-1:minimization}

An alternative approach to solving static problems (problems without an
inertia term) is to consider the minimization of potential energy. This
approach leads to the same governing equation when applied to a
standard problem, but may be a preferable framework for problems that are
naturally posed in terms of stored energy densities and for which external
forcing terms are conservative (see \citet[p.~159]{Holzapfel2000} for
an explanation of conservative loading), and for problems that involve
coupled physical phenomena that are best described energetically.

Consider a system for which the total potential energy $\Pi$ associated
with a body can be expressed as
%%
\begin{equation}
\label{eq:oelgaard-1:gov_potential_energy}
  \Pi = \Pi_{\rm int} + \Pi_{\rm ext}.
\end{equation}
%%
We will consider an internal potential energy functional of the form
%%
\begin{equation}
  \Pi_{\rm int} = \int_{\Omega_{0}} \Psi_{0}\brac{u} \dx,
\end{equation}
%
where $\Psi_{0}$ is the stored strain energy density, and an
external potential energy functional of the form
%%
\begin{equation}
  \Pi_{\rm ext} = - \int_{\Omega_{0}} b_{0} \cdot u \dx
      - \int_{\Gamma_{0, N}} h_{0} \cdot u \ds.
\end{equation}
%
It is the form of the stored energy density function $\Psi_{0}$ that
defines a particular constitutive model.  For later convenience,
the potential energy terms have been presented on the reference
domain~$\Omega_{0}$.

A stable solution $u$ to~\eqref{eq:oelgaard-1:gov_potential_energy}
minimizes the potential energy:
%%
\begin{equation}
  \min_{u \in V} \Pi,
\end{equation}
%%
where $V$ is a suitably defined function space.  Minimization of $\Pi$
corresponds to the directional derivative of $\Pi$ being zero for all
possible variations of~$u$.  Therefore, minimization of $\Pi$ corresponds
to solving~\eqref{eq:oelgaard-1:gov_generic} with
%%
\begin{equation}
\label{eq:oelgaard-1:gov_energy_variation}
  F(u; w) := D_{w} \Pi\brac{u}
      = \left. \frac{d \Pi\brac{u + \epsilon w}}{d \epsilon} \right|_{\epsilon = 0}.
\end{equation}
%%
For suitable definitions of the stress tensor, it is straightforward
to show that minimizing $\Pi$ is equivalent to solving the balance of
momentum problem, for the static case.

%------------------------------------------------------------------------------
\section{Constitutive models}

A constitutive model describes the relationship between stress and
deformation.  The stress can be defined explicitly in terms of primal
functions, it can be implicitly defined via stored energy density
functions, or it can be defined as the solution to a secondary problem.
The constitutive model can be either linear or nonlinear.  In the
following sections we present examples of these cases in the form of
linearized elasticity, plasticity and hyperelasticity.  The expressions
for the stress or stored energy density presented in this section can
be inserted into the balance equations or the minimization framework in
the preceding section to yield a governing equation.

%------------------------------------------------------------------------------
\subsection{Linearized elasticity}
\label{sec:oelgaard-1:lin_elasticity}

For linearized elasticity the stress tensor as a function of the strain
tensor for an isotropic, homogeneous material is given by
%
\begin{equation}
  \sigma = 2\mu \varepsilon + \lambda {\rm tr}(\varepsilon) I,
\label{eq:oelgaard-1:elas_stress_def}
\end{equation}
%
where $\varepsilon = \brac{\nabla u + \brac{\nabla u}^{\rm T}}/2$
is the strain tensor, $\mu$ and $\lambda$ are the Lam\'{e} parameters,
and $I$ is the second-order identity tensor.  The relationship between
the stress and the strain can also be expressed as
%
\begin{equation}
  \sigma = \mathcal{C} : \varepsilon,
\end{equation}
%%
where
%%
\begin{equation}
  \mathcal{C}_{ijkl} = \mu\brac{\delta_{ik}\delta_{jl} + \delta_{il}\delta_{jk}} +
  \lambda\delta_{ij}\delta_{kl},
\label{eq:oelgaard-1:elas_constitutive_tensor}
\end{equation}
%%
and $\delta_{ij}$ is the Kronecker-Delta.

%------------------------------------------------------------------------------
\subsection{Flow theory of plasticity}
\label{sec:oelgaard-1:plasticity}

We consider the standard flow theory model of plasticity, and present
only the background necessary to support the examples that we will
present. In depth coverage can be found in many textbooks, such
as \citet{Lubliner2008}.

For a geometrically linear plasticity problem, the stress tensor is
computed by
%%
\begin{equation}
  \sigma = \mathcal{C} : \varepsilon^{\rm e},
\label{eq:oelgaard-1:plas_stress_def}
\end{equation}
%
where $\varepsilon^{\rm e}$ is the elastic part of the strain tensor.
It is assumed that the strain tensor can be decomposed additively
into elastic and plastic parts:
%
\begin{equation}
  \varepsilon = \varepsilon^{\rm e} + \varepsilon^{\rm p}.
\label{eq:oelgaard-1:plas_strain_decomposed}
\end{equation}
%
If $\varepsilon^{\rm e}$ can be determined, then the stress can be
computed.

The stress tensor in classical plasticity models must satisfy the yield
criterion:
%
\begin{equation}
  f\brac{\sigma, \varepsilon^{\rm p}, \kappa} :=
  \phi\brac{\sigma, q_{\rm kin}\brac{\varepsilon^{\rm p}}} - q_{\rm iso}\brac{\kappa} - \sigma_{\rm y} \leqslant 0,
\label{eq:oelgaard-1:plas_yield_function}
\end{equation}
%%
where $\phi\brac{\sigma, q_{\rm kin}\brac{\varepsilon^{\rm p}}}$ is a
scalar effective stress measure, $q_{\rm kin}$ is a stress-like internal
variable used to model kinematic hardening, $q_{\rm iso}$ is a scalar
stress-like term used to model isotropic hardening, $\kappa$ is a scalar
internal variable and $\sigma_{\rm y}$ is the initial scalar yield stress.
For the commonly adopted von Mises model (also known as $J_{2}$-flow)
with linear isotropic hardening, $\phi$ and $q_{\rm iso}$ read:
%
\begin{align}
\label{eq:oelgaard-1:plas_von_mises_linear_hardening}
  \phi\brac{\sigma} &= \sqrt{\frac{3}{2}s_{ij}s_{ij}},
\\
  q_{\rm iso}\brac{\kappa} &= H\kappa,
\end{align}
%
where $s_{ij} = \sigma_{ij} - \sigma_{kk}\delta_{ij}/3$ is the deviatoric
stress and the constant scalar $H > 0$ is a hardening parameter.

In the flow theory of plasticity, the plastic strain rate is given by:
%
\begin{equation}
  \dot{\varepsilon}^{\rm p} = \Dot{\lambda}\frac{\partial g}{\partial\sigma},
\label{eq:oelgaard-1:plas_plas_strain}
\end{equation}
%
where $\Dot{\lambda}$ is the rate of the plastic multiplier and the
scalar $g$ is known as the plastic potential.  In the case of associative
plastic flow, $g = f$.  The term $\dot{\lambda}$ determines the magnitude
of the plastic strain rate, and the direction is given by $\partial
g/\partial\sigma$.  For isotropic strain-hardening, it is usual to set
%
\begin{equation}
  \dot{\kappa} = \sqrt{\frac{2}{3} \dot{\varepsilon}^{\rm p}_{ij}\dot{\varepsilon}^{\rm p}_{ij}},
\end{equation}
%
which for associative von Mises plasticity implies that $\dot{\kappa}
= \dot{\lambda}$.

A feature of the flow theory of plasticity is that the constitutive
model is postulated in a rate form. This requires the application
algorithms to compute the stress from increments of the total strain.
A discussion of algorithmic aspects on how the stress tensor can be
computed from the equations presented in this section is postponed to
Section~\ref{sec:oelgaard-1:examples_plasticity}.

%------------------------------------------------------------------------------
\subsection{Hyperelasticity}
\label{sec:oelgaard-1:hyperelasticity}

Hyperelastic models are characterized by the existence of a stored
strain energy density function $\Psi_{0}$. The linearized model presented at
the start of this section falls with the class of hyperelastic models.
Assuming linearized kinematics, the stored energy function
%
\begin{equation}
\label{eq:oelgaard-1:hyp_small_strain_energy}
\Psi_{0} = \frac{\lambda}{2} \brac{{\rm tr} \, \varepsilon}^{2} +
       \mu \varepsilon:\varepsilon,
\end{equation}
%
corresponds to the linearized model
in~\eqref{eq:oelgaard-1:elas_stress_def}.  It is straightforward to
show that using this stored energy function in the potential energy
minimization approach in~\eqref{eq:oelgaard-1:gov_energy_variation}
leads to the same equation as inserting the stress
from~\eqref{eq:oelgaard-1:elas_stress_def} into the weak momentum balance
equation~\eqref{eqn:oelgaard-1:F}.

More generally, stored energy functions that correspond to nonlinear
models can be defined. A wide range of stored energy functions for
hyperelastic models have been presented and analyzed in the literature
(see, for example, \citet{bonet1997} for a selection). In order to
present concrete examples, it is necessary to introduce some kinematics,
and in particular strain measures.  The Green--Lagrange strain tensor
$E$ is defined in terms of the deformation gradient $F : \Omega_{0}
\times I \rightarrow \mathbb{R}^{d} \times \mathbb{R}^{d}$, and right
Cauchy--Green tensor~$C: \Omega_{0} \times I \rightarrow \mathbb{R}^{d}
\times \mathbb{R}^{d}$:
%
\begin{align}
\label{eq:oelgaard-1:hyp_green_strain}
F &= I + \nabla u,
\\
C &= F^{\rm T} F,
\\
E &= \frac{1}{2} \brac{C - I},
\end{align}
%
where $I$ is the second-order identity tensor.  Using $E$ in
\eqref{eq:oelgaard-1:hyp_small_strain_energy} in place of the
infinitesimal strain tensor $\varepsilon$, we obtain the following
expression for the strain energy density function:
%
\begin{equation}
\label{eq:oelgaard-1:hyp_st_venant_kirchhoff}
  \Psi_{0} = \frac{\lambda}{2} \brac{{\rm tr} \, E}^{2} + \mu E:E,
\end{equation}
%
which is known as the St. Venant--Kirchhoff model.  Unlike the linearized
case, this energy density function is not linear in~$u$ (or spatial
derivatives of $u$), which means that when minimizing the total potential
energy~$\Pi$, the resulting equations are nonlinear.  Another example
of a hyperelastic model is the compressible neo-Hookean model:
%
\begin{equation}
\label{eq:oelgaard-1:hyp_neo_hookean}
  \Psi_{0} = \frac{\mu}{2} (I_{C} -3) - \mu \ln J + \frac{\lambda}{2} (\ln J)^{2},
\end{equation}
%
where $I_{C} = {\rm tr} \, C $ and $J = \det F$.

In most presentations of hyperelastic models, one would proceed from the
definition of the stored energy function to the derivation of a stress
tensor, and then often to a linearization of the stress for use in a
Newton method.  This process can be lengthy and tedious. For a range of
models, features of the Unified Form Language (UFL, Chapter~\ref{chap:alnes-1})
will permit problems to be posed as energy
minimization problems, and it will not be necessary to compute expression
for a stress tensor, or its linearization, explicitly. A particular model
can then be posed in terms of a particular expression for~$\Psi_{0}$. It
is also possible to follow the momentum balance route, in which case
\ufl{} can be used to compute the stress tensor and its linearization
automatically from an expression for~$\Psi_{0}$.

%------------------------------------------------------------------------------
\section{Time integration}
\label{sec:oelgaard-1:time_integration}

In this chapter we focus on the Newmark family of methods, which are
widely used in structural dynamics.  It is a direct integration method,
in which the equations are evaluated at discrete points in time separated
by a time increment $\Delta t$.  Thus, the time step $t_{n + 1}$ is equal
to $t_{n} + \Delta t$.  While this chapter addresses the Newmark scheme,
it is straightforward to extend the approach (and implementation) to
generalized-$\alpha$ methods.

The Newmark relations between displacements, velocities and accelerations
at $t_{n}$ and $t_{n + 1}$ read:
%
\begin{align}
\label{eq:oelgaard-1:time_u_n1}
  u_{n + 1} &= u_{n} + \Delta t \dot{u}_{n} +
              \frac{1}{2}\Delta t^2 \brac{2\beta \ddot{u}_{n + 1} +
                \brac{1 - 2\beta}\Ddot{u}_{n + 1} },
\\
\label{eq:oelgaard-1:time_v_n1}
\dot{u}_{n + 1} &= \dot{u}_{n} + \Delta t \brac{\gamma \ddot{u}_{n + 1} +
                    \brac{1 - \gamma} \ddot{u}_{n}},
\end{align}
%
where $\beta$ and $\gamma$ are parameters.  Various well-known schemes are
recovered for particular combinations of $\beta$ and $\gamma$. Setting
$\beta = 1/4$ and $\gamma = 1/2$ leads to the trapezoidal scheme, and
setting $\beta = 0$ and $\gamma = 1/2$ leads to a central difference
scheme.  For $\beta > 0$, re-arranging~\eqref{eq:oelgaard-1:time_u_n1}
and using \eqref{eq:oelgaard-1:time_v_n1} leads to:
%
\begin{align}
\label{eq:oelgaard-1:time_a_n1}
\Ddot{u}_{n + 1} &= \frac{1}{\beta \Delta t^2}
                      \brac{u_{n + 1} - u_{n} - \Delta t \dot{u}_{n}} -
                      \brac{\frac{1}{2\beta} - 1} \ddot{u}_{n},
\\
\label{eq:oelgaard-1:time_v(u)_n1}
 \dot{u}_{n + 1} &= \frac{\gamma}{\beta \Delta t}
                      \brac{u_{n + 1} - u_{n}} -
                      \brac{\frac{\gamma}{\beta} - 1} \dot{u}_{n} -
                      \Delta t \brac{\frac{\gamma}{2\beta} - 1} \ddot{u}_{n},
\end{align}
%
in which $u_{n+1}$ is the only unknown term on the right-hand side.

To solve a time dependent problem, the governing equation can be posed
at time $t_{n+1}$,
%
\begin{equation}
  F\left( u_{n+1}; w \right) = 0 \quad \foralls w \in V,
\end{equation}
%
with the expressions in~\eqref{eq:oelgaard-1:time_a_n1}
and~\eqref{eq:oelgaard-1:time_v(u)_n1} used for first and second time
derivatives of~$u$ at time~$t_{n+1}$.

%------------------------------------------------------------------------------
\section{Linearization issues for complex constitutive models}
\label{sec:oelgaard-1:linearization}

Solving problems with nonlinear constitutive models, such
as plasticity, using Newton's method requires linearization
of~\eqref{eqn:oelgaard-1:F}. There are two particular issues that deserve
attention. The first is that if the stress $\sigma$ is computed via
some algorithm, then proper linearization of $F$ requires linearization
of the algorithm for computing the stress, and not linearization of the
continuous problem. This point is well known in computational plasticity,
and has been extensively studied \citep{SimoTaylor1985}. The second issue
is that the stress field, and its linearization, will not in general
come from a finite element space.  Hence, if all functions are assumed
to be in a finite element space, or are interpolated in a finite element
space, suboptimal convergence of a Newton method will be observed.

%------------------------------------------------------------------------------
\subsection{Consistency of linearization}

To illustrate the second issue raised in the preceding paragraph, we
consider the representation that the FEniCS Form Compiler (FFC,
Chapter~\ref{chap:logg-1}) would generate for a simple model
problem, and linearize this representation. We then consider how FFC
would represent a linearization of the original problem, which turns out
not to be consistent with the linearization of the FFC representation
of the original problem.

Consider the following one-dimensional problem:
%
\begin{equation}
\label{eq:oelgaard-1:lin_issue_F}
F\brac{u; w} := \int_{\Omega} \sigma w,_{x} \dx,
\end{equation}
%
where the scalar stress $\sigma$ is a nonlinear function of the strain
field $u_{,x}$, and will be computed via a separate algorithm outside of
the main forms. We consider a continuous, piecewise quadratic displacement
field (and likewise for $w$), and a strain field that is computed via
an $L^2$-projection onto the space of discontinuous, piecewise linear
elements (for the considered spaces, this is equivalent to a direct
evaluation of the strain).  We also represent the stress $\sigma$ on the
discontinuous, piecewise linear basis.  Since the polynomial degree of
the integrand is two, \eqref{eq:oelgaard-1:lin_issue_F} can be integrated
using two Gauss quadrature points on an element $T \in \mathcal{T}_h$:
%
\begin{equation}
\label{eq:oelgaard-1:lin_issue_f}
f_{T,i_{1}} := \sum_{q=1}^2 \sum_{\alpha=1}^2 \psi^{T}_{\alpha}\brac{x_{q}} \sigma_{\alpha}
\phi^{T}_{i_{1},x}\brac{x_q} W_{q},
\end{equation}
%
where $q$ is the integration point index, $\alpha$ is the degree of freedom
index for the local basis of $\sigma$, $\psi^{T}$ and $\phi^{T}$ denotes
the linear and quadratic basis functions on the element $T$, respectively,
and $W_{q}$ is the quadrature weight at integration point~$x_{q}$.
Note that $\sigma_{\alpha}$ is the stress at the element node~$\alpha$.

To apply a Newton method, the Jacobian (linearization)
of~\eqref{eq:oelgaard-1:lin_issue_f} is required. This will be
denoted by $A^{\star}_{T,i}$.  To achieve quadratic convergence of
a Newton method, the linearization must be exact.  The Jacobian of
\eqref{eq:oelgaard-1:lin_issue_f} is:
%
\begin{equation}
\label{eq:oelgaard-1:lin_issue_linearised_stiffness}
A^{\star}_{T,i} := \frac{df_{T,i_{1}}}{du_{i_{2}}},
\end{equation}
%
where $u_{i_{2}}$ are the displacement degrees of freedom.
In~\eqref{eq:oelgaard-1:lin_issue_linearised_stiffness}, only
$\sigma_{\alpha}$ depends on~$du_{i_{2}}$, and the linearization of this
terms reads:
%
\begin{equation}
\label{eq:oelgaard-1:lin_issue_stress}
\frac{ d \sigma_{\alpha}}{du_{i_{2}}}
    = \frac{d \sigma_{\alpha}}{d \varepsilon_{\alpha}}
    \frac{d \varepsilon_{\alpha}}{du_{i_{2}}}
    = D_{\alpha} \frac{d \varepsilon_{\alpha}}{du_{i_{2}}},
\end{equation}
%
where $D_{\alpha}$ is the tangent. To compute the values of the strain
at nodes, $\varepsilon_{\alpha}$, from the displacement field, the
derivative of the displacement field is evaluated at~$x_{\alpha}$:
%
\begin{equation}
\label{eq:oelgaard-1:lin_issue_strain}
\varepsilon_{\alpha} = \sum_{i_{2}=1}^3 \phi^{T}_{i_{2},x}\brac{x_{\alpha}}u_{i_{2}}.
\end{equation}
%
Inserting~\eqref{eq:oelgaard-1:lin_issue_stress}
and~\eqref{eq:oelgaard-1:lin_issue_strain}
into~\eqref{eq:oelgaard-1:lin_issue_linearised_stiffness} yields:
%
\begin{equation}
\label{eq:oelgaard-1:lin_issue_k_star}
A^{\star}_{T,i} = \sum_{q=1}^{2}\sum_{\alpha=1}^2 \psi^{T}_{\alpha}(x_{q}) D_{\alpha}
\phi^{T}_{i_{2},x}(x_{\alpha}) \phi^{T}_{i_{1},x}(x_{q}) W_{q}.
\end{equation}
%
This is the exact linearization of~\eqref{eq:oelgaard-1:lin_issue_f}.

We now consider linearization of \eqref{eq:oelgaard-1:lin_issue_F},
which leads to the bilinear form:
%
\begin{equation}
a(u, w) = \int_{\Omega} D u,_{x} w,_{x} \dx,
\end{equation}
%
where $D = d\sigma / d\varepsilon$. If $D$ is represented using a
discontinuous, piecewise linear basis, and two quadrature points are
used to integrate the form (which is exact for this form), the resulting
element matrix is:
%
\begin{equation}
\label{eq:oelgaard-1:lin_issue_k}
A_{T,i} = \sum_{q=1}^{2}\sum_{\alpha=1}^2 \psi^{T}_{\alpha}(x_{q}) D_{\alpha}
\phi^{T}_{i_{2},x}(x_{q}) \phi^{T}_{i_{1},x}(x_{q}) W_{q}.
\end{equation}
%
The above representation is what would be produced by FFC.

Equations~\eqref{eq:oelgaard-1:lin_issue_k_star}
and~\eqref{eq:oelgaard-1:lin_issue_k} are not identical since
$x_{q} \neq x_{\alpha}$. As a consequence, the bilinear form in
\eqref{eq:oelgaard-1:lin_issue_k} is not an exact linearization
of~\eqref{eq:oelgaard-1:lin_issue_F}, and a Newton method will therefore
exhibit suboptimal convergence.  In general, the illustrated problem
arises when some coefficients in a form are computed by a nonlinear
operation elsewhere, and then interpolated and evaluated at a point
that differs from where the coefficients were computed. This situation
is different from the use of nonlinear operators in UFL, and compiled
by FFC. An example of such an operator is the $\ln J$ term in the neo-Hookean
model~\eqref{eq:oelgaard-1:hyp_neo_hookean} where $J$ will be computed
at quadrature points during assembly after which the operator $\ln$ is applied
to compute~$\ln J$.

The linearization issue highlighted in this section is further illustrated
in the following section, as too is a solution that involves the
definition of so-called `quadrature elements'.

%------------------------------------------------------------------------------
\subsection{Quadrature elements}

To introduce the concept of quadrature elements, we first present a
model problem that will be used in numerical examples. Given the finite
element space
%
\begin{equation}
  V = \left\{ w \in H^{1}_{0}(\Omega), \ w \in P_{q}(T) \ \foralls T \ \in \mathcal{T}_h \right\},
\end{equation}
%
where $\Omega \subset \mathbb{R}$ and $q \ge 1$, the model problem of
interest involves: given $f \in V$, find $u \in V$ such that
%
\begin{equation}
  F := \int_{\Omega} \brac{1 + u^2} u_{,x} w_{,x} \dx
            - \int_{\Omega} f w \dx
    = 0 \quad \foralls w \in V.
\label{eq:oelgaard-1:plas_nonlinear_poisson}
\end{equation}
%
Solving this problem via Newton's method involves solving a series of
linear problems with
%
\begin{align}
L\brac{w}
   &:= \int_{\Omega} \brac{1 + u^{2}_{n}} u_{n,x} w_{n,x} \dx
            - \int_{\Omega} f w \dx,\\
a\brac{du_{n+1}, w}
   &:= \int_{\Omega} \brac{1 + u^{2}_{n}} du_{n+1, x} w_{,x} \dx
        + \int_{\Omega} 2u_{n}u_{n,x} du_{n+1} w_{,x} \dx,
\label{eq:oelgaard-1:plas_nonlinear_poisson_forms}
\end{align}
%
with the update $u_{n} \leftarrow u_{n} - du_{n+1}$.  To draw an analogy
with complex constitutive models, we rephrase the above as:
%
\begin{align}
L\brac{w}
   &:= \int_{\Omega} \sigma_{n} w_{,x} \dx - \int_{\Omega} f w \dx,
\\
a\brac{du_{n+1}, w}
   &:=  \int_{\Omega} C_{n} du_{n+1, x} w_{,x} \dx
      + \int_{\Omega} 2u_{n}u_{n,x} du_{n+1} w_{,x} \dx,
\label{eq:oelgaard-1:plas_nonlinear_poisson_forms2}
\end{align}
%
where $\sigma_{n} = \brac{1 + u^{2}_{n}} u_{n, x}$ and $C_{n} = \brac{1
+ u^{2}_{n}}$.  The forms now resemble those for a plasticity problem
where, $\sigma$ is the `stress', $C$ is the `tangent' and $u_{,x}$ is the
`strain'.

Similar to a plasticity problem, we wish to compute $\sigma$ and $C$
`off-line', and to supply $\sigma$ and $C$ as functions in a space $Q$
to the forms used in the Newton solution process. To access $u_{n ,x}$
for use off-line, an approach is to perform an $L^{2}$-projection of
the derivative of $u$ onto a space~$Q$.  For the example in question,
we will also project $1 + u^{2}$ onto $Q$.  A natural choice would be
to make $Q$ one polynomial order less that $V$ and discontinuous across
cell facets.  However, following this approach leads to a convergence
rate for a Newton solver that is less than the expected quadratic rate.
The reason for this is that the linearization that follows from this
process is not consistent with the problem being solved as explained in the
previous section.

To resolve this issue within the context of \ufl and \ffc, the concept
of \emph{quadrature elements} \index{quadrature element} has been developed.
This special type of element is used to represent `functions' that can only
be evaluated at particular points (quadrature points), and cannot be
differentiated. In the remainder of this section we present some key features
of the quadrature element and a demonstration of its use for the model problem
considered above.  A quadrature element is declared in UFL by:
%
\begin{python}
element = FiniteElement("Quadrature", tetrahedron, q)
\end{python}
%
where \emp{q} is the polynomial degree that the underlying quadrature
rule will be able to integrate exactly.  The declaration of a quadrature
element is similar to the declaration of any other element in \ufl
and it can be used as such, with some limitations. Note, however,
the subtle difference that the element order does not refer to the
polynomial degree of the finite element shape functions, but instead
relates to the quadrature scheme.
For `sufficient' integration of a second-order
polynomial in three dimensions, \ffc{} will use four quadrature points
per cell.  \ffc{}
interprets the quadrature points of the quadrature element as degrees of
freedom where the value of a shape function for a degree of freedom is
equal to one at the quadrature point and zero otherwise.  This has the
implication that a function that is defined on a quadrature element can
only be evaluated at quadrature points.  Furthermore, it is not possible
to take derivatives of functions defined on a quadrature element.

Before demonstrating the importance of quadrature elements when computing
terms off-line, we illustrate a simple usage of a quadrature element.
Consider the bilinear form for a mass matrix weighted by a coefficient
$f$ that is defined on a quadrature element:
%
\begin{equation}
  a\brac{u, w} = \int_{\Omega} f u w \dx.
\label{eq:oelgaard-1:quad_elem_mass_form}
\end{equation}
%
If the test and trial functions $w$ and $u$ come from a space of linear
Lagrange functions, the polynomial degree of their product is two.
This means that the coefficient $f$ should be defined as:
%
\begin{python}
ElementQ = FiniteElement("Quadrature", tetrahedron, 2)
f = Coefficient(ElementQ)
\end{python}
%
to ensure appropriate integration of the form
in~\eqref{eq:oelgaard-1:quad_elem_mass_form}.  The reason for this is
that the quadrature element in the form dictates the quadrature scheme
that \ffc{} will use for the numerical integration since the quadrature
element, as described above, only have nonzero values at points that
coincide with the underlying quadrature scheme of the quadrature element.
Thus, if the degree of \emp{ElementQ} is set to one, the form will be
integrated using only one integration point, since one point is enough
to integrate a linear polynomial exactly, and as a result the form
is under integrated.  If quadratic Lagrange elements are used for $w$
and $u$, the polynomial degree of the integrand is four, therefore the
declaration for the coefficient $f$ should be changed to:
%
\begin{python}
ElementQ = FiniteElement("Quadrature", tetrahedron, 4)
f = Coefficient(ElementQ)
\end{python}

The DOLFIN code for solving the nonlinear model problem with a
source term $f = x^2 - 4$, and Dirichlet boundary conditions
$u = 1$ at $x = 0$, continuous quadratic elements for $V$,
and quadrature elements of degree two for $Q$ is shown in
Figure~\ref{fig:oelgaard-1:plas_nonlinear_poisson_code}.  The
relative residual norm after each iteration of the Newton solver
for four different combinations of spaces $V$ and $Q$ is shown
in Table~\ref{tab:oelgaard-1:plas_nonlinear_poisson_convergence}.
Continuous, discontinuous and quadrature elements are denoted by $CG_{q}$,
$DG_{q}$ and $Q_{q}$ respectively where $q$ refers to the polynomial
degree as discussed previously.  It is clear from the table that using
quadratic elements for $V$ requires the use of quadrature elements in
order to ensure quadratic convergence of the Newton solver.

\begin{figure}
\bwfig
\begin{python}
from dolfin import *

# Sub domain for Dirichlet boundary condition
class DirichletBoundary(SubDomain):
    def inside(self, x, on_boundary):
        return abs(x[0] - 0.0) < DOLFIN_EPS and on_boundary

# Class for interfacing with the Newton solver
class NonlinearModelProblem(NonlinearProblem):
    def __init__(self, a, L, u, C, S, Q, bc):
        NonlinearProblem.__init__(self)
        self.a, self.L = a, L
        self.u, self.C, self.S, self.Q, self.bc = u, C, S, Q, bc

    def F(self, b, x):
        assemble(self.L, tensor=b)
        self.bc.apply(b, x)

    def J(self, A, x):
        assemble(self.a, tensor=A)
        self.bc.apply(A)

    def form(self, A, b, x):
        C = project((1.0 + self.u**2), self.Q)
        self.C.vector()[:] = C.vector()

        S = project(Dx(self.u, 0), self.Q)
        self.S.vector()[:] = S.vector()
        self.S.vector()[:] = self.S.vector()*self.C.vector()

# Create mesh and define function spaces
mesh = UnitInterval(8)
V = FunctionSpace(mesh, "Lagrange", 2)
Q = FunctionSpace(mesh, "Q", 2)

# Define boundary condition
bc = DirichletBC(V, Constant(1.0), DirichletBoundary())

# Define source and functions
f = Expression("x[0]*x[0] - 4")
u, C, S = Function(V), Function(Q), Function(Q)

# Define variational problems
w  = TestFunction(V)
du = TrialFunction(V)
L  = S*Dx(w, 0)*dx - f*w*dx
a  = C*Dx(du, 0)*Dx(w, 0)*dx + 2*u*Dx(u, 0)*du*Dx(w, 0)*dx

# Create nonlinear problem, solver and solve
problem = NonlinearModelProblem(a, L, u, C, S, Q, bc)
solver  = NewtonSolver()
solver.solve(problem, u.vector())
\end{python}
\caption{DOLFIN implementation for the nonlinear model problem
in~\eqref{eq:oelgaard-1:plas_nonlinear_poisson} with `off-line'
computation of terms used in the variational forms.}
\label{fig:oelgaard-1:plas_nonlinear_poisson_code}
\end{figure}

\begin{table}
  \caption{Computed relative residual norms after each iteration of
    the Newton solver for the nonlinear model problem using different
    elements for $V$ and~$Q$. Quadratic convergence is observed when
    using quadrature elements, and when using piecewise constant
    functions for $Q$, which coincides with a one-point quadrature
    element. The presented results are computed using the code in
    Figure~\ref{fig:oelgaard-1:plas_nonlinear_poisson_code}.}
  \label{tab:oelgaard-1:plas_nonlinear_poisson_convergence}
  \centering
  \begin{tabular}{crrrr}
    \toprule
    \multicolumn{1}{c}{Iteration} & \multicolumn{1}{c}{$CG_{1}$/$DG_{0}$}
    & \multicolumn{1}{c}{$CG_{1}$/$Q_{1}$}
    & \multicolumn{1}{c}{$CG_{2}$/$DG_{1}$}
    & \multicolumn{1}{c}{$CG_{2}$/$Q_{2}$}\\
    \midrule
    1         & 1.114e+00 & 1.101e+00 & 1.398e+00 & 1.388e+00\\
    2         & 2.161e-01 & 2.319e-01 & 2.979e-01 & 2.691e-01\\
    3         & 3.206e-03 & 3.908e-03 & 2.300e-02 & 6.119e-03\\
    4         & 7.918e-07 & 7.843e-07 & 1.187e-03 & 1.490e-06\\
    5         & 9.696e-14 & 3.662e-14 & 2.656e-05 & 1.242e-13\\
    6         &           &           & 5.888e-07 & \\
    7         &           &           & 1.317e-08 & \\
    8         &           &           & 2.963e-10 & \\
    \bottomrule
  \end{tabular}
\end{table}

%------------------------------------------------------------------------------
\section{Implementation and examples}
\label{sec:oelgaard-1:examples}

We present in this section implementation examples that correspond
to the afore presented models. Where feasible, complete solvers are
presented. When this is not feasible, relevant code extracts are
presented.  Python examples are preferred due the compactness of the
code extracts, however, in the case of plasticity efficiency demands
a C++ implementation.  It is possible in the future that an efficient
Python interface for plasticity problems will be made available via
just-in-time compilation.

The examples are chosen to highlight some implementation aspects that
are typical for solid mechanics applications. In the code extracts, we
do not provide commentary on generic aspects, such as the creation of
meshes, application of boundary conditions and the solution of linear
systems. For an explanation of such aspects in the code examples, we
refer to Chapters~\ref{chap:langtangen} and~\ref{chap:logg-2}.

%------------------------------------------------------------------------------
\subsection{Linearized elasticity}

This example is particularly simple since the stress can be expressed as
a straightforward function of the displacement field, and the expression
for the stress in~\eqref{eq:oelgaard-1:elas_stress_def} can be inserted
directly into~\eqref{eqn:oelgaard-1:F}.  For the steady case (inertia
terms are ignored), a complete solver for a linearized elasticity
problem is presented in Figure~\ref{fig:oelgaard-1:elas_code}.
The solver in Figure~\ref{fig:oelgaard-1:elas_code} is for a
simulation on a unit cube with a source term $b = (1, 0, 0)$ and
$u = 0$ on $\partial \Omega$.  A continuous, piecewise quadratic
finite element space is used.  The expressiveness of the UFL input
means that the expressions for \emp{sigma} and \emp{F} in
Figure~\ref{fig:oelgaard-1:elas_code} resemble closely the mathematical
expressions used in the text for $\sigma$ and $F$. We have presented
this problem in Figure~\ref{fig:oelgaard-1:elas_code} in terms of $F$
to unify our presentation of linear and nonlinear equations, and used
the UFL functions \emp{lhs} and \emp{rhs} to automatically extract the
bilinear and linear forms, respectively, from~\emp{F}.

\begin{figure}
\bwfig
\begin{python}
from dolfin import *

# Create mesh
mesh = UnitCube(8, 8, 8)

# Create function space
V = VectorFunctionSpace(mesh, "Lagrange", 2)

# Create test and trial functions, and source term
u, w = TrialFunction(V), TestFunction(V)
b = Constant((1.0, 0.0, 0.0))

# Elasticity parameters
E, nu = 10.0, 0.3
mu, lmbda = E/(2.0*(1.0 + nu)), E*nu/((1.0 + nu)*(1.0 - 2.0*nu))

# Stress
sigma = 2*mu*sym(grad(u)) + lmbda*tr(grad(u))*Identity(w.cell().d)

# Governing balance equation
F = inner(sigma, grad(w))*dx - dot(b, w)*dx

# Extract bilinear and linear forms from F
a, L = lhs(F), rhs(F)

# Dirichlet boundary condition on entire boundary
c  = Constant((0.0, 0.0, 0.0))
bc = DirichletBC(V, c, DomainBoundary())

# Set up PDE and solve
u = Function(V)
problem = LinearVariationalProblem(a, L, u, bcs=bc)
solver = LinearVariationalSolver(problem)
solver.parameters["symmetric"] = True
solver.solve()
\end{python}
\caption{DOLFIN Python solver for a simple linearized elasticity problem
on a unit cube.}
\label{fig:oelgaard-1:elas_code}
\end{figure}

%------------------------------------------------------------------------------
\subsection{Plasticity}
\label{sec:oelgaard-1:examples_plasticity}

The computation of the stress tensor, and its linearization,
for the model outlined in Section~\ref{sec:oelgaard-1:plasticity}
in a displacement-driven finite element model is rather involved.
A method of computing point-wise a stress tensor that satisfies
\eqref{eq:oelgaard-1:plas_yield_function} from the strain, strain
increment and history variables is known as a `return mapping
algorithm'.  Return mapping strategies are discussed in detail in
\citet{SimoHughes1998}.  A widely used return mapping approach, the
`closest-point projection', is summarized below for a plasticity model
with linear isotropic hardening.

From~\eqref{eq:oelgaard-1:plas_stress_def}
and~\eqref{eq:oelgaard-1:plas_strain_decomposed} the stress at the end
of a strain increment reads:
%
\begin{equation}
\label{eq:oelgaard-1:rm_stress_strain}
\sigma_{n+1} = \mathcal{C} : (\varepsilon_{n+1} - \varepsilon^{\rm p}_{n+1}).
\end{equation}
%
Therefore, given $\varepsilon_{n+1}$, it is necessary to determine the
plastic strain $\varepsilon^{\rm p}_{n+1}$ in order to compute the stress.
In a closest-point projection method the increment in plastic strain
is computed from:
%
\begin{equation}
\label{eq:oelgaard-1:rm_plas_strain}
\varepsilon^{\rm p}_{n+1} - \varepsilon^{\rm p}_{n} =
  \Delta \lambda \frac{\partial g \brac{\sigma_{n+1}}}{\partial \sigma},
\end{equation}
%
where $g$ is the plastic potential function and $\Delta \lambda
= \lambda_{n+1} - \lambda_{n}$.  Since $\partial_{\sigma} g$ is
evaluated at $\sigma_{n+1}$, \eqref{eq:oelgaard-1:rm_stress_strain}
and \eqref{eq:oelgaard-1:rm_plas_strain} constitute as system of coupled
equations with unknowns $\Delta \lambda$ and $\sigma_{n+1}$. In general,
the system is nonlinear.  To obtain a solution, Newton's method is
employed as follows, with $k$ denoting the iteration number.  First, a
`trial stress' is computed:
%
\begin{equation}
\label{eq:oelgaard-1:rm_trial_stress}
\sigma_{\rm trial} = \mathcal{C} : (\varepsilon_{n+1} - \varepsilon^{\rm p}_{n}).
\end{equation}
%
Subtracting \eqref{eq:oelgaard-1:rm_trial_stress}
from~\eqref{eq:oelgaard-1:rm_stress_strain} and
inserting~\eqref{eq:oelgaard-1:rm_plas_strain}, the following equation is
obtained:
%
\begin{equation}
\label{eq:oelgaard-1:rm_residual_stress}
R_{n+1} := \sigma_{n+1} - \sigma_{\rm trial} + \Delta \lambda \mathcal{C} :
                \frac{\partial g \brac{\sigma_{n+1}}}{\partial \sigma} = 0,
\end{equation}
%
where $R_{n+1}$ is the `stress residual'.  During the Newton
iterations this residual is driven towards zero.  If the trial stress
in~\eqref{eq:oelgaard-1:rm_trial_stress} leads to satisfaction of
the yield criterion in~\eqref{eq:oelgaard-1:plas_yield_function},
then $\sigma_{\rm trial}$ is the new stress and the Newton procedure
is terminated.  Otherwise, the Newton increment of $\Delta \lambda$
is computed from:
%
\begin{equation}
\label{eq:oelgaard-1:rm_dlambda}
d\lambda_{k} = \frac{f_{k} - R_{k} : Q_{k} : \partial_{\sigma} f_{k}}
{\partial_{\sigma} f_{k} : \Xi_{k} : \partial_{\sigma} g_{k} + h},
\end{equation}
%
where $Q = \left[I + \Delta \lambda \mathcal{C} : \partial^{2}_{\sigma
\sigma} g\right]^{-1}$, $\Xi = Q : \mathcal{C}$ and $h$ is a hardening
parameter, which for the von Mises model with linear hardening is equal
to $H$ (the constant hardening parameter).  The stress increment is
computed from:
%
\begin{equation}
\label{eq:oelgaard-1:rm_dsigma}
\Delta \sigma_{k} = \left[-d\lambda_{k} \mathcal{C} : \partial_{\sigma} g_{k}
- R_{k}\right] : Q_{k},
\end{equation}
%
after which the increment of the plastic multiplier and the stresses for the
next iteration can be computed:
%
\begin{align}
\label{eq:oelgaard-1:rm_update_values}
  \Delta \lambda_{k+1} &= \Delta \lambda_{k} + d \lambda_{k},
\\
  \sigma_{k+1} &= \sigma_{k} + \Delta \sigma_{k}.
\end{align}
%
The yield criterion is then evaluated again using the updated values,
and the procedure continues until the yield criterion is satisfied to
within a prescribed tolerance.  Note that to start the procedure $\Delta
\lambda_{0} = 0$ and $\sigma_{0} = \sigma_{\rm trial}$.  After convergence
is achieved, the consistent tangent can be computed:
%
\begin{equation}
\label{eq:oelgaard-1:rm_consistent_tangent}
C_{\rm tan} = \Xi - \frac{\Xi : \partial_{\sigma} g \otimes \partial_{\sigma} f : \Xi}
{\partial_{\sigma} f : \Xi : \partial_{\sigma} g + h},
\end{equation}
%
which is used when assembling the global Jacobian (stiffness matrix).
The return mapping algorithm is applied at all quadrature points.

The closest-point return mapping algorithm described above is common
to a range of plasticity models that are defined by the form of the
functions $f$ and~$g$. The process can be generalized for models with
more complicated hardening behavior.  To aid the implementation of
different models, a return mapping algorithm and support for quadrature
point level history parameters is provided by the FEniCS Plasticity
library (\url{https://launchpad.net/fenics-plasticity/}).  The library
adopts a polymorphic design, with the base class \emp{PlasticityModel}
providing an interface for users to implement, and thereby supply
functions for
$f$,
$\partial_{\sigma} f$,
$\partial_{\sigma} g$,
and $\partial_{\sigma \sigma} g$.
Figure~\ref{fig:oelgaard-1:plasticity_interface} shows the
\emp{PlasticityModel} class public interface.  Supplied with details
of $f$ (and possibly $g$), the library can compute stress updates and
linearizations using the closest-point projection method.

\begin{figure}
\bwfig
\begin{c++}
class PlasticityModel
{
public:

  /// Constructor
  PlasticityModel(double E, double nu);

  /// Return hardening parameter
  virtual double hardening_parameter(double eps_eq) const;

  /// Equivalent plastic strain
  virtual double kappa(double eps_eq, const arma::vec& stress,
                       double lambda_dot) const;

  /// Value of yield function f
  virtual double f(const arma::vec& stress,
                   double equivalent_plastic_strain) const = 0;

  /// First derivative of f with respect to sigma
  virtual void df(arma::vec& df_dsigma,
                  const arma::vec& stress) const = 0;

  /// First derivative of g with respect to sigma
  virtual void dg(arma::vec& dg_dsigma,
                  const arma::vec& stress) const;

  /// Second derivative of g with respect to sigma
  virtual void ddg(arma::mat& ddg_ddsigma,
                   const arma::vec& stress) const = 0;

};
\end{c++}
\caption{\emp{PlasticityModel} public interface defined by the plasticity
library. Users are required to supply implementations for at least the
pure virtual functions. These functions describe the plasticity model.}
\label{fig:oelgaard-1:plasticity_interface}
\end{figure}

Computational efficiency is important in the return mapping algorithm
as the stress and its linearization are computed at all quadrature
points at each global Newton iteration. Therefore, it is necessary to
execute the algorithm in C++ rather than in Python. For this reason, the
FEniCS Plasticity library provides a C++ interface only at this stage.
To reconcile ease and efficiency, it would be possible to use just-in-time
compilation for a Python implementation of the \emp{PlasticityModel}
interface, just as DOLFIN presently does for the \emp{Expression} class
(see Chapter~\ref{chap:logg-2}).

We now outline a solver based on the FEniCS Plasticity library.
Firstly, the \ufl{} input for a formulation in three dimensions
using a continuous, piecewise quadratic basis is shown in
Figure~\ref{fig:oelgaard-1:plas_ufl_code}. Note that the stress
and the linearized tangent are supplied as coefficients to the
form as they are computed inside the plasticity library.  Symmetry
has been exploited to flatten the stress and the tangent terms.
Note also in Figure~\ref{fig:oelgaard-1:plas_ufl_code} that quadrature
elements are used for the coefficients \emp{s} and \emp{t}.  Recall from
Section~\ref{sec:oelgaard-1:linearization} that when constitutive updates
are computed outside of the form file care must be taken to ensure
quadratic convergence of a Newton method.  By using quadrature elements
in Figure~\ref{fig:oelgaard-1:plas_ufl_code}, it is possible to achieve
quadratic convergence during a Newton solve for plasticity problems.

\begin{figure}
\bwfig
\begin{python}
element  = VectorElement("Lagrange", tetrahedron, 2)
elementT = VectorElement("Quadrature", tetrahedron, 2, 36)
elementS = VectorElement("Quadrature", tetrahedron, 2, 6)

u, w =  TrialFunction(element), TestFunction(element)
b, h = Coefficient(element), Coefficient(element)
t, s = Coefficient(elementT), Coefficient(elementS)

def eps(u):
	return as_vector([u[i].dx(i) for i in range(3)] \
+ [u[i].dx(j) + u[j].dx(i) for i, j in [(0, 1), (0, 2), (1, 2)]])

def sigma(s):
	return as_matrix([[s[0], s[3], s[4]],
                    [s[3], s[1], s[5]],
                    [s[4], s[5], s[2]]])

def tangent(t):
  return as_matrix([[t[i*6 + j] for j in range(6)] for i in range(6)])

a = inner(dot(tangent(t), eps(u)), eps(w))*dx
L = inner(sigma(s), grad(w))*dx - dot(b, w)*dx - dot(h, w)*ds
\end{python}
\caption{Definition of the linear and bilinear variational forms for
plasticity expressed using \ufl syntax.}
\label{fig:oelgaard-1:plas_ufl_code}
\end{figure}

The solver is implemented in C++, and
Figure~\ref{fig:oelgaard-1:plas_cpp_code} shows an extract of the
most relevant parts of the solver in the context of plasticity.
First, the necessary function spaces are created. \emp{V} is used to define
the bilinear and linear forms and the displacement field $u$, while \emp{Vt}
and \emp{Vs} are used for the two coefficient spaces: the consistent tangent
and the stress, which enter the bilinear and linear forms of the plasticity
problem.
The forms defining the plasticity problem are then created and the relevant
functions are attached to the forms.
Then the object defining the plasticity model is created.  The class
\emp{VonMises} is a subclass of the \emp{PlasticityModel} class shown
in Figure~\ref{fig:oelgaard-1:plasticity_interface} and it implements
functions for $f$, $\partial_{\sigma} f$ and $\partial_{\sigma\sigma} g$.
It is constructed with values for the Young's modulus, Poisson's ratio,
yield stress and linear hardening parameter.  This object can then be
passed to the constructor of the \emp{PlasticityProblem} class along
with the forms, displacement field $u$, coefficient functions
and boundary conditions.  \emp{PlasticityProblem} is a subclass
of the DOLFIN class \emp{NonlinearProblem}, which is described in
Chapter~\ref{chap:logg-2}. The \emp{PlasticityProblem} class handles
the assembly over cells, loops over cell quadrature points, and variable
updates.  The \emp{PlasticityProblem} is solved by the \emp{NewtonSolver}
like any other \emp{NonlinearProblem} object. After each Newton solver
the history variables are updated by calling the \emp{update\_variables},
function before proceeding with the next solution increment.

\begin{figure}
\bwfig
\begin{c++}
// Create mesh and define function spaces
UnitCube mesh(4, 4, 4);
Plasticity::FunctionSpace V(mesh);
Plasticity::BilinearForm::CoefficientSpace_t Vt(mesh);
Plasticity::LinearForm::CoefficientSpace_s Vs(mesh);

// Create forms and attach functions
Function tangent(Vt);
Plasticity::BilinearForm a(V, V);
a.t = tangent;

Function stress(Vs);
Plasticity::LinearForm L(V);
L.s = stress;

// Displacements
Function u(V);

// Young's modulus and Poisson's ratio
double E = 20000.0; double nu = 0.3;

// Slope of hardening (linear) and hardening parameter
double E_t(0.1*E);
double hardening_parameter = E_t/(1.0 - E_t/E);

// Yield stress
double yield_stress = 200.0;

// Object of class von Mises
fenicsplas::VonMises J2(E, nu, yield_stress, hardening_parameter);

// Create PlasticityProblem
fenicsplas::PlasticityProblem nonlinear_problem(a, L, u, tangent, stress, bcs, J2);

// Create nonlinear solver
NewtonSolver nonlinear_solver;

// Pseudo time stepping parameters
double t = 0.0; double dt  = 0.005; double T = 0.02;

// Apply load in steps
while (t < T)
{
  // Increment time and solve nonlinear problem
  t += dt;
  nonlinear_solver.solve(nonlinear_problem, u.vector());

  // Update variables for next load step
  nonlinear_problem.update_variables();
}
\end{c++}
\caption{Code extract for solving a plasticity problem.}
\label{fig:oelgaard-1:plas_cpp_code}
\end{figure}

%------------------------------------------------------------------------------
\subsection{Hyperelasticity}
\label{sec:oelgaard-1:examples_hyp_elas}

We present the construction of a solver for a hyperelastic problem that is
phrased as a minimization problem, following the minimization framework
that was presented in Section~\ref{sec:oelgaard-1:minimization}. The
compressible neo-Hookean model in~\eqref{eq:oelgaard-1:hyp_neo_hookean}
is adopted.  The automatic functional differentiation features of UFL
permit the solver code to resemble closely the abstract mathematical
presentation. Noteworthy in this approach is that it is not necessary to
provide an explicit expression for the stress tensor.  Changing model is
therefore as simple as redefining the stored energy density function~$\Psi_{0}$.

A complete hyperelastic solver is presented
in Figure~\ref{fig:oelgaard-1:hyperelastic_solver_code}.
It corresponds to a problem posed on a unit cube, and loaded by
a body force $b_{0} = (0, -0.5, 0)$, and restrained such that $u =
(0, 0, 0)$ where $x = 0$. Elsewhere on the boundary the traction
$h_{0} = (0.1, 0, 0)$ is applied.  Continuous, piecewise linear
functions for the displacement field are used. The code in
Figure~\ref{fig:oelgaard-1:hyperelastic_solver_code} adopts the
same notation used in Sections~\ref{sec:oelgaard-1:minimization}
and~\ref{sec:oelgaard-1:hyperelasticity}. The problem is posed on the
reference domain, and for convenience the subscripts `$0$' have been
dropped in the code.

\begin{figure}
\bwfig
\begin{python}
from dolfin import *

# Optimization options for the form compiler
parameters["form_compiler"]["cpp_optimize"] = True

# Create mesh and define function space
mesh = UnitCube(16, 16, 16)
V = VectorFunctionSpace(mesh, "Lagrange", 1)

def left(x):
    return x[0] < DOLFIN_EPS

# Define Dirichlet boundary (x = 0 or x = 1)
zero = Constant((0.0, 0.0, 0.0))
bc = DirichletBC(V, zero, left)

# Define test and trial functions
du, w = TrialFunction(V), TestFunction(V)

# Displacement from previous iteration
u  = Function(V)
b  = Constant((0.0, -0.5, 0.0))  # Body force per unit mass
h  = Constant((0.1,  0.0, 0.0))  # Traction force on the boundary

# Kinematics
I = Identity(V.cell().d)    # Identity tensor
F = I + grad(u)             # Deformation gradient
C = F.T*F                   # Right Cauchy-Green tensor

# Invariants of deformation tensors
Ic, J = tr(C), det(F)

# Elasticity parameters
E, nu = 10.0, 0.3
mu, lmbda = E/(2*(1 + nu)), E*nu/((1 + nu)*(1 - 2*nu))

# Stored strain energy density (compressible neo-Hookean model)
Psi = (mu/2)*(Ic - 3) - mu*ln(J) + (lmbda/2)*(ln(J))**2

# Total potential energy
Pi = Psi*dx - dot(b, u)*dx - dot(h, u)*ds

# Compute first variation of Pi (directional derivative about u in the direction of v)
F = derivative(Pi, u, w)

# Compute Jacobian of F
dF = derivative(F, u, du)

# Create nonlinear variational problem and solve
problem = NonlinearVariationalProblem(F, u, bcs=bc, J=dF)
solver = NonlinearVariationalSolver(problem)
solver.solve()

# Save solution in VTK format
file = File("displacement.pvd");
file << u;
\end{python}
\caption{Complete DOLFIN solver for the compressible neo-Hookean model,
formulated as a minimization problem.}
\label{fig:oelgaard-1:hyperelastic_solver_code}
\end{figure}

The solver in Figure~\ref{fig:oelgaard-1:hyperelastic_solver_code}
solves the problem using one Newton step. For problems with stronger
nonlinearities, perhaps as a result of greater volumetric or surface
forcing terms, it may be necessary to apply a pseudo time-stepping
approach and solve the problem in number of Newton increments, or it
may be necessary to apply a path following solution method.

%------------------------------------------------------------------------------
\subsection{Elastodynamics}

We present now a linearized elastodynamics problem to
illustrate the solution of time-dependent problems. The
example is based on the Newmark family of methods presented in
Section~\ref{sec:oelgaard-1:time_integration}.  For this example, we
consider a viscoelastic model that is a minor extension of the elasticity
model in~\eqref{eq:oelgaard-1:elas_stress_def}.  For this model, the
stress tensor is given by:
%%
\begin{equation}
  \sigma = 2\mu \varepsilon
   + \left( \lambda {\rm tr}(\varepsilon) + \eta {\rm tr}(\Dot{\varepsilon}) \right) I,
\label{eq:oelgaard-1:viscoelastic}
\end{equation}
%%
where the constant scalar $\eta \ge 0$ is a viscosity parameter.

A simple, but complete, elastodynamics solver is
presented in Figures~\ref{fig:oelgaard-1:time_ufl_code}
and~\ref{fig:oelgaard-1:time_ufl_code2}.  The solver mirrors the notation
used in Section~\ref{sec:oelgaard-1:time_integration}, with expressions
for the acceleration, velocity and displacement at time $t_{n}$
(\emp{a0}, \emp{v0}, \emp{u0}), and expressions for the acceleration
and velocity at time $t_{n+1}$ (\emp{a1}, \emp{v1}) in terms of the
displacement at $t_{n+1}$ (\emp{u1}) and other fields at time $t_{n}$.
For simplicity, the source term $b = (0, 0, 0)$.  The body is fixed such
that $u = (0, 0, 0)$ at $x=0$ and the initial conditions are
$u_{0} = v_{0} = (0, 0, 0)$.
A traction $h$ is applied at $x=1$ and is increased linearly from zero
to one over the first five time steps.  Therefore, no forces are acting
on the body at $t=0$ and the initial acceleration is zero.  Again, the
\ufl{} functions \emp{lhs} and \emp{rhs} have been used to extract the
bilinear and linear terms from the form.  This is particularly convenient
for time-dependent problems since it allows the code implementation to
be posed in the same format as is usually adopted in the mathematical
presentation, with the equation of interest posed in terms of fields at
some point between times $t_{n}$ and~$t_{n+1}$.  The presented solver
could be made more efficient by exploiting linearity of the governing
equation and thereby re-using the factorization of the system matrix.

\begin{figure}
\bwfig
\begin{python}
from dolfin import *

# Form compiler options
parameters["form_compiler"]["cpp_optimize"] = True
parameters["form_compiler"]["optimize"] = True

# External load
class Traction(Expression):
    def __init__(self, end):
        Expression.__init__(self)
        self.t = 0.0
        self.end = end

    def eval(self, values, x):
        values[0] = 0.0
        values[1] = 0.0
        if x[0] > 1.0 - DOLFIN_EPS:
            values[0] = self.t/self.end if self.t < self.end else 1.0

    def value_shape(self):
        return (2,)

def update(u, u0, v0, a0, beta, gamma, dt):
    # Get vectors (references)
    u_vec, u0_vec  = u.vector(), u0.vector()
    v0_vec, a0_vec = v0.vector(), a0.vector()

    # Update acceleration and velocity
    a_vec = (1.0/(2.0*beta))*( (u_vec - u0_vec - v0_vec*dt)/(0.5*dt*dt) - (1.0-2.0*beta)*a0_vec )

    # v = dt * ((1-gamma)*a0 + gamma*a) + v0
    v_vec = dt*((1.0-gamma)*a0_vec + gamma*a_vec) + v0_vec

    # Update (t(n) <-- t(n+1))
    v0.vector()[:], a0.vector()[:] = v_vec, a_vec
    u0.vector()[:] = u.vector()

# Load mesh and define function space
mesh = UnitSquare(32, 32)

# Define function space
V = VectorFunctionSpace(mesh, "Lagrange", 1)

# Test and trial functions
u1, w = TrialFunction(V), TestFunction(V)

E, nu  = 10.0, 0.3
mu, lmbda = E/(2.0*(1.0 + nu)), E*nu/((1.0 + nu)*(1.0 - 2.0*nu))

# Mass density and viscous damping coefficient
rho, eta = 1.0, 0.2

# Time stepping parameters
beta, gamma = 0.25, 0.5
dt   = 0.1
t, T = 0.0, 20*dt

# Fields from previous time step (displacement, velocity, acceleration)
u0, v0, a0 = Function(V), Function(V), Function(V)
h = Traction(T/4.0)
\end{python}
\caption{Python code for solving for a dynamic
problem using an implicit Newmark scheme. Program continues in
Figure~\ref{fig:oelgaard-1:time_ufl_code2}.}
\label{fig:oelgaard-1:time_ufl_code}
\end{figure}

\begin{figure}
\bwfig
\begin{python}
# Velocity and acceleration at t_(n+1)
v1 = (gamma/(beta*dt))*(u1 - u0) - (gamma/beta - 1.0)*v0 - dt*(gamma/(2.0*beta) - 1.0)*a0
a1 = (1.0/(beta*dt**2))*(u1 - u0 - dt*v0) - (1.0/(2.0*beta) - 1.0)*a0

# Stress tensor
def sigma(u, v):
    return 2.0*mu*sym(grad(u)) + (lmbda*tr(grad(u)) + eta*tr(grad(v)))*Identity(u.cell().d)

# Governing equation
F = (rho*dot(a1, w) + inner(sigma(u1, v1), sym(grad(w))))*dx - dot(h, w)*ds

# Extract bilinear and linear forms
a, L = lhs(F), rhs(F)

# Set up boundary condition at left end
zero = Constant((0.0, 0.0))
def left(x):
    return x[0] < DOLFIN_EPS
bc = DirichletBC(V, zero, left)

# Set up PDE, advance in time and solve
u = Function(V)
problem = LinearVariationalProblem(a, L, u, bcs=bc)
solver = LinearVariationalSolver(problem)
# Save solution in VTK format
file = File("displacement.pvd")
while t <= T:
    t += dt
    h.t = t
    solver.solve()
    update(u, u0, v0, a0, beta, gamma, dt)
    file << u
\end{python}
\caption{Continuation of Python code extract for solving for a dynamic
problem.}
\label{fig:oelgaard-1:time_ufl_code2}
\end{figure}

%------------------------------------------------------------------------------
\section{Future developments}

In this chapter we have presented a range of solid mechanics problems in
the context of automated modeling.  The implementation of the models
was shown to be relatively straightforward due to the high level of
abstraction provided in the \fenics framework.  The presented cases
cover a range of typical solid mechanics problems that can currently be
solved using \fenics tools.  To broaden the range of problems that can
be handled in the \fenics framework the following two extensions are of
particular interest from a solid mechanics viewpoint:
%%
\begin{description}
\item [Assembly of forms on manifolds] Currently, it is assumed that
two-dimensional elements, like triangles, are embedded in $\R^2$ and
three-dimensional elements, like tetrahedra, are embedded in $\R^3$.
An improvement would be to support two-dimensional elements embedded
in $\R^3$ and one-dimensional elements embedded in $\R^2$ or $\R^3$.
This would, among other things, provide support for shell and truss
problems within the automated framework.

\item [Isoparametric elements] This issue relates to quadrilateral
and hexahedral elements, which are currently not supported, and to
elements with higher order mappings that allow curved mesh boundaries
to be represented.
\enlargethispage{12pt}
\end{description}

%------------------------------------------------------------------------------
