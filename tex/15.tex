\fenicschapter{SyFi and SFC: symbolic finite elements and form compilation}
              {SyFi and SFC: symbolic finite elements and \\ form compilation}
              {SyFi and SFC: symbolic finite elements and form compilation}
              {Martin Sandve Aln\ae{}s and Kent-Andre Mardal}
              {alnes-3}

\renewcommand{\AA}{A}
\newcommand{\LL}{L}
\newcommand{\JJ}{J}
\newcommand{\syfi}{SyFi}

\index{SyFi}
\index{SFC}
\index{form compiler}

This chapter concerns the finite element library \syfi{} and its form
compiler \sfc{}.  \syfi{} is a framework for defining finite elements
symbolically, using the C++ library
GiNaC~\citep{BauerFrinkKreckel2000} and its Python interface
Swiginac~\citep{SkavhaugCertik2009}. In many respects, \syfi{} is the
equivalent of \fiat (see Chapter~\ref{chap:kirby-2}), whereas \sfc{}
corresponds to \ffc{} (see Chapter~\ref{chap:logg-1}). \syfi{}
and \sfc{} come with an extensive manual \citep{AlnaesMardal2009a}
which can be found on the \fenics{} web page. \sfc{} can be used in
\fenics{} as a form compiler. Similarly to \ffc{}, it translates
\ufl{} code (see Chapter~\ref{chap:alnes-1}) into UFC code (see
Chapter~\ref{chap:alnes-2}), which can be used by the \dolfin{}
assembler described in Chapter~\ref{chap:logg-3}. The UFC code is
JIT-compiled using Instant, see Chapter~\ref{chap:wilbers}.

This chapter is deliberately short and only gives the reader a taste
of the capabilities of \syfi{} and \sfc{}. However, most features are
covered by the more comprehensive manual. This chapter is organized as
follows.  We begin with a short description of GiNaC and Swiginac
before we present how finite elements are used and defined
using \syfi{}. Then, we present how to use \sfc{} in the \dolfin{}
environment, and end with a short description of the basic structure
of \sfc{}. \syfi{} is implemented in C++, but has a Python
interface. \sfc{} is implemented in Python because code generation is
much more convenient in this language.
%The code examples accompanying this chapter can be found in
%\emp{\$FENICSBOOK/src/syfi-sfc}.

\section{GiNaC and Swiginac}

GiNaC~\citep{BauerFrinkKreckel2000} is an open source C++ library
for symbolic calculations.  It contains the tools for doing basic
manipulations of polynomials like algebraic operations, differentiation,
and integration.  The following example shows basic usage of the library,
\begin{c++}
// create a polynomial function
symbol x("x");
ex f = x*(1-x);

// evaluate f
ex fvalue = f.subs(x == 0.5);
std::cout << " f(0.5) = " << fvalue << std::endl;

// differentiation
ex dfdx  = diff(f,x);
std::cout << " df/dx = " << dfdx << std::endl;

// integration
ex intf  = integral(x,0,1,f).eval_integ();
std::cout << " integral of f from 0 to 1 is: " << intf << std::endl;
\end{c++}
We will not go deeply into GiNaC here, but refer the reader to the
GiNaC tutorial and reference which can be found on its web page. There
are, however, a few issues we need to address to explain basic GiNaC
usage. First of all, GiNaC contains many different types like
\emp{symbol}, \emp{matrix}, \emp{function}, etc.  Normally, one does not
need to worry about these types since the type \emp{ex}, which was
used above, can represent any mathematical object (\emp{ex} is
basically a place-holder for the underlying object). Still, there are
mathematical operations that can only be applied to particular
types. For instance, functions can only be differentiated with respect
to symbols, as shown above.  Notice also that GiNaC overloads
operators like \emp{==} to enable creation of equations and
inequalities, which may be represented as expressions of
type \emp{relations} (or \emp{ex}).

Symbolic calculations can be computationally demanding. Therefore, GiNaC
separates between the construction and evaluation of expressions. This is
illustrated in the above example by the fact that we create an integral
object using the function \emp{integral}, but we need to explicitly call
the function \emp{eval\_integ} to compute the integral. In a similar
fashion one may use functions like \emp{eval}, \emp{evalm}, \emp{expand},
\emp{simplify}, and \emp{collect\_common\_factors} etc. to evaluate
and simplify expressions.  Finally, GiNaC implements its own memory
management system using reference counting.  The complete code can be
found at \emp{syfi-sfc/ginac}.

Swiginac is a Python interface to GiNaC created using SWIG.  Swiginac
provides a more or less direct translation of GiNaC to Python, but has
features that makes it easy to program in a Pythonic way.  For instance,
Swiginac unwraps the \emp{ex} objects and provides typemaps between
Python lists and GiNaC lists (\emp{lst}).  The following code translates
the above C++ example to Python, using Swiginac:
\begin{python}
from swiginac import *
x = symbol("x")
f = x*(1-x)

fvalue = f.subs(x == 0.5)
print "fvalue = ", fvalue

dfdx = diff(f,x)
print "df/dx = ", dfdx

intf = integral(x,0,1,f).eval_integ()
print "integral of f from 0 to 1 is:", intf
\end{python}

\section{\syfi{}: symbolic finite elements}

GiNaC provides the basic utilities for \syfi{} in the sense that it provides
manipulation of polynomials, as well as differentiation and integration
with respect to one variable. \syfi{} extends GiNaC with polynomial
spaces and differentiation operators like $\nabla$, $\nabla\cdot$, and
$\nabla\times$, in addition to integration over a number of polygonal
domains.  With these utilities it is easy to define finite elements.

Some elements that have been implemented include: continuous and
discontinuous Lagrange elements, the Crouzeix--Raviart element, the
Raviart--Thomas element, various $\Hdiv$ and $\Hcurl$ \nedelec{} elements, and
the Hermite elements. See Chapter~\ref{chap:kirby-6} for a description of
the above-mentioned elements.  A complete list of elements can be found in
the user manual. The mentioned elements are defined for arbitrary order,
except for the Crouzeix--Raviart and Hermite elements. Not all of these
elements are, however, supported by the form compiler.

The following example illustrates how to use \syfi{} to do finite element
calculations in Python.  Here, we create a Lagrange element of second
order and use the basis functions to compute a element stiffness matrix
on a reference triangle. We also print both the integrand and the element
matrix entries to the screen.\vspace*{-6pt}
\begin{python}
from swiginac import *
from SyFi import *

#initialize SyFi in 2D
initSyFi(2)

# create reference triangle
t = ReferenceTriangle()

# create second order Langrange element
fe = Lagrange(t,2)

for i in range(0, fe.nbf()):
    for j in range(0, fe.nbf()):
        integrand = inner(grad(fe.N(i)), grad(fe.N(j)))
        print "integrand[%d, %d]  =%s;" % (i, j, integrand.printc())
        integral = t.integrate(integrand)
        print "A[%d, %d]          =%s;" % (i, j, integral.printc())
\end{python}
The output from executing the above code is:\vspace*{-6pt}
\begin{c++}
integrand[0, 0]  =2.0*pow( 4.0*y+4.0*x-3.0,2.0);
A[0, 0]          =1.0;
integrand[0, 1]  = -4.0*( 4.0*y+4.0*x-3.0)*x+-4.0*( y+2.0*x-1.0)*( 4.0*y+4.0*x-3.0);
A[0, 1]          =-(2.0/3.0);
integrand[0, 2]  =( 4.0*y+4.0*x-3.0)*( 4.0*x-1.0);
A[0, 2]          =(1.0/6.0);
integrand[0, 3]  = -4.0*y*( 4.0*y+4.0*x-3.0)+-4.0*( 4.0*y+4.0*x-3.0)*( 2.0*y+x-1.0);
A[0, 3]          =-(2.0/3.0);
....
\end{c++}
Here, we see that the expressions are printed to the screen as symbolic
expressions in C++ syntax. Hence, the output is very reader--friendly
and this can be very useful during debugging.  We remark that also
Python and LaTeX output can be generated using the \emp{printpython}
and \emp{printlatex} functions.

All elements in \syfi{} are implemented in C++. Here, however, for
simplicity we list a definition of the Crouzeix--Raviart element in Python.
The following code is the complete finite element definition:
\begin{python}
from swiginac import *
from SyFi import *

class CrouzeixRaviart(object):
    """Python implementation of the Crouzeix-Raviart element."""

    def __init__(self, polygon):
        """Constructor"""
        self.Ns = []
        self.dofs = []
        self.polygon = polygon
        self.compute_basis_functions()

    def compute_basis_functions(self):
        """Compute the basis functions and degrees of freedom
        and put them in Ns and dofs, respectively."""

        # create the polynomial space
        N, variables, basis = bernstein(1,self.polygon,"a")

        # define the degrees of freedom
        for i in range(0,3):
            edge = self.polygon.line(i)
            dofi = edge.integrate(N)
            self.dofs.append(dofi)

        # compute and solve the system of linear equations
        for i in range(0,3):
            equations = []
                for j in range(0,3):
                    equations.append(self.dofs[j] == dirac(i,j))
                    sub = lsolve(equations, variables)
                    Ni = N.subs(sub)
                    self.Ns.append(Ni);

    def N(self,i): return self.Ns[i]
    def dof(self,i): return self.dofs[i]
    def nbf(self): return len(self.Ns)
\end{python}
The process of defining a finite element in \syfi{} is similar for all
elements.  As the above code shows, it resembles the Ciarlet
definition closely, see also the Chapters~\ref{chap:kirby-6}
and \ref{chap:kirby-1}. First, we construct a polynomial space. In
the code above, this is performed by calling the \emp{bernstein}
function. The \emp{bernstein} function takes as input a simplex and
the order of the Bernstein polynomial. Arbitrary order polynomials are
supported. This function produces a tuple consisting of the
polynomial, its coefficients (or degrees of freedom), and the basis
functions representing the polynomial space $P$:
\begin{python}
In  : bernstein(1, triangle, "a")
Out : [-a0_2*(-1+y+x)+y*a0_0+x*a0_1, [a0_0, a0_1, a0_2], [y, x, 1-y-x]]
\end{python}
In the above code, we used a triangle and the order of the polynomial
was one. The next task is to define a set of degrees of freedom; that is,
a set of functionals $L_i : P \rightarrow \R$.  For the Crouzeix--Raviart
element, the degrees of freedom are simply the integrals over an edge;
that is, $L_i(P) = \int_{E_i} P \dx$, where $E_i$ for $i=(0,1,2)$ are the
edges of the triangle. Alternatively we could have used the value at
the midpoint of the edges since the polynomial $P$ is linear.  Finally,
the different basis functions $\{N_i\}$ are determined by the set of
equations $L_i(N_j) = \delta_{ij}$.  These equations are then solved,
using \emp{lsolve}, to compute the basis functions of the elements;
that is, the coefficients \emp{[a0\_0, a0\_1, a0\_2]} are determined for
each specific basis function.

The basis functions of this element can then be displayed as follows:
\begin{python}
p0 = [0,0,0]; p1 = [1,0,0]; p2 = [0,1,0];
triangle = Triangle(p0, p1, p2)

fe = CrouzeixRaviart(triangle)
for i in range(0,fe.nbf()):
  print "N(%d)       = "%i,   fe.N(i)
  print "grad(N(%d)) = "%i,   grad(fe.N(i))
\end{python}
giving the following output:
\vspace*{9pt}\begin{c++}
N(0)       =  1/6*(-3+3*x+3*y+z)*2**(1/2)+1/6*2**(1/2)*(3*x-z)+1/6*2**(1/2)*(3*y-z)
grad(N(0)) =  [[2**(1/2)],[2**(1/2)],[-1/6*2**(1/2)]]
N(1)       =  1-2*x-1/3*z
grad(N(1)) =  [[-2],[0],[-1/3]]
...
\end{c++}

\section{\sfc{}: SyFi form compiler}

As mentioned earlier, \sfc{} translates UFL code to UFC code. Consider the
following UFL code for defining the variational problem for solving the
Poisson problem, implemented in a file \emp{Poisson.ufl}:
\begin{python}
cell = triangle
element = FiniteElement("Lagrange", cell, 1)

u = TrialFunction(element)
v = TestFunction(element)
c = Coefficient(element)
f = Coefficient(element)
g = Coefficient(element)

a = c*dot(grad(u),grad(v))*dx
L = -f*v*dx + g*v*ds
\end{python}
\sfc{} translates this UFL form to UFC code as follows:
\begin{bash}
sfc -w1  -ogenerated_code Poisson.ufl
\end{bash}
Here, \emp{-w1} means that \dolfin{} wrappers are generated, while
\emp{-ogenerated\_code} means that the generated code should be
located in the directory \emp{generated\_code}. Notice that the flags
and corresponding options are not separated by spaces. A complete
list of options is obtained with \emp{sfc -h}.  The generated code can be
utilized in DOLFIN in a standard fashion.  For a complete example consider
the demo \emp{demo/Poisson2D/cpp} that comes with the \syfi{} package.

In DOLFIN, the form compiler may be chosen at run-time by setting:
\begin{python}
parameters["form_compiler"]["name"] = "sfc"
\end{python}
The form compiler can be tuned with a range of options. A complete list
of options is obtained as follows:
\begin{python}
from sfc.common.options import default_options
sfc_options = default_options()
\end{python}
The object \emp{sfc\_options} is of type \emp{ParameterDict}, which is a
dictionary with some additional functionality. One may use the following
forms to set options before passing them to \emp{assemble}.
\begin{python}
sfc_options.code.integral.integration_method = "symbolic" # default is "quadrature"
# alternatively:
# sfc_options["code"]["integral"]["integration_method"] = "symbolic"
A = assemble(a, form_compiler_parameters=sfc_options)
\end{python}

Earlier versions of \sfc{} produced slow code for complicated nonlinear
equations as shown in \citet{AlnaesMardal2009b}. Furthermore, the code
generation was expensive both in terms of memory and the number of
operations required in the computations, because the \sfc{} implementation
did not scale linearly with the complexity of the equations.  However, a
significant speed-up came with the introduction of UFL with its expression
tree traversal algorithms.  Now, quite complicated equations can be
handled without losing computational efficiency. Consider for example
an elasticity problem where the constitutive law is a quite complicated
variant of \citet{Fung1993}, described by the following equations:
%
\begin{align}
F &= I + (\nabla u), \\
C &= F^{\top}: F, \\
E &= (C-I)/2, \\
\psi &= \frac{\lambda}{2} \, \mbox{tr}(E)^2 + K\exp((E A,E)), \\
P &= \frac{\partial \psi}{\partial E}, \\
\LL &= \int_\Omega P: (\nabla v) \dx, \\
\JJ_F &= \frac{\partial \LL}{\partial u}.
\end{align}
Here, $u$ is the unknown displacement, $v$ is a
test function, $I$ is the identity matrix, $A$ is a
matrix, $\lambda$ and $K$ are material parameters, $\LL$ is the system
of nonlinear equations to be solved, and $\JJ_F$ is the corresponding
Jacobian.  This variational form is implemented in DOLFIN as follows:
\begin{python}
mesh = UnitSquare(N, N)
V = VectorFunctionSpace(mesh, "Lagrange", order)
Q = FunctionSpace(mesh, "Lagrange", order)

U = Function(V)

v = TestFunction(V)
u = TrialFunction(V)

lamda = Constant(1.0)
A = Expression ((("1.0", "0.3"), ("0.3", "2.3")))
K = Constant(1.0)
n = U.cell().n

I = Identity(U.cell().d)
F = I + grad(U)
J = det(F)

C = F.T*F
E = (C-I)/2
E = variable(E)

psi = lamda/2 * tr(E)**2  +  K*exp(inner(A*E,E))
P = F*diff(psi, E)
a_f = psi*dx
L = inner(P, grad(v))*dx
J = derivative(L, U, u)

A = assemble(J)
\end{python}

To test the computational efficiency of the generated code for this
problem, we compare the assembly of $\JJ_F$ with the assembly of a
corresponding linear elasticity problem with the following matrix $\JJ_L$:
\begin{equation}
\JJ_L = \int_\Omega \lambda \nabla \cdot u \nabla \cdot v + (\lambda + \mu)\nabla u:\nabla v \dx.
\end{equation}
In Table \ref{SFCtest} we see a comparison of the efficiency for the
above examples.  Clearly, the nonlinear example is no more than 4 times
as slow as the linear problem when using linear elements, and only a
factor 2 when using quadratic elements.
\begin{table}
  \centering
  \begin{tabular}{ccccccc}
    \toprule
    $N$              & 100   & 200  & 400    \\
    \midrule
    $\JJ_L, p=1$     & 0.08  & 0.27  & 1.04  \\
    $\JJ_L, p=2$     & 0.36  & 1.41  & 5.45  \\
    $\JJ_F, p=1$     & 0.33  & 0.84  & 3.36  \\
    $\JJ_F, p=2$     & 0.68  & 2.26  & 8.55  \\
    \bottomrule
  \end{tabular}
  \caption{Comparison of the time (in seconds) for computing the Jacobian
    matrix for the two elasticity problems on a $N\times N$ unit square mesh
    for linear ($p=1$) and quadratic elements ($p=2$).}
  \label{SFCtest}
\end{table}

We refer to
\citet{AlnaesMardal2009b,OelgaardLoggWells2008,KirbyLogg2008,OelgaardWells2010}
for more discussions on the topic of efficient compilation of linear
and nonlinear variational formulations.

\section{Code generation design}

Finally, we briefly describe the overall design of the code
generation software.  UFC defines the interface of the code produced
by \sfc{}.  In \sfc{}, each UFC class is mirrored by subclasses of the
class \emp{CodeGenerator} called \emp{FormCG}, \emp{DofMapCG},
\emp{FiniteElementCG}, and \emp{CellIntegralCG}, etc.  These classes
are used to generate code for the corresponding UFC classes, \emp{form},
\emp{dofmap}, \emp{finite\_element}, \emp{cell\_integral}, etc. These
classes have a common function for generating the code, called\break
\emp{generate\_code\_dict}.  The function \emp{generate\_code\_dict}
generates a dictionary containing named pieces of UFC code, most of
which are function body implementations.  This dictionary with code is
then combined with format strings from the UFC utility Python module
to generate UFC compliant code.  An example of a format string is
shown below.

\begin{python}
cell_integral_implementation = """\
/// Constructor
%(classname)s::%(classname)s() : ufc::cell_integral()
{
%(constructor)s
}

/// Destructor
%(classname)s::~%(classname)s()
{
%(destructor)s
}

/// Tabulate the tensor for the contribution from a local cell
void %(classname)s::tabulate_tensor(double* A,
                                    const double * const * w,
                                    const ufc::cell& c) const
{
%(tabulate_tensor)s
}
"""
\end{python}
Using this template, the code generation in \sfc{} is then performed
as follows:
\begin{python}
def generate_cell_integral_code(integrals, formrep):
    sfc_debug("Entering generate_cell_integral_code")
    itgrep = CellIntegralRepresentation(integrals, formrep)

    cg = CellIntegralCG(itgrep)
    vars = cg.generate_code_dict()
    supportcode = cg.generate_support_code()

    hcode = ufc_utils.cell_integral_header % vars
    ccode = supportcode + "\n"*3 + ufc_utils.cell_integral_implementation % vars

    includes = cg.hincludes() + cg.cincludes()

    system_headers = common_system_headers()

    hincludes = "\n".join('#include "%s"' % inc for inc in cg.hincludes())
    cincludes =  "\n".join('#include <%s>' % f for f in system_headers)
    cincludes += "\n"
    cincludes += "\n".join('#include "%s"' % inc for inc in cg.cincludes())

    hcode = _header_template         % \
             { "body": hcode, "name": itgrep.classname, "includes": hincludes }
    ccode = _implementation_template % \
             { "body": ccode, "name": itgrep.classname, "includes": cincludes }
    sfc_debug("Leaving generate_cell_integral_code")
    return itgrep.classname, (hcode, ccode), includes
\end{python}
As seen above, the \emp{CellIntegralCG} class is again mirrored by a
corresponding class \\
\emp{CellIntegralRepresentation},
\begin{python}
class CellIntegralRepresentation(IntegralRepresentation):
    def __init__(self, integrals, formrep):
        IntegralRepresentation.__init__(self, integrals, formrep, False)

    def compute_A(self, data, iota, facet=None):
        "Compute expression for A[iota]."

        if data.integration_method == "quadrature":
            if self.options.safemode:
                integrand = data.integral.integrand()
                data.evaluator.update(iota)
                integrand = data.evaluator.visit(integrand)
            else:
                n = len(data.G.V())
                integrand = data.vertex_data_set[iota][n-1]

            D = self.formrep.D_sym
            A = integrand * D
            ...
\end{python}
The representation classes are quite involved, in particular when
using quadrature where the generated code involves multiple loops and
quite a few temporary variables. To generate quadrature based code,
the computational graph algorithms from UFL (in particular the class
\emp{ufl.algorithms.Graph}) are used to split the expression tree into smaller
subexpressions.  \sfc{} makes GiNaC symbols that represent temporary
variables for all subexpressions.  To place the temporary variables
inside the correct loops in the generated code, the computational
graph is partitioned based on the dependencies of subexpressions. See
Chapter~\ref{chap:alnes-1} for an explanation of the partitioning
algorithm provided by UFL. The subexpression associated with
each temporary variable is then translated to a C/C++ string using
GiNaC. Finally, \sfc{} puts it all together into a tabulate tensor
implementation in the code generation classes (\emp{*CG}).
