\fenicschapter{Electromagnetic waveguide analysis}
              {Electromagnetic waveguide analysis}
              {Electromagnetic waveguide analysis}
              {Evan Lezar and David B. Davidson}
              {lezar}

%------------------------------------------------------------------------------

At their core, Maxwell's equations\index{Maxwell's equations} are a
set of differential equations describing the interactions between
electric and magnetic fields, charges and currents. These equations
provide the tools with which to predict the behavior of
electromagnetic phenomena\index{electromagnetics}, giving us the
ability to use them in a wide variety of applications, including
communication and power generation. Due to the complex nature of
typical problems in these fields, numeric methods such as the finite
element method are often employed.

One of the earliest applications of the finite element method in
electromagnetics was in~\citet{Silvester1969}, where it was applied to
the analysis of waveguide\index{waveguide} structures. These
structures are typically bounded structures --- although open
waveguides do exist --- for which a countably infinite number of modes
satisfy Maxwell's equations and their associated boundary
conditions~\citep{Pozar2005}. The finite element analysis of these
structures is concerned with calculating these waveguide modes, which
are generally characterized by a complex propagation constant as well
as an associated electromagnetic field distribution (which may both be
a function of frequency). The formulation adopted in this work is that
of~\citet{LeeSunCendes1991} for lossless materials, with an extension
to the lossy case presented in~\citet{Lee1994}. An overview of the
state-of-the-art in the field is presented
in~\citet{Davidson2011}. Alternate formulations are discussed
subsequently.

Since waveguides are some of the most common structures in microwave
engineering, especially in areas where high power and low loss are
essential~\citep{Pozar2005}, their analysis is still a topic of much
interest. This chapter considers the use of \fenics{} in the cutoff
and dispersion analysis of these structures as well as the analysis of
waveguide discontinuities. These types of analysis form an important part
of the design and optimization of waveguide structures for a particular
purpose. In these kinds of waveguide problems, the solution of generalized
eigensystems are required with the eigenvalues and eigenvectors of the
systems associated with the waveguide modes that are of interest.

The aim of this chapter is to guide the reader through the process
followed in implementing solvers for various electromagnetic problems
with both cutoff and dispersion analysis considered in depth. To this
end, a brief introduction to electromagnetic waveguide theory, the
mathematical formulation of these problems, and the specifics of their
solutions using the finite element method are presented in
Section~\ref{lezar:sec:formulation}. This lays the groundwork for a
discussion of the details pertaining to the \fenics{} implementation
of these solvers, covered in
Section~\ref{lezar:sec:Implementation}. The translation of the finite
element formulation to \fenics{}, as well as some post-processing
considerations are covered. In Section~\ref{lezar:sec:Examples}, the
solution results for three typical waveguide configurations are
presented and compared to analytical or previously published
data. This serves to validate the implementation and illustrates the
kinds of problems that can be solved.

%------------------------------------------------------------------------------
\section{Formulation}
\label{lezar:sec:formulation}

In electromagnetics, the behavior of the electric and magnetic fields
are described by Maxwell's equations~\citep{Jin2002, Smith1997}.
Using these partial differential equations, various boundary value
problems can be obtained depending on the problem being solved.
In the case of time-harmonic fields, the equation used is the vector
Helmholtz\index{vector Helmholtz equation} wave equation. If the problem
is further restricted to a domain surrounded by perfect electrical or
magnetic conductors (as is the case in general waveguide problems)
the wave equation in terms of the electric field, $\vec{E}$, can be
written as~\citep{Jin2002}
\begin{equation}
    \label{eq:lezar:vector_helmholtz}
    \nabla\times\frac{1}{\mu_r}\nabla\times\vec{E} - k_o^2\epsilon_r\vec{E} = 0\qquad \text{in
    $\Omega_v$},
\end{equation}
subject to the boundary conditions
\begin{align}
    \label{eq:lezar:electric_wall_BC}
    \hat{n}\times\vec{E} &= 0\qquad \text{on $\Gamma_{e}$}\\
    \label{eq:lezar:magnetic_wall_BC}
    \hat{n}\times\nabla\times{\vec{E}}{} &= 0\qquad \text{on $\Gamma_{m}$},
\end{align}
with $\Omega_v$ representing the interior of the waveguide and
$\Gamma_{e}$ and $\Gamma_{m}$ electric and magnetic walls
respectively. $\mu_r$ and $\epsilon_r$ are the relative permeability
and relative permittivity respectively.  These are material parameters
that may be inhomogeneous (varying in space) but only the isotropic
case is considered here. In this case, isotropic means that the
medium's response is the same for all directions of the electric field
vector~\citep{RamoWhinneryVanDuzer1994}. It should be noted that the
formulations discussed here can also be extended to the anisotropic
case as in~\citet{PolycarpouLyonsBalanis1996}.

In \eqref{eq:lezar:vector_helmholtz}, $k_o$ is the operating
wavenumber\index{wavenumber!operating} which is related to the
operating frequency ($f_o$) by the expression
\begin{equation}
  \label{eq:lezar:operating_wavenumber}
 k_o = \frac{2\pi f_o}{c_0},
\end{equation}
with $c_0$ the speed of light in free space.  This boundary value
problem can also be written in terms of the magnetic field as
in~\citet{Jin2002}, but as the discussions that follow are applicable
to both formulations this will not be considered here.

One way to solve the boundary value problem is to find the stationary
point of the following variational functional
\begin{equation}
  \label{eq:lezar:standard_functional}
  F(\vec{E})
      = \frac{1}{2}\int_{\Omega_v} \left[ \frac{1}{\mu_r}(\nabla\times{\vec{E}})\cdot(\nabla\times{\vec{E}})
      - k_o^2\epsilon_r\vec{E}\cdot{\vec{E}} \right] \dx,
\end{equation}
which can be found in a number of computational electromagnetic texts,
including those by~\citet{Jin2002}
and~\citet{PelosiCoccioliSelleri1998}, as well as the paper
by~\citet{LeeSunCendes1991}. In the case of the waveguide problems
considered here, a number of simplifications can be made to the
solution process and these will now be discussed. Note that for this
source-free formulation, the $\frac{1}{2}$ factor
in \eqref{eq:lezar:standard_functional} is superfluous, and is
subsequently dropped.

If the guide is sufficiently long, and the $z$-axis is chosen parallel
to its cylinder axis as shown in
Figure~\ref{fig:lezar:long_waveguide}, then the $z$-dependence of the
electric field can be assumed to be of the form $e^{-\gamma z}$ with
$\gamma = \alpha + j\beta$ a complex propagation\index{propagation
constant} constant~\citep{PelosiCoccioliSelleri1998,
Pozar2005}. Making this assumption and splitting the electric field
into transverse ($\vec{E}_t = \hat{x}E_x + \hat{y}E_y$) and axial
($\hat{z}E_z$) components, results in the following expression for the
field
\begin{equation}
    \label{eq:lezar:field_components}
    \vec{E}(x,y,z) = [\vec{E}_t(x,y) + \hat{z}E_z(x,y)]e^{-\gamma z},
\end{equation}
with $x$ and $y$ the Cartesian coordinates in the cross sectional
plane of the waveguide and $z$ the coordinate along the length of the
waveguide. Here $\hat{x}$, $\hat{y}$, and $\hat{z}$, represent unit
vectors in the $x$, $y$, and $z$-direction respectively. For the
purpose of this discussion, also consider the following representation
of $\nabla$ in Cartesian coordinates
\begin{equation}
  \label{eq:lezar:nabla_components}
  \nabla = \nabla_t + \nabla_z,
\end{equation}
with
\begin{equation}
  \nabla_t = \frac{\partial}{\partial x}\hat{x} + \frac{\partial}{\partial y}\hat{y},
\end{equation}
the transverse gradient, and
\begin{equation}
  \nabla_z = \frac{\partial}{\partial z}\hat{z},
\end{equation}
the partial derivative with respect to $z$ in the $z$-direction.

\begin{figure}
  \centering
 \def\svgwidth{\smallfig}
 \import{chapters/lezar/svg/}{arbitrary_waveguide_cross_section.pdf_tex}
 \caption{A long waveguide with an arbitrary cross section aligned
 with the $z$-axis. Note the labels for the domain corresponding to the
 waveguide interior ($\Omega_v$) as well as the electric wall $\Gamma_e$.}
 \label{fig:lezar:long_waveguide}
\end{figure}

\enlargethispage{6pt}

By substituting the expression for the field in
\eqref{eq:lezar:field_components} as well as the decomposition of
$\nabla$ of~\eqref{eq:lezar:nabla_components} into the functional of
\eqref{eq:lezar:standard_functional} and performing a number of vector
manipulations, the following modified functional can be obtained
\begin{multline}
    \label{eq:lezar:functional}
    F(\vec{E}) = \int_{\Omega}\frac{1}{\mu_r}(\nabla_t\times{\vec{E}_t})\cdot(\nabla_t\times{\vec{E}_t})
    - k_o^2\epsilon_r\vec{E}_t\cdot{\vec{E}_t}\\
    +\frac{1}{\mu_r}(\nabla_t{E_z} + \gamma\vec{E}_t)\cdot(\nabla_t{E_z} + \gamma\vec{E}_t)
    -k_o^2\epsilon_r E_z{E_z} \dx.
\end{multline}
Note that in this case the integration domain ($\Omega_v$) of
\eqref{eq:lezar:standard_functional} -- representing the entire
waveguide interior volume -- has been replaced by integration over
the waveguide cross section -- indicated by the domain $\Omega$ in
\eqref{eq:lezar:functional} -- for an arbitrary $z$ position. Functionals
similar to the one shown in \eqref{eq:lezar:functional}
are employed in~\citet{LeeSunCendes1991}, \citet{Jin2002},
and~\citet{PelosiCoccioliSelleri1998}, although in the latter case,
this is derived by substituting~\eqref{eq:lezar:field_components} into
the original Helmholtz equation of~\eqref{eq:lezar:vector_helmholtz}.

Using two dimensional curl-conforming vector basis functions
($\vec{N}_i$) for the discretization of the transverse field (such as
the basis functions from the \nedelec{} function space of the first
kind~\citep{Nedelec1980, Webb1993, Monk2003}), and nodal scalar basis
functions ($L_i$) for the axial components~\citep{Jin2002,
PelosiCoccioliSelleri1998}, the discretized field components
(indicated by the $h$ subscript) of \eqref{eq:lezar:field_components}
are given by~\citep{Jin2002, PelosiCoccioliSelleri1998}
\begin{align}
  \label{eq:lezar:E_t_discretized}
  \vec{E}_{t,h} &= \sum_{i=1}^{N_N} (e_t)_i \vec{N}_i,\\
  \label{eq:lezar:E_z_discretized}
  E_{z,h} &= \sum_{i=1}^{N_L} (e_z)_i L_i.
\end{align}
Here $(e_t)_i$ and $(e_z)_i$ are the coefficient of the $i^\text{th}$
vector and scalar basis functions respectively, while $N_N$ and $N_L$
are the total number of each type of basis function used in the
discretization. The letters $N$ and $L$ are chosen for the basis
function names as a reminder that the basis functions come from
a \nedelec{} function space and a Lagrange polynomial space
respectively. A discussion on these and other basis functions is
presented in Chapter~\ref{chap:kirby-6}.

The formulation used here, where the electric field in the waveguide
is expressed as a combination of transverse and axial components, is
probably one of the most widely used in practice. A number of other
approaches have also been taken, with other vector formulations (most
notably that of Davies in \citet{ItohPelosiSilvester1996}) discussed
by~\citet{DillonWebb1994}. Other formulations, for instance, involve
only nodal elements; some use the axial fields as the working
variable; and the problem has also been formulated in terms of
potentials, rather than fields. A good summary of these may be found
in Chapter 9 of~\citet{ZhuCangellaris2006}.

\subsection{Waveguide cutoff analysis}
\label{lezar:sec:cutoff_formulation}
\index{waveguide!cutoff analysis}
One of the simplest cases to consider, and often a starting point when
testing a new finite element implementation, is waveguide cutoff
analysis. When a waveguide is operating at cutoff, the electric field
is constant along the $z$-axis which corresponds to $\gamma = 0$
in \eqref{eq:lezar:field_components}~\citep{Pozar2005}.  Substituting
$\gamma = 0$ into \eqref{eq:lezar:functional} yields the following
functional for cutoff analysis
\begin{equation}
    \label{eq:lezar:functional_cutoff}
    F_c(\vec{E}) = \int_{\Omega}\frac{1}{\mu_r}(\nabla_t\times{\vec{E}_t})\cdot(\nabla_t\times{\vec{E}_t})
    - k_c^2\epsilon_r\vec{E}_t\cdot{\vec{E}_t} +\frac{1}{\mu_r}(\nabla_t{E_z})\cdot(\nabla_t{E_z})
    -k_c^2\epsilon_r E_z{E_z} \dx.
\end{equation}
The symbol for the operating wavenumber\index{wavenumber!operating},
$k_o$, has been replaced with $k_c$, with the $c$ subscript
indicating that the quantity of interest is now the cutoff
wavenumber\index{wavenumber!cutoff}. This quantity and the electric
field distribution at cutoff are of interest in these kinds of problems.

Substituting the discretized field equations of
\eqref{eq:lezar:E_t_discretized} and \eqref{eq:lezar:E_z_discretized}
into the functional \eqref{eq:lezar:functional_cutoff} and
applying a minimization procedure, yields the following matrix
equation~\citep{Davidson2011}
\begin{equation}
    \label{eq:lezar:matrix_equation_cutoff}
    \begin{bmatrix}
      S_{tt} & 0\\0 & S_{zz}
    \end{bmatrix}
    \begin{Bmatrix}
    e_t
    \\
    e_z
    \end{Bmatrix}
  = k_c^2
    \begin{bmatrix}T_{tt} & 0
    \\
    0 &   T_{zz}\end{bmatrix}
    \begin{Bmatrix}
        e_t
      \\
      e_z
    \end{Bmatrix},
\end{equation}
or simply
\begin{equation}
  \begin{bmatrix}
      S
  \end{bmatrix}
  \begin{Bmatrix}
      e
    \end{Bmatrix}
  =
    k_c^2
    \begin{bmatrix}
      T
    \end{bmatrix}
    \begin{Bmatrix}
      e
    \end{Bmatrix}.
\end{equation}
The matrix equation of \eqref{eq:lezar:matrix_equation_cutoff}
is in the form of a generalized eigenvalue problem with the
square of the cutoff wavenumber the (unknown) eigenvalue. The
submatrices $S_{oo}$ and $T_{oo}$ (with $oo = tt, zz$)
represent the stiffness and mass matrices common in the finite element
literature~\citep{Davidson2011, Jin2002}. The subscripts $tt$ and $zz$
indicate transverse and axial components respectively. The entries of
the matrices of \eqref{eq:lezar:matrix_equation_cutoff} are defined
as \citep{PelosiCoccioliSelleri1998, Jin2002}
\begin{align}
\label{eq:lezar:S_tt_ij}
(S_{tt})_{ij} &= \int_\Omega \frac{1}{\mu_r}(\nabla_t\times\vec{N}_i)\cdot(\nabla_t\times\vec{N}_j) \dx,\\
\label{eq:lezar:T_tt_ij}
(T_{tt})_{ij} &= \int_\Omega \epsilon_r\vec{N}_i\cdot\vec{N}_j \dx,\\
\label{eq:lezar:S_zz_ij}
(S_{zz})_{ij} &= \int_\Omega \frac{1}{\mu_r}(\nabla_t L_i)\cdot(\nabla_t L_j) \dx,\\
\label{eq:lezar:T_zz_ij}
(T_{zz})_{ij} &= \int_\Omega \epsilon_r L_i L_j \dx,
\end{align}
with $\int_\Omega\cdot \dx$ representing integration over the cross
section of the waveguide.

In \eqref{eq:lezar:matrix_equation_cutoff} the possible cutoff
wavenumbers $k_c$ are the square roots of the eigenvalues of the system
and the elements of the corresponding eigenvectors are the coefficient
of the basis functions as in \eqref{eq:lezar:E_t_discretized}
and \eqref{eq:lezar:E_z_discretized}. As such, the solution of
the eigensystem not only allows for the computation of the cutoff
wavenumbers, but also for the visualization of the fields associated with
the modes by substituting the elements of the computed eigenvector into
\eqref{eq:lezar:E_t_discretized} and \eqref{eq:lezar:E_z_discretized}. It
should also be noted that transverse electric ($TE$) modes
will have zeros as coefficients for the scalar basis functions
($\begin{Bmatrix}e_z\end{Bmatrix} = 0$) whereas transverse magnetic
modes will have $\begin{Bmatrix}e_t\end{Bmatrix} = 0$, although this
condition only holds at cutoff \citep{Pozar2005}.

\subsection{Waveguide dispersion analysis}
\label{lezar:sec:propagation_curves}
\index{waveguide!dispersion analysis}

In the case of cutoff analysis discussed
in~\ref{lezar:sec:cutoff_formulation}, one attempts to obtain the value
of $k_o^2 = k_c^2$ for a given propagation constant $\gamma$, namely
$\gamma = 0$.  For most waveguide design applications however, $k_o$ is
specified and the propagation constant is calculated from the resultant
eigensystem~\citep{Jin2002, PelosiCoccioliSelleri1998}. This calculation
can be simplified somewhat by making the following substitution into
\eqref{eq:lezar:functional} (after multiplying by
$\gamma^2$)
\begin{equation}
    \vec{E}_{t,\gamma} = \gamma\vec{E}_t,
\end{equation}
which yields the modified functional
\begin{multline}
    \label{eq:lezar:functional_scaled}
    F_\gamma(\vec{E}) =
    \int_{\Omega}\frac{1}{\mu_r}(\nabla_t\times{\vec{E}_{t,\gamma}})\cdot(\nabla_t\times{\vec{E}_{t,\gamma}}) -k_o^2\epsilon_r\vec{E}_{t,\gamma}\cdot{\vec{E}_{t,\gamma}}\\
    +\gamma^2\left[\frac{1}{\mu_r}(\nabla_t{E_z} +
    \vec{E}_{t,\gamma})\cdot(\nabla_t{E_z} + \vec{E}_{t,\gamma})-k_o^2\epsilon_r
    E_z{E_z}\right] \dx.
\end{multline}
Using the same discretization as for the cutoff analysis discussed in
the preceding section, the matrix equation associated with the solution
of the variational problem is given by~\citep{PelosiCoccioliSelleri1998}
\begin{equation}
    \label{eq:lezar:matrix_equation_dispersion}
    \begin{bmatrix} A_{tt} & 0\\0 & 0\end{bmatrix}\begin{Bmatrix}e_t\\e_z\end{Bmatrix}
    =
    -\gamma^2\begin{bmatrix} B_{tt} & B_{tz}\\B_{zt} &
    B_{zz}\end{bmatrix}\begin{Bmatrix}e_t\\e_z\end{Bmatrix},
\end{equation}
with
\begin{align}
    \label{eq:lezar:A_tt}
    A_{tt} = S_{tt} - k_o^2 T_{tt},\\
    \label{eq:lezar:B_zz}
    B_{zz} = S_{zz} - k_o^2 T_{zz},
\end{align}
which is also in the form of a generalized eigenvalue
problem\index{eigenvalue problem} with the eigenvalues a function of
the square of the complex propagation constant ($\gamma$).

The matrices $S_{tt}$, $T_{tt}$, $S_{zz}$, and $T_{zz}$ are identical to
those defined for the waveguide cutoff analysis of the previous section,
with entries given by \eqref{eq:lezar:S_tt_ij}, \eqref{eq:lezar:T_tt_ij},
\eqref{eq:lezar:S_zz_ij}, and \eqref{eq:lezar:T_zz_ij} respectively. The
entries of the other submatrices, $B_{tt}$, $B_{tz}$, and $B_{zt}$,
are defined by
\begin{align}
\label{eq:lezar:B_tt_ij}
(B_{tt})_{ij} &=
\int_\Omega \frac{1}{\mu_r}\vec{N}_i\cdot\vec{N}_j \dx,\\
\label{eq:lezar:B_tz_ij}
(B_{tz})_{ij} &=
\int_\Omega \frac{1}{\mu_r}\vec{N}_i\cdot\nabla_t{L_j} \dx,\\
\label{eq:lezar:B_zt_ij}
(B_{zt})_{ij} &=
\int_\Omega \frac{1}{\mu_r}\nabla_t{L_i}\cdot\vec{N}_j \dx.
\end{align}

A common challenge in electromagnetic eigenvalue problems such as these is
the occurrence of spurious modes which are discussed in~\citet{Jin2002}
and~\citet{Davidson2011}\index{spurious modes}. These are non-physical
modes that fall in the null space of the $\nabla\times{\nabla\times}$
operator of \eqref{eq:lezar:vector_helmholtz}~\citep{Bossavit1998},
with the issue of spurious modes revisited in the work
by~\citet{FernandesRaffetto2002}.

One of the strengths of the curl-conforming vector basis functions
(edge elements) used in the discretization of the transverse component
of the field, is that they allow for a better representation of the
null-space in question and improve the modelling of singularities when
compared to nodal basis functions \citep{Webb1993}. This means that the
null-space modes can be more readily identified \citep{Davidson2011,
Jin2002}. A number of other solutions to the problem have
been proposed. These include the use of Lagrange multipliers as
in~\citet{VardapetyanDemkowicz2002}, the use of a divergence term to
regularize the $\nabla\times{\nabla\times}$ operator in the functional
of \eqref{eq:lezar:standard_functional}~\citep{ConstableDauge2002},
and the use of a discontinuous Galerkin formulation as presented
in~\citet{BuffaHoustonPerugia2007}, but are not discussed further in
this chapter.

\section{Implementation}
\label{lezar:sec:Implementation}

This section considers the details of the implementation of a
\fenics-based solver for waveguide cutoff mode and dispersion
curve problems, as described in the preceding section.
%~\ref{lezar:sec:cutoff_formulation}
%and~\ref{lezar:sec:propagation_curves}.
A number of code snippets
illustrate some of the finer points of the implementation.

\subsection{Formulation}

The code listing that follows shows the definitions of the function
spaces used in the solution of the cutoff and dispersion problems
considered here. As already discussed, the \nedelec{} basis functions of
the first kind are used to approximate the transverse component of the
electric field. This ensures the tangential continuity of the discrete
transverse field~\citep{Jin2002}.  The axial component of the field is
modelled using a set of Lagrange basis functions, with the integration
domain ($\Omega$) the waveguide cross section. The finite element mesh
(generated using the \dolfin{} \emp{Rectangle} class) for the
rectangular waveguide problems considered here is shown in
Figure~\ref{fig:lezar:rectangular_fenics_mesh}.
\begin{python}
V_N = FunctionSpace(mesh, "Nedelec 1st kind H(curl)", transverse_order)
V_L = FunctionSpace(mesh, "Lagrange", axial_order)

combined_space = V_N * V_L

(N_v, L_v) = TestFunctions(combined_space)
(N_u, L_u) = TrialFunctions(combined_space)
\end{python}

\begin{figure}
  \centering
    \def\svgwidth{\smallfig}
  \import{chapters/lezar/svg/}{rectangular_fenics_mesh.pdf_tex}
  \caption{An illustration of the\break finite element mesh used for the
  rectangular waveguide problems considered here. The mesh corresponds
  to 64 triangular elements and it should be noted that only the interior
  of the waveguide is meshed.}
  \index{dispersion curves}
  \label{fig:lezar:rectangular_fenics_mesh}
\end{figure}

In order to deal with material properties, the \emp{Expression} class
is subclassed and the \emp{eval} method is overridden. This is illustrated
in the next listing, which shows the implementation of the dielectric
properties of a half-filled rectangular guide defined as follows
\begin{equation}
\label{eq:lezar:half_filled_dielectric}
\epsilon_r (x,y) =
\begin{cases}
  4\quad &\text{if $y < 0.25$ },\\
  1\quad &\text{otherwise}.\\
\end{cases}
\end{equation}
This class is then instantiated for the relative permittivity
($\epsilon_r$) and a constant expression is used for the relative
permeability (or more specifically its inverse ($\frac{1}{\mu_r}
= 1$)). The listing also shows the Expression class used for the
square of the operating wavenumber ($k_o^2$), which is frequency
dependent. Note that although it is set to zero, this value can
be set for each frequency step as part of the dispersion analysis of a
waveguide structure.

\begin{python}
class HalfLoadedDielectric(Expression):
    def eval(self, values, x):
        if x[1] < 0.25:
            values[0] = 4.0
        else:
            values[0] = 1.0;

e_r = HalfLoadedDielectric()
one_over_u_r = Expression("1.0")

k_o_squared = Expression("value", {"value" : 0.0})
\end{python}

The testing and trial functions shown as well as the
desired material properties can now be used to create
the forms required for matrix assembly as specified in
\eqref{eq:lezar:S_tt_ij} through \eqref{eq:lezar:T_zz_ij}, and
\eqref{eq:lezar:A_tt} through \eqref{eq:lezar:B_zt_ij}. The
implementations of the forms are shown in the listing below,
and the matrices of \eqref{eq:lezar:matrix_equation_cutoff} and
\eqref{eq:lezar:matrix_equation_dispersion} can be assembled using the
required combinations of these forms. It should be noted that the use of
the \emp{Expression} class for the representation of $k_o^2$ means that the forms need
not be recompiled each time the operating frequency is changed. This
is especially beneficial when the calculation of dispersion curves
is considered since the same calculation is performed for a range of
operating frequencies.

\begin{python}
s_tt = one_over_u_r*dot(curl_t(N_v), curl_t(N_u))
t_tt = e_r*dot(N_v, N_u)

s_zz = one_over_u_r*dot(grad(M_v), grad(M_u))
t_zz = e_r*M_v*M_u

b_tt = one_over_u_r*dot(N_v, N_u)
b_tz = one_over_u_r*dot(N_v, grad(M_u))
b_zt = one_over_u_r*dot(grad(M_v), N_u)

a_tt = s_tt - k_o_squared*t_tt
b_zz = s_zz - k_o_squared*t_zz
\end{python}

From \eqref{eq:lezar:electric_wall_BC} it follows that the tangential
component of the electric field must be zero on perfectly electrical
conducting (PEC) surfaces~\citep{Smith1997}. What this means in
practice is that the degrees of freedom associated with both the
Lagrange and \nedelec{} basis functions on the boundary must be set
to zero. An implementation example for a PEC surface surrounding the
entire computational domain is shown in the code listing below as the
\emp{ElectricWalls} class. This subdomain is then used to create a
Dirichlet boundary condition that can be applied to the constructed
matrices before solving the eigenvalue systems.

\begin{python}
class ElectricWalls(SubDomain):
    def inside(self, x, on_boundary):
        return on_boundary

zero = Expression("0.0","0.0","0.0")
dirichlet_bc = DirichletBC(combined_space, zero, ElectricWalls())
\end{python}

The boundary condition given in \eqref{eq:lezar:magnetic_wall_BC}
is a natural boundary condition for the problems and formulations
considered and thus it is not necessary to explicitly enforce
it~\citep{PelosiCoccioliSelleri1998}. Such magnetic walls and the
symmetry of a problem are often used to decrease the size of the
computational domain although this does limit the solution obtained to
even modes~\citep{Jin2002}.

Once the required matrices have been assembled and the boundary conditions
applied, the resultant eigenproblem can be solved. This can be done by
saving the matrices and solving the problem externally, or by making
use of the eigensolvers provided by SLEPc -- which is discussed in more
detail in Chapter~\ref{chap:rognes} -- through the \fenics{} package.

\subsection{Post-processing}

After the eigenvalue system has been solved, an eigenpair can
be post-processed to obtain various quantities of interest. For the
cutoff wavenumber\index{wavenumber!cutoff}, this is a relatively
straight-forward process and only involves simple operations on the
eigenvalues of the system. For the calculation of dispersion curves
and visualization of the resultant field components the process is
slightly more complex.

\paragraph{Dispersion curves.}
\index{dispersion curves}

For dispersion curves the computed value of the propagation constant
($\gamma = \alpha + j\beta$) is plotted as a function of the operating
frequency ($f_o$). Since $\gamma$ is a complex variable, a mapping is
required to represent the data on a single two-dimensional graph. This
is achieved by choosing the $f_o$-axis to represent the value $\gamma
= 0$, effectively dividing the {$\gamma-f_o$} plane into two
regions. The region above the $f_o$-axis is used to represent the
magnitude of the imaginary part of $\gamma$ ($|\beta|$), whereas the
magnitude of the real part ($|\alpha|$) falls in the lower region. A
mode that propagates along the guide for a given frequency will thus
lie in the upper half-plane of the plot, an evanescent mode will fall
in the lower half-plane, and a complex mode will be represented by a
data point above and below the $f_o$-axis.  This procedure is followed
in~\citet{PelosiCoccioliSelleri1998} and allows for quick comparisons
and validation of results.

\paragraph{Field visualization.}

In order to visualize the fields associated with a given solution, the
basis functions need to be weighted with coefficients corresponding to
the entries in an eigenvector obtained from one of the eigenvalue
problems. In addition, the transverse or axial components of the field
may need to be extracted. An example for plotting the transverse and
axial components of the field is given in the code listing below.
Here the variable \emp{x} assigned to the function vector is one of
the eigenvectors obtained by solving the eigenvalue
problem. The \emp{eval} method of the \emp{transverse} and \emp{axial}
functions can also be called in order to evaluate the functions at a
given spatial coordinate, allowing for further visualization or
post-processing options.
\begin{python}
f = Function(combined_space, x)

(transverse, axial) = f.split()

plot(transverse)
plot(axial)
\end{python}

\section{Examples}
\label{lezar:sec:Examples}

The first of the examples considered is the canonical one of a hollow
rectangular waveguide, which has been covered in a multitude of texts
on the subject~\citep{Davidson2011, Jin2002,
PelosiCoccioliSelleri1998, Pozar2005}. Since the analytical solutions
for this structure are known, it provides an excellent benchmark and
is a typical starting point for the validation of a computational
electromagnetic solver for solving waveguide problems.

The second and third examples are a partially filled rectangular guide
and a shielded microstrip line on a dielectric substrate,
respectively. In each case results are compared to published results
from the literature for validation.

\subsection{Hollow rectangular waveguide}
\index{waveguide!hollow rectangular|(}

Figure~\ref{fig:lezar:hollow_rectangular_guide} shows the cross
section of a hollow rectangular waveguide with dimensions
$a = 1\text{m}$ and $b = 0.5\text{m}$.
The analytical expressions for the electric field
components of a hollow rectangular guide with width $a$ and height $b$
are given by~\citep{Pozar2005}
\begin{align}
    \label{eq:lezar:rect:E_x analytical}
    E_x &= \frac{n}{b}A_{mn}\cos\left(\frac{m\pi
    x}{a}\right)\sin\left(\frac{n\pi y}{b}\right),\\
    \label{eq:lezar:rect:E_y analytical}
    E_y &= -\frac{m}{a}A_{mn}\sin\left(\frac{m\pi x}{a}\right)\cos\left(\frac{n\pi y}{b}\right),
\end{align}
for the $TE_{mn}$ (transverse electric) modes. These modes have
electric field components in the waveguide cross section and
correspond with the transverse part ($\vec{E}_t$) of the finite
element solution. The subscripts $mn$ are used to identify the modes,
with $m$ and $n$ non-negative integers subject to the restriction that
at least one of them must be nonzero. These transverse electric modes
have electric field components only in the plane perpendicular to the
direction of propagation~\citep{Pozar2005}.

\begin{figure}
    \centering
    \def\svgwidth{\smallfig}
    \import{chapters/lezar/svg/}{hollow_rectangular_waveguide.pdf_tex}
    \caption{A diagram showing the cross section ($\Omega$) and dimensions
    of a $1\text{m}\times~0.5\text{m}$ hollow rectangular waveguide.
    The electric wall $\Gamma_e$, where the zero Dirichlet
    boundary condition of \eqref{eq:lezar:electric_wall_BC} is applied, is also shown.}
    \label{fig:lezar:hollow_rectangular_guide}
\end{figure}

The $z$-directed (axial) electric field corresponds to the $TM_{mn}$
(transverse magnetic) modes and has the form~\citep{Pozar2005}
\begin{equation}
 \label{eq:lezar:rect_E_z_analytical}
 E_z = B_{mn}\sin\left(\frac{m\pi x}{a}\right)\sin\left(\frac{n\pi y}{b}\right).
\end{equation}
Once again the subscript $mn$ is used to identify the mode, but in
this case neither $m$ nor $n$ may be zero. Such a TM mode has
components of the magnetic field in the $xy$-plane, while the electric
field has only an axial component. In \eqref{eq:lezar:rect:E_x
analytical}, \eqref{eq:lezar:rect:E_y analytical},
and \eqref{eq:lezar:rect_E_z_analytical}, $A_{mn}$ and $B_{mn}$ are
constants for a given mode.

For a hollow rectangular guide, the propagation constant, $\gamma$,
has the form
\begin{equation}
 \label{eq:lezar:rectangular_propagation}
 \gamma = \sqrt{k_c^2 - k_o^2},
\end{equation}
with $k_o$ the operating wavenumber dependent on the operating frequency
as in \eqref{eq:lezar:operating_wavenumber}, and
\begin{equation}
 \label{eq:lezar:rectangular_cutoff}
 k_c^2 = \left(\frac{m\pi}{a}\right)^2 + \left(\frac{n\pi}{b}\right)^2,
\end{equation}
the analytical solution for the square of the cutoff
wavenumber\index{wavenumber!cutoff} for both the $TE_{mn}$ and
$TM_{mn}$ modes.

\paragraph{Cutoff analysis.}
\index{waveguide!cutoff analysis}

Figure~\ref{fig:lezar:rectangular_cutoff_TE} and
Figure~\ref{fig:lezar:rectangular_cutoff_TM} show the calculated $TE_{10}$
and $TM_{11}$ cutoff modes, respectively, for the hollow rectangular
guide of Figure~\ref{fig:lezar:hollow_rectangular_guide}.
\begin{figure}
\bwfig
\centering
    \def\svgwidth{\smallfig}
    \import{chapters/lezar/svg/}{hollow_cutoff_TE.pdf_tex}
\caption{The calculated $TE_{10}$ cutoff mode for the
$1\text{m}\times~0.5\text{m}$ hollow rectangular waveguide shown in
Figure~\ref{fig:lezar:hollow_rectangular_guide}.}
\label{fig:lezar:rectangular_cutoff_TE}
\vspace*{6pt}
\end{figure}

\begin{figure}
\bwfig
\centering
  \def\svgwidth{\smallfig}
    \import{chapters/lezar/svg/}{hollow_cutoff_TM.pdf_tex}
\caption{The calculated $TM_{11}$ cutoff mode for the
$1\text{m}\times~0.5\text{m}$ hollow rectangular waveguide shown in
Figure~\ref{fig:lezar:hollow_rectangular_guide}.}
\label{fig:lezar:rectangular_cutoff_TM}
\end{figure}
Table~\ref{tab:lezar:rectangular_cutoff_comparison} gives a comparison
of the calculated and analytical values for the square of the cutoff
wavenumber\index{wavenumber!cutoff} of a number of modes for a hollow
rectangular guide. As can be seen from the table, there is excellent
agreement between the values.

\begin{table}
 \centering
    \caption{Comparison of analytical and calculated cutoff
    wavenumber squared ($k_c^2$) for various TE and TM modes of a
    $1\text{m}\times0.5\text{m}$ hollow rectangular waveguide.}
    \label{tab:lezar:rectangular_cutoff_comparison}
    \begin{tabular}{cccc}
      \toprule
      Mode & Analytical [m$^{-2}$] & Calculated [m$^{-2}$] & Relative Error \\
      \midrule
      $TE_{10}$ & 9.8696 & 9.8696 & 1.4452e-06\\
      $TE_{01}$ & 39.4784 & 39.4784 & 2.1855e-05\\
      $TE_{20}$ & 39.4784 & 39.4784 & 2.1894e-05\\
      \midrule
      $TM_{11}$ & 49.3480 & 49.4048 & 1.1514e-03 \\
      $TM_{21}$ & 78.9568 & 79.2197 & 3.3295e-03\\
      $TM_{31}$ & 128.3049 & 129.3059 & 7.8018e-03\\
      \bottomrule
    \end{tabular}
\end{table}

\paragraph{Dispersion analysis.}
\index{waveguide!dispersion analysis}

When considering the calculation of the dispersion curves for the
hollow rectangular waveguide, the mixed formulation as discussed in
\ref{lezar:sec:propagation_curves} is used. The calculated dispersion
curves for the first 5 modes of the hollow rectangular guide are shown
in Figure~\ref{fig:lezar:hollow_rectangular_dispersion_curves}
along with the analytical results. For the rectangular
guide a number of modes are degenerate~\cite[see][Chapter
10]{Davidson2011} with the same dispersion and cutoff properties
as predicted by \eqref{eq:lezar:rectangular_propagation} and
\eqref{eq:lezar:rectangular_cutoff}. (As an example consider the $TE_{01}$
and $TM_{20}$ modes that will be degenerate for any rectangular waveguide
that is twice as wide as it is high, as is the case here.) There is
excellent agreement between the analytical and computed results.
\begin{figure}
\bwfig
  \centering
    \def\svgwidth{\largefig}
  \import{chapters/lezar/svg/}{hollow_rectangular_waveguide_dispersion_curve.pdf_tex}
  \caption{Dispersion curves for the first 5 modes of a
  $1\text{m}\times~0.5\text{m}$ hollow rectangular waveguide of
  Figure~\ref{fig:lezar:hollow_rectangular_guide}. Markers are used to
  indicate the analytical results with $\blacksquare$ and $\blacklozenge$
  indicating TE and TM modes respectively. Note that the analytical
  $TE_{01}$ and $TE_{20}$ form a degenerate pair, as do the $TE_{11}$
  and $TM_{11}$ modes.}
  \index{dispersion curves}
  \label{fig:lezar:hollow_rectangular_dispersion_curves}
\end{figure}
\index{waveguide!hollow rectangular|)}

\subsection{Half-loaded rectangular waveguide}
\index{waveguide!half-loaded rectangular|(}

In some cases, a hollow rectangular guide may not be the ideal structure
to use due to, for example, limitations on its dimensions. If the guide
is filled with a dielectric material with a relative permittivity
$\epsilon_r > 1$, the cutoff frequency of the dominant mode will be
lowered. Consequently a loaded waveguide will be more compact than
a hollow guide for the same dominant mode frequency. Furthermore,
in many practical applications, such as impedance matching or phase
shifting sections, a waveguide that is only partially loaded is
used~\citep{Pozar2005}.

Figure~\ref{fig:lezar:half_filled_rectangular_guide} shows the cross
section of such a guide. The guide considered here has the same dimensions
as the hollow rectangular waveguide used in the previous section, but
its lower half is filled with an $\epsilon_r = 4$ dielectric material.
\begin{figure}
    \centering
    \def\svgwidth{\smallfig}
    \import{chapters/lezar/svg/}{half_filled_rectangular_waveguide.pdf_tex}
    \caption{A diagram showing the cross section ($\Omega$)
    and dimensions of a $1\text{m}\times~0.5\text{m}$ half-loaded
    rectangular waveguide. The lower half of the guide is filled with
    an $\epsilon_r = 4$ dielectric material. The electric
    wall $\Gamma_e$, where the zero Dirichlet boundary condition of
    \eqref{eq:lezar:electric_wall_BC} is applied, is also shown.}
    \label{fig:lezar:half_filled_rectangular_guide}
\end{figure}

\paragraph{Cutoff analysis.}
\index{waveguide!cutoff analysis}

Figure~\ref{fig:lezar:half_filled_rectangular_cutoff_TE} and
Figure~\ref{fig:lezar:half_filled_rectangular_cutoff_TM} show the
$TE_{10}$ and $TM_{11}$ cutoff modes of the half-loaded guide (shown in
Figure~\ref{fig:lezar:half_filled_rectangular_guide}) respectively. Note
the concentration of the transverse electric field in the hollow part of
the guide. This is due to the fact that the displacement flux, $\vec{D}
= \epsilon\vec{E}$, must be normally continuous at the dielectric
interface~\citep{Pozar2005, Smith1997}.
\begin{figure}
\bwfig
\centering
  \def\svgwidth{\smallfig}
    \import{chapters/lezar/svg/}{half_filled_cutoff_TE.pdf_tex}
\caption{The first calculated cutoff mode of a
$1\text{m}\times0.5\text{m}$ half-filled rectangular waveguide as shown
in Figure~\ref{fig:lezar:half_filled_rectangular_guide}. The dielectric
surface is shown as a dashed horizontal line.}
\label{fig:lezar:half_filled_rectangular_cutoff_TE}
\end{figure}

\begin{figure}
\bwfig
\centering
  \def\svgwidth{\smallfig}
    \import{chapters/lezar/svg/}{half_filled_cutoff_TM.pdf_tex}
\caption{The forth calculated cutoff mode of a
$1\text{m}\times0.5\text{m}$ half-filled rectangular waveguide as shown
in Figure~\ref{fig:lezar:half_filled_rectangular_guide}. The dielectric
surface is shown as a dashed horizontal line.}
\label{fig:lezar:half_filled_rectangular_cutoff_TM}
\end{figure}

\paragraph{Dispersion analysis.}
\index{waveguide!dispersion analysis}

The dispersion curves for the first 4
modes of the half-loaded waveguide are shown in
Figure~\ref{fig:lezar:half_loaded_rectangular_dispersion_curves}
with results for the same modes from~\citet{Jin2002} provided as
reference. Here it can be seen that the cutoff frequency of the dominant
mode has decreased and there is no longer the same degeneracy in the
modes when compared to the hollow guide of the same dimensions.

\enlargethispage{12pt}

\subsection{Shielded microstrip}
\index{shielded microstrip|(}
\index{microstrip|see{shielded microstrip}}
\label{lezar:sec:shielded_microstrip}

Microstrip line is a very popular type of planar transmission
line, primarily due to the fact that it can be constructed using
photolithographic processes and integrates easily with other microwave
components~\citep{Pozar2005}. Such a structure typically consists of a
thin conducting strip on a dielectric substrate above a ground plane. In
addition, the strip may be shielded by enclosing it in a PEC box to reduce
electromagnetic interference. A cross section of a lossless shielded
microstrip line is shown in Figure~\ref{fig:lezar:shielded_microstrip}
with the thickness of the strip, $t$, exaggerated for clarity. The
dimensions used to obtain the results discussed here, are the same as
those in \citet{PelosiCoccioliSelleri1998}, and are indicated in the figure.

Since the shielded microstrip structure consists of two
conductors, it supports a dominant transverse electromagnetic
(TEM) wave that has no axial component of the electric or
magnetic field~\citep{Pozar2005}. Such a mode has a cutoff
wavenumber\index{wavenumber!cutoff} of zero and thus propagates for all
frequencies~\citep{Jin2002,PelosiCoccioliSelleri1998}. Although it can
be performed, the cutoff analysis of this structure is not considered
here explicitly and only the dispersion analysis is performed. The
cutoff wavenumbers for the higher order modes (which are hybrid TE-TM
modes~\citep{Pozar2005}) can however be determined from the dispersion
curves by the intersection of a curve with the $f_o$-axis.\pagebreak

\begin{figure}
\bwfig
 \centering
  \def\svgwidth{\largefig}
  \import{chapters/lezar/svg/}{half_filled_rectangular_waveguide_dispersion_curve.pdf_tex}
 \caption{Dispersion curves for the first 4 modes of a
 $1\text{m}\times~0.5\text{m}$ half-filled rectangular waveguide as shown
 in Figure~\ref{fig:lezar:half_filled_rectangular_guide}. Reference values
 for the first 4 modes from~\citet{Jin2002} are shown as $\blacksquare$.}
 \index{dispersion curves}
 \label{fig:lezar:half_loaded_rectangular_dispersion_curves}
\end{figure}
\index{waveguide!half-loaded rectangular|)}

\begin{figure}
\bwfig
        \centering
    \def\svgwidth{\smallfig}
    \import{chapters/lezar/svg/}{shielded_microstrip_waveguide.pdf_tex}
    \caption{A diagram showing the cross section and dimensions of a
    shielded microstrip line. The microstrip is etched on a dielectric
    material with a relative permittivity of $\epsilon_r = 8.875$. The
    plane of symmetry is indicated by a dashed line and is modelled
    as a magnetic wall ($\Gamma_m$) in order to reduce the size of the
    computational domain. The electric wall ($\Gamma_e$) is also shown.}
    \label{fig:lezar:shielded_microstrip}
\end{figure}\clearpage

\begin{figure}
\bwfig
 \centering
  \def\svgwidth{\largefig}
    \import{chapters/lezar/svg/}{boxed_microstrip_dispersion_curves.pdf_tex}
 \caption{Dispersion curves for the first 7 even modes of shielded
 microstrip line of Figure~\ref{fig:lezar:shielded_microstrip}
 using a magnetic wall to enforce symmetry. Reference values
 from~\citet{PelosiCoccioliSelleri1998} are shown as $\blacksquare$. The
 presence of complex mode pairs are indicated by $\blacktriangle$ and
 $\bullet$ and highlighted in grey.}
 \index{dispersion curves}
 \label{fig:lezar:shielded_microstrip_dispersion_curves}
\end{figure}
\label{lezar:sec:shielded_microstrip|)}


\paragraph{Dispersion analysis.}
\index{waveguide!dispersion analysis}

The dispersion analysis presented in~\citet{PelosiCoccioliSelleri1998}
is repeated here for validation, with the resultant curves shown in
Figure~\ref{fig:lezar:shielded_microstrip_dispersion_curves}. As is the
case with the half-loaded guide, the results calculated with \fenics{}
agree well with previously published results. In this figure, it is shown
that for certain parts of the frequency range of interest, mode six
and mode seven have complex propagation constants. Since the matrices in the
eigenvalue problem are real valued, the complex eigenvalues -- and thus
the propagation constants -- must occur in complex conjugate pairs as
is the case here and reported earlier in~\citet{HuangItoh1988}. It
should be noted that for lossy materials (not considered here),
complex modes are expected but do not necessarily occur in conjugate
pairs \citep{PelosiCoccioliSelleri1998}.

\section{Conclusion.}

In this chapter, the solutions of cutoff and dispersion problems
associated with electromagnetic waveguiding structures have been
implemented and the results analyzed. In all cases, the results obtained
agree well with previously published or analytical results.

It should be noted that although the examples are limited to
two-dimensional resonant problems, the formulations presented here can
be extended to include three-dimensional eigenvalue problems (where
resonant cavities are considered) as well as driven problems in both
two and three dimensions. Details can be found in~\citet{Jin2002}
and~\citet{PelosiCoccioliSelleri1998}.

This chapter has also illustrated the ease with which complex formulations
can be implemented and how quickly solutions can be obtained. This
is largely due to the almost one-to-one correspondence between the
expressions at a formulation level and the high-level \fenics{} code
that is used to implement a particular solution. Even in cases where
the required functionality is limited or missing, the use of \fenics{}
in conjunction with external packages greatly reduces development time.

