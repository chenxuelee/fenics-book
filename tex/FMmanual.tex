% Title page
\thispagestyle{empty}

\noindent
{\Huge The FEniCS Manual}

\vspace{6ex}

\noindent
{\Large
Excerpts from the FEniCS Book}

\vspace{6ex}

\noindent
{\large With contributions from}

\vspace{2ex}

\noindent
\bigskip
{\Large Martin Sandve Aln\ae{}s} \\
\bigskip
{\Large Johan Hake} \\
\bigskip
{\Large Robert C. Kirby} \\
\bigskip
{\Large Hans Petter Langtangen} \\
\bigskip
{\Large Anders Logg} \\
\bigskip
{\Large Garth N. Wells}

\vfill

\noindent
{\large Version \today}
\hfill
\includegraphics[width=0.55\textwidth]{png/fenics_logo_text.png}

% Copyright page
\newpage
\thispagestyle{empty}

\null
\vfill

\noindent

\vspace{2ex}

\noindent
{\footnotesize Copyright \copyright\ \the\year\ The FEniCS Project.}

\vspace{2ex}

\noindent
{\footnotesize
Permission is granted to copy, distribute and/or modify this document
under the terms of the GNU Free Documentation License, Version 1.3 or
any later version published by the Free Software Foundation; with no
Invariant Sections, no Front-Cover Texts, and no Back-Cover Texts.
\index{license}

% Table of contents
\tableofcontents

% Preface
\chapter*{Preface}

\thispagestyle{empty}

The FEniCS Project set out in 2003 with an idea to automate the
solution of mathematical models based on differential equations.
Initially, the FEniCS Project consisted of two libraries: DOLFIN and
FIAT. Since then, the project has grown and now consists of the core
components DOLFIN, FFC, FIAT, Instant, UFC and UFL. Other FEniCS
components and applications described in this book are SyFi/SFC,
FErari, ASCoT, Unicorn, CBC.Block, CBC.RANS, CBC.Solve and DOLFWAVE.

This book is written by researchers and developers behind the FEniCS
Project. The presentation spans mathematical background, software
design and the use of FEniCS in applications. The mathematical
framework is outlined in Part~I, the implementation of central
components is described in Part~II, while Part~III concerns a wide
range of applications. New users of FEniCS may find the tutorial
included as the opening chapter particularly useful.

Feedback on this book is welcome, and can be given at
\url{https://launchpad.net/fenics-book}. Use the Launchpad system to
file bug reports if you find errors in the text. For more information
about the FEniCS Project, access to the software presented in this
book, documentation, articles and presentations, visit the FEniCS
Project web site at \url{http://fenicsproject.org}. Some of the
chapters in this book are accompanied by supplementary material in the
form of code examples. These code examples can be downloaded from
\url{http://fenicsproject.org/book/}.

\vspace{1em}

\noindent
Anders Logg, Kent-Andre Mardal and Garth N. Wells \\*
\emph{Oslo and Cambridge, October 2011}

\vspace{1em}

\noindent
{\it This document (``The FEniCS Manual'') contains excerpts from the
book ``Automated Solution of Differential Equations by the Finite
Element Method'' (``The FEniCS Book''). If you like this manual, buy
the book.}

% Cross references
\index{Dirichlet boundary condition|see{boundary condition}}
\index{Neumann boundary condition|see{boundary condition}}
\index{Robin boundary condition|see{boundary condition}}
\index{Lagrange finite element|see{finite element}}
\index{CG element|see{finite element}}
\index{discontinuous Lagrange element|see{finite element}}
\index{DG element|see{finite element}}
\index{Crouzeix--Raviart element|see{finite element}}
\index{Raviart--Thomas element|see{finite element}}
\index{Brezzi--Douglas--Marini element|see{finite element}}
\index{Mardal--Tai--Winther element|see{finite element}}
\index{Arnold--Winther element|see{finite element}}
\index{N\'ed\'elec element|see{finite element}}
\index{Argyris element|see{finite element}}
\index{Hermite element|see{finite element}}
\index{Morley element|see{finite element}}
\index{Bubble element|see{finite element}}
\index{finite element assembly|see{assembly}}
\index{goal oriented error estimate|see{error estimate}}
\index{a priori error estimate|see{error estimate}}
\index{a posteriori error estimate|see{error estimate}}
\index{JIT|see{just-in-time compilation}}
\index{AD|see{automatic differentiation}}
\index{forward mode AD|see{automatic differentiation}}
\index{reverse mode AD|see{automatic differentiation}}
\index{XFEM|see{extended finite element method}}
\index{SUPG|see{stabilization}}
\index{PDE|see{partial differential equation}}
\index{nonlinear PDE|see{partial differential equation}}
\index{time-dependent PDE|see{partial differential equation}}
\index{XML format|see{file formats}}
\index{IPCS|see{incremental pressure correction}}
\index{Ciarlet finite element definition|see{finite element}}
\index{CSS|see{consistent splitting scheme}}
\index{LBB conditions|see{Ladyzhenskaya--\babuska{}--Brezzi conditions}}
\index{multilinear form|see{form}}
\index{Navier--Stokes|see{incompressible Navier--Stokes equations}}
