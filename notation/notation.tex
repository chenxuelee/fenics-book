\documentclass{article}

\newcommand{\inner}[2]{\langle #1, #2 \rangle}

\begin{document}

\linespread{1.2}

\section*{General}

Justify all text

Use dollars for math

Remove all chapter macros or possibly add prefixed by chapter

Refer to other chapters

No psfrag

Text for author list

Code environments: begin{python}, begin{c++}

Space-time slab S_n = \mathcal{T}\times I_n

\begin{itemize}
\item
  Titles of chapters, sections etc. should \emph{not} be capitalized
\item
  Only number referenced equations
\end{itemize}

\section*{Variable names}

\begin{tabular}{ccl}
  $T$ &--&
  cell (note: changed from $K$) \\
  $A_T$ &--&
  element tensor (note: changed from $A^K$) \\
\end{tabular}

\section*{Math notation}

\begin{tabular}{ccl}
  $\inner{v}{w}$ &--&
  inner product, use command \texttt{inner} \\
\end{tabular}

\section*{Other notation}

\begin{tabular}{ccl}
  Use \emph{subcomponent}, \emph{subelement} instead of
  \emph{sub component}, \emp{sub element} etc., cf.
  http://www.thefreedictionary.com/subcomponent.
\end{tabular}

\section*{Useful macros}

\begin{itemize}
\item
  Use \texttt{eqref} to refer to equations
\end{itemize}

\section*{Things to discuss}

\begin{itemize}
\item
  $a(v, u)$ or $a(u, v)$?
\item
  British or American English?
\end{itemize}

\end{document}
\fenicsappendix{Notation}{Notation}

%------------------------------------------------------------------------------

The following notation is used throughout this book.

\small
\linespread{1.2}
  \begin{longtable}{ccl}
    $A$ &--&
    the \emph{global tensor} with entries $\{A_i\}_{i\in\mathcal{I}}$ \\
    $A^K$ &--&
    the \emph{element tensor} with entries $\{A^K_i\}_{i\in\mathcal{I}_K}$ \\
    $A^0$ &--&
    the \emph{reference tensor} with entries $\{A^0_{i\alpha}\}_{i\in\mathcal{I}_K,\alpha\in\mathcal{A}}$ \\
    $a$ &--&
    a multilinear form \\
    $a_K$ &--&
    the local contribution to a multilinear form $a$ from a cell $K$ \\
    $\mathcal{A}$ &--&
    the set of \emph{secondary indices} \\
    $\mathcal{B}$ &--&
    the set of \emph{auxiliary indices} \\
    $e$ &--&
    the \emph{error}, $e = u_h - u$ \\
    $F_K$ &--&
    the mapping from the reference cell $K_0$ to $K$ \\
    $G_K$ &--&
    the \emph{geometry tensor} with entries $\{G_K^{\alpha}\}_{\alpha\in\mathcal{A}}$ \\
    $\mathcal{I}$ &--&
    the set $\prod_{j=1}^{\rho} [1,N^j]$ of indices for the global tensor $A$ \\
    $\mathcal{I}_K$ &--&
    the set $\prod_{j=1}^{\rho} [1,n_K^j]$ of indices for the element tensor $A^K$ (\emph{primary indices}) \\
    $\iota_K$ &--&
    the \emph{local-to-global mapping} from $[1,n_K]$ to $[1,N]$ \\
    $K$ &--&
    a \emph{cell} in the mesh~$\mathcal{T}$ \\
    $K_0$ &--&
    the \emph{reference cell} \\
    $L$ &--&
    a linear form (functional) on $\hat{V}$ or $\hat{V}_h$ \\
    $\mathcal{L}$ &--&
    the degrees of freedom (linear functionals) on $V_h$ \\
    $\mathcal{L}_K$ &--&
    the degrees of freedom (linear functionals) on $\mathcal{P}_K$ \\
    $\mathcal{L}_0$ &--&
    the degrees of freedom (linear functionals) on $\mathcal{P}_0$ \\
    $N$ &--&
    the dimension of $\hat{V}_h$ and $V_h$ \\
    $n_K$ &--&
    the dimension of $\mathcal{P}_K$ \\
    $\ell_i$ &--&
    a degree of freedom (linear functional) on $V_h$ \\
    $\ell^K_i$ &--&
    a degree of freedom (linear functional) on $\mathcal{P}_K$ \\
    $\ell^0_i$ &--&
    a degree of freedom (linear functional) on $\mathcal{P}_0$ \\
    $\mathcal{P}_K$ &--&
    the local function space on a cell $K$ \\
    $\mathcal{P}_0$ &--&
    the local function space on the reference cell $K_0$ \\
    $P_q(K)$ &--&
    the space of polynomials of degree $\leq q$ on $K$ \\
    $r$ &--&
    the (weak) residual, $r(v) = a(v, u_h) - L(v)$ or $r(v) = F(u_h; v)$ \\
    $u_h$ &--&
    the finite element solution, $u_h \in V_h$ \\
    $U$ &--&
    the vector of degrees of freedom for $u_h = \sum_{i=1}^N U_i \phi_i$ \\
    $u$ &--&
    the exact solution of a variational problem, $u \in V$ \\
    $\hat{V}$ &--&
    the test space \\
    $V$ &--&
    the trial space \\
    $\hat{V}^*$ &--&
    the dual test space, $\hat{V}^* = V_0$ \\
    $V^*$ &--&
    the dual trial space, $V^* = \hat{V}$ \\
    $\hat{V}_h$ &--&
    the discrete test space \\
    $V_h$ &--&
    the discrete trial space \\
    $\phi_i$ &--&
    a basis function in $V_h$ \\
    $\hat{\phi}_i$ &--&
    a basis function in $\hat{V}_h$ \\
    $\phi_i^K$ &--&
    a basis function in $\mathcal{P}_K$ \\
    $\Phi_i$ &--&
    a basis function in $\mathcal{P}_0$ \\
    $z$ &--&
    the \emph{dual solution}, $z \in V^*$ \\
    $\mathcal{T}$ &--&
    the \emph{mesh}, $\mathcal{T} = \{K\}$ \\
    $\Omega$ &--&
    a bounded domain in~$\R^d$ \\
  \end{longtable}
\linespread{1.0}
* Anders:  Import all bibliographies OK
* Anders:  Update notation appendix OK

* Editors: Check notation appendix
* Editors: Spell check
* Editors: Choose environment for code listings

* Authors: Use common bibliography in all chapters
* Authors: Prefix all labels by chapter
* Authors: Use common width in figures: \smallwidth, \largewidth
This repository contains all files for the FEniCS book.

Instructions for authors
------------------------

* All files for chapter [foo] are located in the subdirectory chapters/foo.

* Write the chapter in chapters/foo/chapter.tex.

* Put SVG and EPS files under

    chapters/foo/svg
    chapters/foo/eps

* References are stored in the common file bibliography.bib. If you find
  that a reference is missing, submit it to one of the editors in BibTeX
  format and it will be imported into the common bibliography.

  Most (all?) references used in the currently included chapters have been
  imported into the common bibliography using JabRef with keys [authors2][year]
  but the citations have not been updated in all chapters.

* All labels inside a chapter should be prefixed by the chapter prefix.
  For chapter [foo], the following labels should be used:

    \label{foo:eq:bar}  for equations
    \label{foo:fig:bar} for figures
    \label{foo:alg:bar} for algorithms
    \label{foo:sec:bar} for sections

* All chapters are automatically labeled by their prefix.
  To refer to another chapter named [foo], use

    In Chapter~\ref{chap:foo}, ...

* Only use EPS figures. We rely on psfrag for replacement of text
  in figures and so cannot use pdflatex.

* Use psfrag to replace axis labels etc with LaTeX:

    \psfrag{x}{$x$}
    \psfrag{y}{$y$}

* Use Inkscape to draw figures. You can use one of the SVG files from
  chapter [kirby-6] as a template for line widths and colors.

* Use a spell checker (American English).

* Use \index{} to include important terms in the book index.

* Build the book by typing 'make' (for DVI) or 'make final' (for PDF).

* Send comments and questions to fenics-book@fenics.org.

* This repository is primarily intended for editing. Developing the
  text itself should preferrably be done offline (or online in another
  repository) as not to flood the mailing list with very frequent
  commits for all 40 chapters.

Anders Logg <logg@simula.no>
Kent-Andre Mardal <kent-and@simula.no>
Garth N. Wells <gnw20@cam.ac.uk>
