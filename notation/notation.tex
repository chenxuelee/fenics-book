\documentclass{article}

\usepackage{url}

\title{Instructions for FEniCS book authors}
\date{\today}

\begin{document}

\maketitle

\section*{General instructions}

\begin{itemize}
\item
  Carefully read the comments from the referees and editors and update
  your chapter accordingly. Send a detailed response to the referee
  comments along with your revised chapter.
\item
  Make sure to run a spell-checker (American English) before you
  submit your chapter.
\item
  Make sure that all references are complete and accurate and that
  your chapter compiles without warnings.
\item
  Submit your book chapter as a patch against the latest book
  repository. Note in particular that the style files for the book
  have changed, so changes should be make relative to the latest
  version of your chapter in the book repository (not your local
  copy). Note that your chapter may already have gone through some
  minor modifications in order to compile against the new style files.
\item
  Along with your submission, supply a short text for your
  affiliation, including your email address and any funding
  acknowledgments. Your affiliation will be added to an appendix of
  the book that lists the affiliations of all authors.
\item
  Make sure to browse through all other chapters and add appropriate
  references to other relevant chapters.
\end{itemize}

\section*{Spectific typesetting instructions}

\begin{itemize}
\item
  Avoid excessive use of \LaTeX{} macros in chapters. Not only does it
  complicate editing if for example \verb|\begin{itemize}| is replaced
  by \verb|\bit|, but it may also conflict with macros defined in
  other chapters. If it is necessary to define macros, then prefix the
  macros with the chapter prefix, or ask the editors to add them
  globally.
\item
  Justify your text (line-breaking) and make sure it looks clean. In
  Emacs, this can be handled by \verb|M-x auto-fill-mode| or pressing
  \verb|M-q| on each paragraph.
\item
  Use \verb|~| in references and citations: \verb|... in~\ref{...}|,
  \verb|... in~\cite{...}|.
\item
  Chapters should be referred to using \verb|Chapter~\ref{chap:prefix}|,
  where \verb|prefix| is the chapter prefix.
\item
  Use \verb|\eqref{}| to refer to equations. A typical usage would be \\
  \verb|It follows from~\eqref{eq:hoffman-1:ns} that...|.
\item
  All chapter labels should be prefixed with the type of label and the
  chapter prefix, for example \verb|fig:hoffman-1:foo| or
  \verb|tab:narayanan:bar|.
\item
  Use the environments \verb|\begin/end{python}| and
  \verb|\begin/end{c++}| to typeset Python and C++ code.
\item
  The book must be built using \verb|pdflatex| (to work with the new
  style files). This means that EPS figures can no longer be
  used. Instead, PDF figures should be used. All EPS figures in the
  book repository have been converted to PDF using \verb|epstopdf|,
  which preserves vector graphics. Make sure all images in your
  chapter are supplied as PDF files and, where possible, in a scalable
  vector graphics format (no bitmaps). It follows that
  \verb|\psfrag{}| can no longer be used so all images that rely on
  \verb|\psfrag{}| must be replaced by appropriate PDF images.
\item
  Supply original image files (SVG) where possible to enable editing
  of images if this becomes necessary.
\item
  Use Inkscape to draw figures. You can use one of the SVG files from
  chapter [kirby6] as a template for line widths and colors.
\item
  Where possible, use \verb|width=\smallfig| or \verb|width=\largefig|
  to specify image sizes.
\item
  Place image files in subdirectories named \verb|pdf|, \verb|svg|
  etc. inside the chapter subdirectory.
\item
  Use \verb|$...$| for inline math, not \verb|\(...\)|. Use of
  \verb|\(...\)| in captions breaks the new style files.
\item
  Don't put \verb|\label{...}| inside \verb|\caption{...}|. This
  breaks the new style files.
\item
  Titles of chapters, sections etc. should \emph{not} be capitalized
  (with the exception of names and the first letter).
\item
  Only number referenced equations. Use \verb|\begin/end{displaymath}|
  or \verb|\begin/end{equation*}| until you find out that you actually
  need to reference an equation.
\item
  Use \emph{subcomponent}, \emph{subelement} instead of \emph{sub
    component}, \emph{sub element} etc.,
  cf. \url{http://www.thefreedictionary.com/subcomponent}.
\item
  Use \verb|\index{}| to include important terms in the book index.
\item
  Consider the list of variable names given below and try to follow it
  where possible.
\end{itemize}

\section*{Names of variables}


\section*{Useful macros}


\end{document}

\section*{Names of variables}


\small \linespread{1.2}
  \begin{longtable}{ccl}
    $A$ &--&
    the \emph{global tensor} with entries $\{A_i\}_{i\in\mathcal{I}}$ \\
    $A^K$ &--&
    the \emph{element tensor} with entries $\{A^K_i\}_{i\in\mathcal{I}_K}$ \\
    $A^0$ &--&
    the \emph{reference tensor} with entries $\{A^0_{i\alpha}\}_{i\in\mathcal{I}_K,\alpha\in\mathcal{A}}$ \\
    $a$ &--&
    a multilinear form \\
    $a_K$ &--&
    the local contribution to a multilinear form $a$ from a cell $K$ \\
    $\mathcal{A}$ &--&
    the set of \emph{secondary indices} \\
    $\mathcal{B}$ &--&
    the set of \emph{auxiliary indices} \\
    $e$ &--&
    the \emph{error}, $e = u_h - u$ \\
    $F_K$ &--&
    the mapping from the reference cell $K_0$ to $K$ \\
    $G_K$ &--&
    the \emph{geometry tensor} with entries $\{G_K^{\alpha}\}_{\alpha\in\mathcal{A}}$ \\
    $\mathcal{I}$ &--&
    the set $\prod_{j=1}^{\rho} [1,N^j]$ of indices for the global tensor $A$ \\
    $\mathcal{I}_K$ &--&
    the set $\prod_{j=1}^{\rho} [1,n_K^j]$ of indices for the element tensor $A^K$ (\emph{primary indices}) \\
    $\iota_K$ &--&
    the \emph{local-to-global mapping} from $[1,n_K]$ to $[1,N]$ \\
    $K$ &--&
    a \emph{cell} in the mesh~$\mathcal{T}$ \\
    $K_0$ &--&
    the \emph{reference cell} \\
    $L$ &--&
    a linear form (functional) on $\hat{V}$ or $\hat{V}_h$ \\
    $\mathcal{L}$ &--&
    the degrees of freedom (linear functionals) on $V_h$ \\
    $\mathcal{L}_K$ &--&
    the degrees of freedom (linear functionals) on $\mathcal{P}_K$ \\
    $\mathcal{L}_0$ &--&
    the degrees of freedom (linear functionals) on $\mathcal{P}_0$ \\
    $N$ &--&
    the dimension of $\hat{V}_h$ and $V_h$ \\
    $n_K$ &--&
    the dimension of $\mathcal{P}_K$ \\
    $\ell_i$ &--&
    a degree of freedom (linear functional) on $V_h$ \\
    $\ell^K_i$ &--&
    a degree of freedom (linear functional) on $\mathcal{P}_K$ \\
    $\ell^0_i$ &--&
    a degree of freedom (linear functional) on $\mathcal{P}_0$ \\
    $\mathcal{P}_K$ &--&
    the local function space on a cell $K$ \\
    $\mathcal{P}_0$ &--&
    the local function space on the reference cell $K_0$ \\
    $P_q(K)$ &--&
    the space of polynomials of degree $\leq q$ on $K$ \\
    $r$ &--&
    the (weak) residual, $r(v) = a(v, u_h) - L(v)$ or $r(v) = F(u_h; v)$ \\
    $u_h$ &--&
    the finite element solution, $u_h \in V_h$ \\
    $U$ &--&
    the vector of degrees of freedom for $u_h = \sum_{i=1}^N U_i \phi_i$ \\
    $u$ &--&
    the exact solution of a variational problem, $u \in V$ \\
    $\hat{V}$ &--&
    the test space \\
    $V$ &--&
    the trial space \\
    $\hat{V}^*$ &--&
    the dual test space, $\hat{V}^* = V_0$ \\
    $V^*$ &--&
    the dual trial space, $V^* = \hat{V}$ \\
    $\hat{V}_h$ &--&
    the discrete test space \\
    $V_h$ &--&
    the discrete trial space \\
    $\phi_i$ &--&
    a basis function in $V_h$ \\
    $\hat{\phi}_i$ &--&
    a basis function in $\hat{V}_h$ \\
    $\phi_i^K$ &--&
    a basis function in $\mathcal{P}_K$ \\
    $\Phi_i$ &--&
    a basis function in $\mathcal{P}_0$ \\
    $z$ &--&
    the \emph{dual solution}, $z \in V^*$ \\
    $\mathcal{T}$ &--&
    the \emph{mesh}, $\mathcal{T} = \{K\}$ \\
    $\Omega$ &--&
    a bounded domain in~$\R^d$ \\
  \end{longtable}
\linespread{1.0}


\end{document}


\newcommand{\inner}[2]{\langle #1, #2 \rangle}
\llbracket  \cdot \rrbracket  - jump  terms
k_n - time step
I_n  - time interval of length k_{n-1} - k_{n}
S_n - space-time slab \mathcal{T}\times I_n
R_K - strong residual on element K
R_{\partial K} - strong residual on facet \partial K
u_{hk} - the finite element solution in space-time

Space-time slab S_n = \mathcal{T}\times I_n

\section*{Variable names}

\begin{tabular}{ccl}
  $T$ &--&
  cell (note: changed from $K$) \\
  $A_T$ &--&
  element tensor (note: changed from $A^K$) \\
\end{tabular}

\section*{Math notation}

\begin{tabular}{ccl}
  $\inner{v}{w}$ &--&
  inner product, use command \texttt{inner} \\
\end{tabular}

\section*{Other notation}

\begin{itemize}
\item
  $a(v, u)$ or $a(u, v)$?
\end{itemize}

\end{document}
