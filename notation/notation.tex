\documentclass{article}

\usepackage{url}
\usepackage{longtable}
\usepackage{amssymb}
\usepackage{amsmath}
\usepackage{stmaryrd}

\newcommand{\inner}[2]{\langle #1, #2 \rangle}
\newcommand{\jump}[1]{\llbracket #1 \rrbracket}
\newcommand{\avg}[1]{\langle #1 \rangle}
\newcommand{\dx}{\, \mathrm{d}x}
\newcommand{\ds}{\, \mathrm{d}s}
\newcommand{\dt}{\, \mathrm{d}t}

\title{Instructions for FEniCS book authors} \date{\today}

\begin{document}

\maketitle

\section*{General instructions}

\begin{itemize}
\item
  Carefully read the comments from the referees and editors, and
  update your chapter accordingly. Send a detailed response to the
  referee comments along with your revised chapter.
\item
  The deadline for submitting revised chapters is December 1 2010.
\item
  Make sure to run a spell-checker (American English) before you
  submit your chapter.
\item
  Make sure that all references are complete and accurate and that
  your chapter compiles without warnings.
\item
  Make sure that all code examples are accurate and
  up-to-date. \emph{All code presented in the book should be
    compatible with DOLFIN 0.9.9!} Only minor interface changes (if
  any) are expected before the release of DOLFIN 1.0 (expected in
  January 2011). If further changes will be necessary after the
  submission of your final chapter as a result of changes in the
  DOLFIN interface, the editors will make those changes.
\item
  Submit your book chapter as a patch against the latest book
  repository. Note that the style files for the book have changed, so
  changes should be made relative to the latest version of your
  chapter in the book repository (not your local copy). Note that your
  chapter may already have gone through some minor modifications in
  order to compile against the new style files.
\item
  Check that you have referenced all figures in your chapter and that
  the figures look good. Avoid adding very large image files (in MB).
\item
  Along with your submission, supply a short text for your
  affiliation, including your email address and any funding
  acknowledgments. Your affiliation will be added to an appendix of
  the book that lists the affiliations of all authors.
\item
  Make sure to browse through all other chapters and add appropriate
  references to other relevant chapters.
\item
  Each chapter should begin with a short 1--2 paragraph pre-intro
  that introduces the chapter. Chapters should not have a first
  section named ``Introduction''.
\end{itemize}

\section*{Specific typesetting instructions}

\begin{itemize}
\item
  Avoid excessive use of \LaTeX{} macros in chapters. Not only does it
  complicate editing if for example \verb|\begin{itemize}| is replaced
  by \verb|\bit|, but it may also conflict with macros defined in
  other chapters. If it is necessary to define macros, then prefix the
  macros with the chapter prefix, or ask the editors to add them
  globally.
\item
  Justify your text (line-breaking) and make sure it looks clean. In
  Emacs, this can be handled by \verb|M-x auto-fill-mode| or pressing
  \verb|M-q| on each paragraph.
\item
  Use \verb|~| in references and citations: \verb|... in~\ref{...}|,
  \verb|... in~\cite{...}|.
\item
  Chapters should be referred to using \verb|Chapter~\ref{chap:prefix}|,
  where \verb|prefix| is the chapter prefix.
\item
  Use \verb|\eqref{}| to refer to equations. A typical usage would be:
  \begin{verse}
    \verb|It follows from~\eqref{eq:hoffman-1:ns} that...|.
  \end{verse}
\item
  All chapter labels should be prefixed with the type of label and the
  chapter prefix, for example \verb|fig:hoffman-1:foo| or
  \verb|tab:narayanan:bar|.
\item
  Use the environments \verb|\begin/end{python}| and
  \verb|\begin/end{c++}| to typeset Python and C++ code.
\item
  The book must be built using \verb|pdflatex| (to work with the new
  style files). This means that EPS figures can no longer be
  used. Instead, PDF figures should be used. All EPS figures in the
  book repository have been converted to PDF using \verb|epstopdf|,
  which preserves vector graphics. Make sure all images in your
  chapter are supplied as PDF files and, where possible, in a scalable
  vector graphics format (no bitmaps). It follows that
  \verb|\psfrag{}| can no longer be used so all images that rely on
  \verb|\psfrag{}| must be replaced by appropriate PDF images.
\item
  Supply original image files (SVG) where possible to enable editing
  of images if this becomes necessary.
\item
  Use Inkscape to draw figures. You can use one of the SVG files from
  chapter [kirby-6] as a template for line widths and colors.
\item
  Where possible, use \verb|width=\smallfig| or \verb|width=\largefig|
  to specify image sizes.
\item
  Place image files in subdirectories named \verb|pdf|, \verb|svg|
  etc. inside the chapter subdirectory.
\item
  Use \verb|$...$| for inline math, not \verb|\(...\)|. Use of
  \verb|\(...\)| in captions breaks the new style files.
\item
  Don't put \verb|\label{...}| inside \verb|\caption{...}|. This
  breaks the new style files.
\item
  Titles of chapters, sections etc. should \emph{not} be capitalized
  (with the exception of names and the first letter).
\item
  Only number referenced equations. Use \verb|\begin/end{displaymath}|
  or \verb|\begin/end{equation*}| until you find out that you actually
  need to reference an equation.
\item
  Use \emph{subcomponent}, \emph{subelement} instead of \emph{sub
    component}, \emph{sub element} etc.,
  cf. \url{http://www.thefreedictionary.com/subcomponent}.
\item
  Use \verb|\index{}| to include important terms in the book index.
\item
  Consider the list of variable names given below and try to follow it
  where possible.
\end{itemize}

\section*{Notation}

\begin{itemize}
\item
  The order of test and trial functions has changed. The
  canonical variational is now $a(u, v) = L(v)$. Generally, we write:
  \begin{equation}
    F(f_1, \cdots, f_m; v_0, \dots, v_n)
  \end{equation}
  for a form $F$ that is linear in the variables $v_1, \dots, v_m$ (and
  non-linear in the variables $f_1, \dots, f_m$.)
\item
  Second, the name of a cell/element has changed from $K$ to $T$.
\item
  Don't use bold fonts unless really, really required. In particular,
  do not use bold fonts (nor super-imposed bars/arrows) to denote
  vector fields.
\item
  Use either $\mathrm{grad}$, $\mathrm{curl}$, $\mathrm{div}$ or
  $\nabla$, $\nabla \times$, $\nabla \cdot$, and make sure to follow the
  new row-wise notation used in UFL, that is, $(\mathrm{grad} \, v)_{ij} =
  \partial v_i / \partial x_j$ and $(\mathrm{div} \, v)_i = \partial v_{ij}
  / \partial x_j$. Macros \verb|\Grad|, \verb|\Div|, \verb|\Curl| can
  be used.
\item
  Use $\dx$ for integration, not $\mathrm{d}\Omega$.
\item
  Write the nonlinear term in Navier--Stokes as $\mathrm{grad} u \cdot u$ or
  $\nabla u \cdot u$.
\end{itemize}

\subsection*{Function spaces}

\begin{longtable}{lcl}
  $H^1(\Omega)$ & -- & the Sobolev space $W^{1,2}(\Omega)$ \\
  $H^1_0(\Omega)$ & -- & the Sobolev space $W^{1,2}(\Omega)$, zero boundary trace \\
  $[H^1(\Omega)]^n$ & -- & the Sobolev space $W^{1,2}(\Omega)$, vector-valued with $n$ components \\
  $\mathrm{CG}_q(\Omega)$ & -- & continuous piecewise polynomial function space of degree $q$ \\
  $\mathrm{DG}_q(\Omega)$ & -- & discontinuous piecewise polynomial function space of degree $q$ \\
\end{longtable}

\subsection*{Operators}

\begin{longtable}{lcl}
  $| \cdot |$ & -- & $\mathbb{R}^n$ norm \\
  $\inner{v}{w}$ &--& inner product, use macro \verb|\inner| \\
  $\| \cdot \|_V$ & -- & if $V$ is a normed space: norm on space $V$ \\
  $\| \cdot \|_K$ & -- & if $K$ is a domain: $L^2(K)$ norm \\
  $\| \cdot \|$ & -- & defaults to $L^2(\Omega)$ norm \\
  $\| \cdot \|_0$ & -- & also $L^2(\Omega)$ norm \\
  $\| \cdot \|_1$ & -- & $H^1(\Omega)$ norm \\
  $\| \cdot \|_{\mathrm{div}}$ & -- & $H(\mathrm{div})$ norm \\
  $\| \cdot \|_{\mathrm{curl}}$ & -- & $H(\mathrm{curl})$ norm \\
  $\jump{v}$ & -- & jump, use macro \verb|\jump| \\
  $\avg{v}$ & -- & average, use macro \verb|\avg| \\
\end{longtable}

\subsection*{Names of variables}

\begin{longtable}{ccl}
  $A$ &--&
  the \emph{global tensor} with entries $\{A_i\}_{i\in\mathcal{I}}$ \\
  $A_T$ &--&
  the \emph{element tensor} with entries $\{A_{T,i}\}_{i\in\mathcal{I}_T}$ \\
  $A^0$ &--&
  the \emph{reference tensor} with entries $\{A^0_{i\alpha}\}_{i\in\mathcal{I}_T,\alpha\in\mathcal{A}}$ \\
  $a$ &--&
  a multilinear form \\
  $a_T$ &--&
  the local contribution to a multilinear form $a$ from a cell $T$ \\
  $\mathcal{A}$ &--&
  the set of \emph{secondary indices} \\
  $\mathcal{B}$ &--&
  the set of \emph{auxiliary indices} \\
  %$e$ &--&
  %the \emph{error}, $e = u - u_h$ \\
  $F_T$ &--&
  the mapping from the reference cell $T_0$ to $T$ \\
  $G_T$ &--&
  the \emph{geometry tensor} with entries $\{G_T^{\alpha}\}_{\alpha\in\mathcal{A}}$ \\
  $\mathcal{I}$ &--&
  the set $\prod_{j=1}^{\rho} [1,N_j]$ of indices for the global tensor $A$ \\
  $\mathcal{I}^T$ &--&
  the set $\prod_{j=1}^{\rho} [1,n_j^T]$ of indices for the element tensor $A^T$ (\emph{primary indices}) \\
  $\iota_T$ &--&
  the \emph{local-to-global mapping} from $[1,n_T]$ to $[1,N]$ \\
  $T$ &--&
  a \emph{cell} in the mesh~$\mathcal{T}$ \\
  $T_0$ &--&
  the \emph{reference cell} \\
  $n$ &--&
  unit normal vector (to boundary) \\
  $t$ &--&
  unit tangential vector (to boundary) \\
  $L$ &--&
  a linear form (functional) on $\hat{V}$ or $\hat{V}_h$ \\
  $b$ &--&
  a \emph{global} right-hand side vector, corresponding to $L(\phi_i)$ \\
  $\mathcal{L}$ &--&
  the degrees of freedom (linear functionals) on $V_h$ \\
  $\mathcal{L}_T$ &--&
  the degrees of freedom (linear functionals) on $\mathcal{P}_T$ \\
  $\mathcal{L}_0$ &--&
  the degrees of freedom (linear functionals) on $\mathcal{P}_0$ \\
  %$N$ &--&
  %the dimension of $\hat{V}_h$ and $V_h$ \\
  %$n_T$ &--&
  %the dimension of $\mathcal{P}_T$ \\
  $\ell_i$ &--&
  a degree of freedom (linear functional) on $V_h$ \\
  $\ell^T_i$ &--&
  a degree of freedom (linear functional) on $\mathcal{P}_T$ \\
  $\ell^0_i$ &--&
  a degree of freedom (linear functional) on $\mathcal{P}_0$ \\
  %$\mathcal{P}_T$ &--&
  %the local function space on a cell $T$ \\
  %$\mathcal{P}_0$ &--&
  %the local function space on the reference cell $T_0$ \\
  $\mathcal{P}_q(T)$ &--&
  the space of polynomials of degree $\leq q$ on a domain $T$ \\
  %$r$ &--&
  %the (weak) residual, $r(v) = a(v, u_h) - L(v)$ or $r(v) = F(u_h; v)$ \\
  $u$ &--&
  the exact solution of a variational problem, $u \in V$ \\
  $u_h$ &--&
  the finite element solution, $u_h \in V_h$ \\
  $U$ &--&
  the vector of degrees of freedom for $u_h = \sum_{i=1}^N U_i \phi_i$ \\
  $V$ &--&
  the trial space \\
  $\hat{V}$ &--&
  the test space \\
  %$\hat{V}^*$ &--&
  %the dual test space, $\hat{V}^* = V_0$ \\
  $V^*$ &--&
  the dual space (as in duality-based error analysis) of a space $V$ \\
  $V_h$ &--&
  the discrete trial space \\
  $\hat{V}_h$ &--&
  the discrete test space \\
  $\phi_i$ &--&
  a basis function in $V_h$ \\
  $\hat{\phi}_i$ &--&
  a basis function in $\hat{V}_h$ \\
  %$\phi_i^T$ &--&
  %a basis function in $\mathcal{P}_T$ \\
  %$\Phi_i$ &--&
  %a basis function in $\mathcal{P}_0$ \\
  %$z$ &--&
  %the \emph{dual solution}, $z \in V^*$ \\
  $\mathcal{T}_h$ &--&
  the \emph{mesh}, $\mathcal{T}_h = \{T\}$ \\
  $\Omega$ &--&
  a bounded domain in~$\mathbb{R}^d$ \\
  $\partial \Omega$ &--&
  the boundary of $\Omega$ \\
  $k_n$ &--&
  time step \\
  $I_n$ &--&
  time interval of length $k_{n} - k_{n-1}$ \\
  $u_{hk}$ &--&
  the finite element solution in space-time \\
  $S_n$ &--&
  space-time slab $\mathcal{T} \times I_n$ \\
  $\dx, \ds, \dt$ &--&
  integration, use macros \verb|\dx|, \verb|\ds|, \verb|\dt| \\
  $X$�& -- & coordinates in a reference frame \\
  $x$�& -- & coordinates in a physical space \\
\end{longtable}

\end{document}
